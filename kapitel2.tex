\chapter{Kombinatorik} 

\section{Basics} 

\begin{bem}
	Die Haupfrage der Kombinatorik ist ``Wie viele Elemente hat meine endliche Menge?''. Etwas formaler geht es um die Formeln für die Anzahl der Elemente verschiedener endlicher Mengen, die man in der diskreten Mathematik gerne benutzt. 
\end{bem} 

\begin{lem} \label{lem:vereinigung:zwei}
	Seien $A, B$ endliche disjunkte Mengen. Dann ist $|A \cupdot B| = |A| + |B|$. 
\end{lem} 

\begin{proof}
	Ist $A$ oder $B$ leer, so gilt die Formel trivialerweise: etwa bei $B = \emptyset$ gilt $|A \cup B| = | A \cup \emptyset| = |A| = |A| + 0 = |A| + | \emptyset| = |A| + |B|$. 
	
	 Sonst nummerieren wir alle Elemente von $A$ als $a_1,\ldots,a_s$ und $B$ als $b_1,\ldots,b_t$, das heißt, $A$ ist eine $s$-Elementige Menge $A = \{a_1,\ldots,a_s\}$ und $B$ ist eine $t$-elementige Menge $B = \{b_1,\ldots,b_t\}$ mit $s,t \in \N$. Es gilt $a_i \ne a_j$ für $i \ne j$ mit $i,j \in \{1,\ldots,s\}$ und $b_i \ne b_j$ für $i \ne j$ mit $i,j \in \{1,\ldots,t\}$. Da $A$ und $B$ disjunkt sind gilt auch $a_i \ne a_j$ für alle $i \in \{1,\ldots,s\}$ und $j \in \{1,\ldots,t\}$. Somit ist 
	 $A \cupdot B = \{a_1,\ldots,a_s,b_1,\ldots,b_t\}$, sodass wir $A \cupdot B = \{c_1,\ldots,c_n\}$ mit $n = s + t$, $c_i = a_i$ für $i \in\{1,\ldots,s\}$ und $c_i = b_{i-s}$ für $i \in \{s+1,\ldots,n\}$ haben. Hierbei sind $c_1,\ldots,c_n$ nach der Konstruktion paarweise verschieden. Das zeigt, dass $A \cupdot B$ genau $n = s+t$ Elemente hat. 
\end{proof}

\begin{lem} \label{lem:disjunkte:vereinigung}
	Seien $A_1,\ldots,A_n$ ($n \in \N$) endliche paarweise disjunkte Mengen. Dann ist 
	\[
		\left| \bigcup_{i=1}^n A_i \right| = \sum_{i=1}^n |A_i|. 
	\]
\end{lem} 

\begin{proof} 
	Wir beweisen die die Formel durch Induktion über $n$. 
	Die Formel ist rivial für $n=1$, denn $\bigcup_{i=1}^1 A_i = A_1$ und $\sum_{i=1}^n |A_i|$ ist $|A_1|$. 
	Sei $n \in \N$ mit $n \ge 2$ gegeben und sei die Formel im Fall von $n-1$ an der Stelle von $n$ Mengen bereits verifiziert. Da die Mengen $A_1 \cup \cdots \cup A_{n-1}$ und $A_n$ paarweise disjunkt sind, erhalten wir durch die Anwendung von Lemma~\ref{lem:vereinigung:zwei} zu diesen beiden Mengen, dass 
	\[
		| A_1 \cup \cdots \cup A_n| = |A_1 \cup \cdots \cup A_{n-1} | + |A_n| 
	\]
	erfüllt ist. Aus der Induktionsvoraussetzung folgt, dass 
	\[
		| A_1 \cup \cdots \cup A_{n-1} | = \sum_{i=1}^{n-1} |A_i|
	\]
	erfüllt ist. Somit ist 
	\[
		| A_1 \cup \cdots \cup A_n| = \sum_{i=1}^{n-1} |A_i|  + |A_n| = \sum_{i=1}^n |A_i|. 
	\]
\end{proof} 

\begin{lem} \label{lem:inkl:exkl:2}
	Seien $A$ und $B$ endliche Mengen. Dann gilt 
	\[
		|A \cup B| = |A| + |B| - |A \cap B|. 
	\]
\end{lem}

\begin{proof}
		Wir können $A \cup B$ als dijsunkte Vereinigung von $A$ und $B \setminus A$ darstellen. Die Anwendung von Lemma~\ref{lem:vereinigung:zwei} zu $A$ und $B \setminus A$ ergibt 
		\[
			| A \cup B| = |A \cupdot (B \setminus A)| = |A | + |B \setminus A|. 
		\]
		Die Menge $B$ ist disjunkte Vereinigung von $B \setminus A$ und $A \cap B$. Die Anwendung von Lemma~\ref{lem:vereinigung:zwei} zu $A \cap B$ und $B \setminus A$ ergibt
		\[
			|B| = |B \setminus A| + |A \cap B|. 
		\]
		Aus den beiden Gleichungen, die wir auf diese Weise herleiten, folgt dann 
		\[
			| A \cup B|  = |A | + |B \setminus A| = |A| + (|B| - | A \cap B|) = |A| + |B| - | A \cap B|. 
		\]
\end{proof} 

\begin{bem}
	Die Intuition hinter Lemma~\ref{lem:inkl:exkl:2} ist: wir zählen alle Elemente in $A$ sowie $B$ ab. Dadurch werden die Elemente in $A \cap B$ doppelt abgezählt. Wir sollen also die Anzahl der Elemente in $A \cap B$ abbziehen, um auf die Anzahl der Elemente in $A \cup B$ zu kommen. 
\end{bem} 

\begin{lem} \label{lem:anzahl:prod:2}
	Seien $A$ und $B$ endliche Menge. Dann gilt: 
	\[
		|A \times B| = |A| \cdot |B|. 
	\]
\end{lem}
\begin{proof} 
	Ist $A$ oder $B$ leer, so sind die linke sowie rechte Seite der Gleichung gleich $0$. Ansonsten stellen wir die Menge $A \times B$ kann als disjunkte Vereinigung $\bigcup_{a \in A} \{a\} \times B$ da. Lemma~\ref{lem:disjunkte:vereinigung} ergibt
	\[
		|A \times B| = \left| \bigcup_{a \in A} \{a \} \times B \right| = \sum_{a \in A} | \{a \} \times B |.
	\]
	Für ein beliebiges festes $a \in A$ kann nun die Anzahl der Elemente in $\{a\} \times B$ bestimmt werden. Diese Anzahl ist $|B|$, da die Abbildung $f_a : B \to \{a\} \times B$ mit $f_a(b) := (a,b)$ bijektiv ist: denn hat man zwei verschiedene Elemente $b',b'' \in B$ so sind auch $(a,b')$ und $(a,b'')$ verschieden (Injektivität) und hat man ein beliebiges Element aus $\{a\} \times B$ fixiert, etwa $(a,b)$ mit $b \in B$, so erhält man dieses Element als $f_a(b) = (a,b)$. 
\end{proof} 
 

\section{$X^k$ und $B^A$} 

\begin{thm}
	Seien $A_1,\ldots, A_n$ endliche Mengen ($n \in \N$). Dann ist 
	\[
		|A_1 \times \cdots \times A_n|  = \prod_{i=1}^n |A_i|. 
	\]
\end{thm} 
\begin{proof}
	Wir beweisen die Gleichung durch Induktion über $n$.
	Für $n=1$ erhalten wir eine triviale Identät. Sei die Gleichung für $n-1$ Mengen für ein $n \in \N$ mit $n \ge 2$ efüllt. Dann erhält man wegen 
	\[
			A_1 \times \cdots \times A_n = A_1 \times (A_2 \times \cdots \times A_n)
	\]
	durch die Anwendung von Lemma~\ref{lem:anzahl:prod:2} die Gleichung 
	\[
		|A_1 \times \cdots \times A_n | = |A_1| \cdot |A_2 \times \cdots \times A_n|. 
	\]
	Anschließend erhalten wir aus der Induktionsvoraussetzung 
	\[
		|A_2 \times \cdots \times A_n| = \prod_{i=2}^n |A_i|,
	\]
	woraus sich die gewünschte Gleichung für die Mengen $A_1,\ldots,A_n$ ergibt. 
\end{proof} 

\begin{bem}[Das Selbe oder das Gleiche?]
	
\end{bem} 

\begin{thm}
	Sei $X$ eine endliche Menge und sei $k \in \N_0$. Dann ist $|X^k| = |X|^k$. (In dieser und den nachfolgenden kombinatorischen Formeln interpretieren wir $0^0$ als $1$.)
\end{thm} 
\begin{proof} 
	Die Formel ist trivial für in den entarteten Fällen $|X|=0$ und $k=0$.
	(für $k>0$ und $X = \emptyset$ ist $X^k$ ebenfalls die leere Menge und für $k=0$  besteht $X^k$ aus dem einzigen $0$-Tupel).  Wir nehmen also $|X| >0$ und $k>0$ an und beweisen  die Gleichung $|X^k| = |X|^k$  in diesem Fall durch Induktion über $k$. 
	
	Die Formel ist trivial für $k = 1$: es gilt $|X^1| = |X|$ wegen $X^1 = X$. Sei $k \ge 2$ und sei die Formel $|X^{k-1}| = |X|^{k-1}$ bereits verifiziert. 
	
	Sei $n:=|X|$. Dann ist $X^k$ die disjunkte Vereinigung 
	\[
			X^k = \bigcup_{a \in X} X^{k-1} \times \{a\}. 
	\]
	der $n$ Mengen $X^{k-1} \times \{a\}$. Mit anderen Worten zerlegen wir $X^k$ in $n$ paarweise disjunkte Mengen, indem wir $n$ verschiedene Möglichkeiten für die Wahl der letzten Komponente eines $k$-Tupels aus $X^k$ unterscheiden. Nach Lemma~\ref{lem:disjunkte:vereinigung} gilt dann 
	\[
			|X^k| = \sum_{a \in X} |X^{k-1} \times \{a\}|. 
	\]
	Es bleibt, für jedes feste $a \in X$, die Anzahl der Elemente in $X^{k-1} \times \{a\}$ zu bestimmen. Die Abbildung 
	$f_a : X^{k-1} \times \{a\} \to X^{k-1}$ mit $f_a(x_1,\ldots,x_{k-1},a) = (x_1,\ldots,x_k)$, welche die letzte Komponente des Tupels $(x_1,\ldots,x_{k-1},a)$ weglässt ist eine Bijektion (begründen Sie kurz, warum). Daher hat $X^{k-1} \times \{a\}$ für jede Wahl von $a$ genauso viele Elemente wie $X^{k-1}$. Es folgt 
	\[
			|X^k| = \sum_{a \in X} |X^{k-1} \times \{a\}| = \sum_{a \in X} |X^{k-1}|. 
	\]
	Nach der Induktionsvoraussetzung erhalten wir $|X^{k-1}| = |X|^{k-1}$. Da die Summe über die $|X|$-elementige Menge geht, erhalten wir 
	\[
			|X^k| = |X| \cdot |X^{k-1}| = |X|^k. 
	\]
\end{proof} 


\begin{thm}
	Seien $A,B $ endliche Mengen. Dann gilt $|B^A| = |B|^{|A|}$. 
\end{thm}
\begin{proof}
	Die Formel ist trivial, wenn $A$ oder $B$ leer ist. Seien $A$ und $B$ nicht leer. Sei Sei $|A| = k$ und $A = \{a_1,\ldots,a_k\}$. Dann ist die Abbildung $B^A \to B^k$, die jedem $f : A \to B$ das Tupel $(f(a_1),\ldots,f(a_k))$ eine Bijektion. Somit gilt 
	$|B^A | = |B^k| = |B|^k = |B|^{|A|}$.  
\end{proof} 

\section{$\binom{X}{k}$}

\section{Zählen der bijektiven und injektiven Abbildungen} 