\chapter{Kombinatorik} 

\section{Basics} 

\begin{lem}
	Seien $A, B$ endliche disjunkte Mengen. Dann ist $|A \cupdot B| = |A| + |B|$. 
\end{lem} 

\begin{lem} \label{lem:disjunkte:vereinigung}
	Seien $A_1,\ldots,A_n$ endliche paarweise disjunkte Mengen. Dann ist 
	\[
		\left| \bigcup_{i=1}^n A_i \right| = \sum_{i=1}^n |A_i|. 
	\]
\end{lem} 

\begin{lem}
	Seien $A$ und $B$ endliche Mengen. Dann gilt 
	\[
		|A \cup B| = |A| + |B| - |A \cap B|. 
	\]
\end{lem}

\begin{lem}
	Seien $A$ und $B$ endliche Menge. Dann gilt: 
	\[
		|A \times B| = |A| \cdot |B|. 
	\]
\end{lem}
 

\section{$X^k$ und $B^A$} 



\begin{thm}
	Sei $X$ eine endliche Menge und sei $k \in \N_0$. Dann ist $|X^k| = |X|^k$. (In dieser und den nachfolgenden kombinatorischen Formeln interpretieren wir $0^0$ als $1$.)
\end{thm} 
\begin{proof} 
	Die Formel ist trivial für in den entarteten Fällen $|X|=0$ und $k=0$.
	(für $k>0$ und $X = \emptyset$ ist $X^k$ ebenfalls die leere Menge und für $k=0$  besteht $X^k$ aus dem einzigen $0$-Tupel).  Wir nehmen also $|X| >0$ und $k>0$ an und beweisen  die Gleichung $|X^k| = |X|^k$  in diesem Fall durch Induktion über $k$. 
	
	Die Formel ist trivial für $k = 1$: es gilt $|X^1| = |X|$ wegen $X^1 = X$. Sei $k \ge 2$ und sei die Formel $|X^{k-1}| = |X|^{k-1}$ bereits verifiziert. 
	
	Sei $n:=|X|$. Dann ist $X^k$ die disjunkte Vereinigung 
	\[
			X^k = \bigcup_{a \in X} X^{k-1} \times \{a\}. 
	\]
	der $n$ Mengen $X^{k-1} \times \{a\}$. Mit anderen Worten zerlegen wir $X^k$ in $n$ paarweise disjunkte Mengen, indem wir $n$ verschiedene Möglichkeiten für die Wahl der letzten Komponente eines $k$-Tupels aus $X^k$ unterscheiden. Nach Lemma~\ref{lem:disjunkte:vereinigung} gilt dann 
	\[
			|X^k| = \sum_{a \in X} |X^{k-1} \times \{a\}|. 
	\]
	Es bleibt, für jedes feste $a \in X$, die Anzahl der Elemente in $X^{k-1} \times \{a\}$ zu bestimmen. Die Abbildung 
	$f_a : X^{k-1} \times \{a\} \to X^{k-1}$ mit $f_a(x_1,\ldots,x_{k-1},a) = (x_1,\ldots,x_k)$, welche die letzte Komponente des Tupels $(x_1,\ldots,x_{k-1},a)$ weglässt ist eine Bijektion (begründen Sie kurz, warum). Daher hat $X^{k-1} \times \{a\}$ für jede Wahl von $a$ genauso viele Elemente wie $X^{k-1}$. Es folgt 
	\[
			|X^k| = \sum_{a \in X} |X^{k-1} \times \{a\}| = \sum_{a \in X} |X^{k-1}|. 
	\]
	Nach der Induktionsvoraussetzung erhalten wir $|X^{k-1}| = |X|^{k-1}$. Da die Summe über die $|X|$-elementige Menge geht, erhalten wir 
	\[
			|X^k| = |X| \cdot |X^{k-1}| = |X|^k. 
	\]
\end{proof} 


\begin{thm}
	Seien $A,B $ endliche Mengen. Dann gilt $|B^A| = |B|^{|A|}$. 
\end{thm}
\begin{proof}
	Die Formel ist trivial, wenn $A$ oder $B$ leer ist. Seien $A$ und $B$ nicht leer. Sei Sei $|A| = k$ und $A = \{a_1,\ldots,a_k\}$. Dann ist die Abbildung $B^A \to B^k$, die jedem $f : A \to B$ das Tupel $(f(a_1),\ldots,f(a_k))$ eine Bijektion. Somit gilt 
	$|B^A | = |B^k| = |B|^k = |B|^{|A|}$.  
\end{proof} 

\section{$\binom{X}{k}$}

\section{Zählen der bijektiven und injektiven Abbildungen} 