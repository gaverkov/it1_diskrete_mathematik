\chapter{Mathematische Grundlagen}

\section{Aussagen und logische Verknüpfungen}


\begin{defn}
Eine Aussage ist ein Satz (eine Folge von Zeichen mit mathematischer Bedeutung), die einen eindeutigen Wahrheitswert (entweder falsch oder wahr) hat. Den Wahrheitswert kodiert man oft mit Zahlen $0$ (falsch) und $1$ (wahr). 
\end{defn} 


\begin{bsp}\ 
\begin{itemize}
	\item $ 2 < 1 $ (falsch)
	\item $ 2 = 1 $ (falsch)
	\item $ 2 > 1 $ (wahr)
	\item $2$ ist eine Primzahl (wahre Aussage, Primzahl definiert). 
	\item $2$ ist eine schöne Zahl (keine Aussage, es sei denn, die Eigenschaft einer Zahl schön zu sein, wurde definiert). 
	\item Es gibt unendlich viele Primzahlen $n$, für die $n+2$ ebenfalls eine Primzahl ist. (eine Aussage, Wahrheitswert ist noch nicht geklärt). 
	\item Die Gleichung $x^2+1 =0$ hat keine Lösungen. (an sich keine Aussage, es sei denn, ein Kontext war vorher gegeben, in dem die Rolle von $x$ geklärt wurde.)
	\item Die Gleichung $x^2+1=0$ hat keine reellwertigen Lösungen (wahre Aussage). 
\end{itemize}
\end{bsp} 

\begin{defn}
Seien $ A $ und $ B $ Aussagen. Dann definiert man anhand von $ A $ und $ B $ die folgenden Aussagen:
\begin{itemize}
	\item $ A \wedge B $ Konjunktion (\glqq und\grqq) ist genau dann wahr, wenn $A$ und $B$ beide wahr sind. 
	\item $ A \vee B $ Disjunktion (\glqq oder\grqq) ist genau dann falsch, wenn $A$ und $B$ beide falsch sind. 
	\item $ A \Rightarrow B $ Implikation ist genau dann falsch, wenn $A$ wahr und $B$ falsch ist. 
	\item $ A \Leftrightarrow B $ Äquivalenz ist genau dann wahr, wenn die Wahrheitswerte von $A$ und $B$ gleich sind. 
	\item $ A \:\dot{\vee}\: B $ ausschließende Disjunktion ist genau dann wahr, wenn die Wahrheitswerte von $A$ und $B$ verschieden sind. 
	\item $ \neg A $ (wir auch als $ \bar{A} $ bezeichnet), Negation (Verneinung) ist genau dann wahr, wenn $A$ falsch ist. 
\end{itemize}
\end{defn}


\begin{bsp}\
	\begin{itemize}
		\item Seien $ x,y \in \R$. Dann gilt die Implikation: $x = y \Rightarrow x^2 = y^2 $ (wahr)
		\item Seien $x,y \in \R$. Dann gilt die Implikation $ x^2 = y^2 \Rightarrow x = y $ nicht für alle $ x,y \in \R $ (falsch für $x=1$ und $y=-1$)	
	\end{itemize}
\end{bsp}

\begin{bem}
	Alternativbezeichnungen für $\Rightarrow$ und $\Leftrightarrow$ sind $\rightarrow$ bzw. $\leftrightarrow$. 
\end{bem} 

\begin{bem}
	Wenn man in Mathe-Argumenten eine Folge von Implikationen benutzt, so schreibt man auch oft kurz so etwas wie $A \Rightarrow B \Rightarrow C$. Damit meint man  $(A \Rightarrow B) \wedge (B \Rightarrow C)$, d.h., aus $A$ folgt $B$ und aus $B$ folgt $C$. Das Gleiche auch für $\Leftrightarrow$. 
\end{bem} 


\begin{bem}
	Die Aussagenlogik ist die Studie der logischen Verknüpfungen von Aussagen. Dabei spielt die Natur der in den Formeln verwendeten Aussagen, die man mit Symbolen bezeichnet, etwa $a,b, c, d, \ldots$, an sich keine Rolle. Alles, was zählt, ist der Wahrheitswert. Daher kann man auch $a,b,c,d \ldots$ als Variablen aus $\{0,1\}$ auffassen, ohne dass sich an der Studie etwas ändert. Mehr über die Aussagenlogik erfahren wir später in diesem Kurs. 
\end{bem} 

\clearpage
\section{Mengen und Mengenoperationen}


\begin{defn}[Intuitive ``Definition'' einer Menge] 
	Eine Menge $X$ ist durch die Eindeutige Angabe definiert, welche Objekte Elemente der Menge sind. Man schreibt in diesem Fall $x \in X$ dafür, dass das Objekt $x$ Element der Menge $X$ ist, und $x \not\in X$ dafür, dass $x$ kein Element der Menge $X$ ist. 
	
	Mit anderen Worten: für die Angabe einer Menge $X$ soll für jedes Objekt $x$ geklärt sein, ob für dieses Objekt $x \in X$ oder $x\not \in X$ gilt. 
\end{defn} 

\begin{bem}[Zur korrekten Definition einer Menge]
	Unsere Definition der Menge ist etwas intuitiv und  somit streng genommen keine echte Definition); sie reicht aber vorerst für unsere Zwecke völlig aus. Die genaue Definition einer Menge ist durch das Axiomensystem von Zermelo-Fraenkel gegeben. Dieses System legt Folgendes fest: 
	\begin{itemize}
		\item die Existenz der leeren Menge, 
		\item die Bedingung für die Gleichheit von zwei Mengen
		\item die Möglichkeit Mengenfamilien zu vereinigen, 
		\item die Existenz einer sogenannten Potenzmenge für eine beliebige Menge
		\item Fundierungsaxiom (ist etwas technisch)
		\item die Möglichkeit Mengen durch eine Bedingung zu definieren. 
		\item Ersetzungsaxiom (ist etwas technisch)
	\end{itemize} 
	Zu den obigen Axiomen nimmt man noch zusätzlich das sogenannte Auswahlaxiom hinzu. Das Axiomensystem, das auf diese Weise entsteht, wird als ZFC (Zermelo-Fraenkel axioms plus Axiom of Choice) abgekürzt. 
\end{bem} 


\begin{bsp}[Definition einer Menge durch endliche/unendliche Auflistung] 
Eine Weise, Mengen zu definieren, ist durch die Auflistung ihrer Elemente. Dabei stehen die geschweiften Klammern für Mengen, die drei Punkte bedeuten \glqq usw\grqq.
\begin{itemize}
	\item $ \{1,2,5,7\} $ - Menge aus den vier Elementen $1,2,5$ und $7$. 
	\item $ \{1\} $ - Menge aus einem einzigen Element $1$. 
	\item $ \{1,\{2,5\},\{6\}\} $ - Menge aus drei Elementen, von denen zwei Elemente selbst Mengen sind. 
	\item $ \{1,2,3,\ldots\} $ - Menge aller positiven ganzen Zahlen. 
\end{itemize}
\end{bsp}

\begin{defn}
Seien $ A $ und $ B $ Mengen. Dann heißt $ A $ eine \textbf{Teilmenge} von $ B $, wenn jedes Element von $ A $ auch Element von $ B $ ist;  wir verwenden in diesem Fall die Bezeichnung $A \subseteq B$ und nennen die Relation $\subseteq$ \textbf{Inklusion}. 

Wenn $A$ Teilmenge von $B$ aber $B$ keine Teilmenge von $A$ ist, so sagt man, dass $A$ eine \textbf{echte Teilmenge} von $B$ ist; wir verwenden in diesem Fall die Bezeichnung $A \varsubsetneq B$ und nennen die Relation $\varsubsetneq$ eine \textbf{echte} bzw. \textbf{strikte Inklusion}. 
\end{defn}

\begin{bem}
In einigen mathematischen Quellen bezeichnet man die Inklusion als $ \subset $ und nicht als $ \subseteq $. Es ist schwer zu sagen, welche Bezeichnung in der Mehrheit der Quellen benutzt wird. Es gibt aber auch Quellen, in denen $ \subset $ die strikte Inklusion bezeichnet. Durch die Nutzung $\subseteq$ vermeiden wir potenzielle Ungenauigkeiten. %Ambiguitäten. 
\end{bem} 

\begin{defn}
	 Zwei Mengen $ A $ und $ B $ heißen \textbf{gleich}, wenn $ A \subseteq B $ und $ B \subseteq A $ gilt. Wir schreiben dann $A = B$.
\end{defn} 


\begin{bem}[Definition einer Menge durch eine Bedingung]
Eine sehr verbreitete Weise, Mengen zu definieren, ist durch Bedingungen, nach dem Format 
\[
 \setcond{ \text{AUSDRUCK} }{\text{BEDINGUNG}}. 
\] Der Doppelpunkt bedeutet so viel wie \glqq sodass\grqq\ oder \glqq mit der Bedingung\grqq. In manchen Quellen wird ein senkrechter Strich an der Stelle des Doppelpunktes benutzt. 
\end{bem} 

\begin{bem} In der Programmiersprache Python gibt es den Datentyp Menge sowie syntaktische Strukturen zur Konstruktion von Mengen und Listen, die analog zu mathematischen Bezeichnungen sind: 
\begin{center}
	\small 
	\begin{tabular}{l|l}
		Mathematik & Python 
\\ \hline	$\{1,2,5,7\}$ 
& \begin{lstlisting}[language=Python,backgroundcolor=\color{white}]
set([1,2,5,7])
\end{lstlisting} 
\\ $\setcond{i \in \Z}{0 \le i < 6}$ & 
\begin{lstlisting}[language=Python,backgroundcolor=\color{white}]
set(range(6))
\end{lstlisting} 
\\ $\setcond{i ^2}{i \in \Z, \ 0 \le i < 6}$
&  
\begin{lstlisting}[language=Python,backgroundcolor=\color{white}]
set([i**2 for i in range(6)])
\end{lstlisting} 
\\ $\setcond{i^2}{i \in \Z, \ 0 \le i < 6, \ i \ \text{ungerade}}$
& 
\begin{lstlisting}[language=Python,backgroundcolor=\color{white}]
set([i**2 for i in range(6) if i%2==1])
\end{lstlisting} 
\end{tabular}
\end{center}  
\end{bem} 

\begin{aufg}\ 
\begin{itemize}
\item Bestimmen Sie, welche der Zahlen $1,\ldots, 100$ Elemente der Menge 
\[
	 \{ k^2 : k \in \N, k \: \text{ungerade} \} 
\]
sind. 
\item Wie viele Elemente hat die Menge $\setcond{x \in \R}{x^2- 5 x + 6 = 0} $? Welche Elemente sind es genau? 
\end{itemize}
\end{aufg}

\begin{defn}
Die \textbf{leere Menge} ist die Menge, die keine Elemente enthält; sie wird als $ \emptyset $ bezeichnet. 
\end{defn} 

\begin{defn}[Potenzmenge]
Sei $ X $ eine Menge. Dann ist die \textbf{Potenzmenge} von $ X $ die Menge aller Teilmengen von $ X $; für diese Menge benutzen wir die Bezeichnung $ 2^X $. Nach unserer Definition ist $2^X$ als  
\[
	 2^X := \{ A : A \subseteq X \} .
\]
gegeben. 
\end{defn} 

\begin{bem} Hier und im Folgenden verwenden wir das Gleichheitszeichen $:=$ mit Doppelpunkt, wenn es um neue Bezeichnungen geht, die festgelegt werden. Format: 
	\[
		\text{NEUE BEZEICHNUNG} := \text{BEDEUTUNG DER BEZEICHNUNG}
	\]
	Die Nutzung vom Doppelpunkt in diesem Fall ist kein Muss; man kann auch durch den Begleittext verdeutlichen, dass man eine neue Bezeichnung einführt. 
\end{bem} 

\begin{aufg}
	Wenn $X$ genau $n \in \N$ Elemente hat, wie viele Elemente hat $2^X$? Was wäre Ihre Begründung dazu? 
\end{aufg} 

\begin{bem}
	Eine weitere Bezeichnung für die Potenzmenge, die man in der Literatur benutzt, ist $\mathcal{P}(X)$. Ich persönlich finde $2^X$ einleuchtender (zumindest im Kontext der Kombinatorik, die im folgenden Kapitel diskutiert wird). 
\end{bem} 


\begin{bem}
	Zahlenbereiche, die Sie evtl. aus der Schule schon kennen:
\begin{itemize}
\item[$ \N $] $ := \{ 1,2,3,\ldots \} $ die Menge der natürlichen Zahlen. Uns fehlt dort die Null, daher... 
\item[$ \N_0 $] $ := \{ 0,1,2,\ldots \} $. Hier können wir nicht beliebig subtrahieren, daher... 
\item[$ \Z $] $ := \{ 0,1,-1,2,-2,\ldots \} $ die Menge der ganzen Zahlen. Hier können wir nicht beliebig dividieren, daher...
\item[$ \Q $] $ := \{ \frac{p}{q} : p \in \Z, q \in \N \} $ die Menge der rationalen Zahlen. In dieser Menge gibt es \glqq Löcher\grqq , die man bemerkt, wenn man Geometrie oder Analysis betreibt. Gemeint ist das Folgende: betrachtet man unendlich viele rationale Zahlen $a_1 \le a_2 \le a_3 \le \cdots $ und $ \cdots \le b_3 \le b_2 \le b_1$ mit $a_n \le b_n$ für jedes $n \in \N$, so gibt es nicht immer eine \underline{rationale} Zahl $x$ die $a_n \le x \le b_n$ für alle $n \in \N$ erfüllt ist. Informell: zwischen zwei Schranken $a_n$ und $b_n$, die sich mit jeder \glqq Iteration\grqq\ $n \in \N$ immer mehr annähern, wird nicht immer eine rationale Zahl eingefangen. In diesem Fall kann man von einem \glqq Loch\grqq\ sprechen. Durch reelle Zahlen werden solche Löcher gestopft. 
\item[$ \R $] die Menge der reellen Zahlen. Format einer reellen Zahl: Vorzeichen $\pm$ endlich viele Stellen vor dem Komma, unendlich viele Nachkommastellen. Wir benutzen gerne die Computer-Formatierung mit Punkt an der Stelle von Komma (weil man in Mathe die Kommas gerne für viele andere Zwecke benutzt). Beispiele: 
\begin{itemize}
	\item[] $0{.}00000\cdots$ ist die $0$, 
	\item[] $1{.}00000 \cdots$ ist die $1$ 
	\item[] $-0{.}99999 \cdots$ ist das selbe wie $-1{.}000 \cdots$ und ist die $-1$. 
	\item[] $0{.}1\underbrace{0}_11\underbrace{00}_21 \underbrace{ 000}_3 1 \underbrace{0000}_41 \cdots $ ist eine reelle aber keine rationale Zahl. (Warum?)
\end{itemize} 
	\item[$ \C $] die Menge der komplexen Zahlen (werden wir noch einführen) 
\end{itemize}
\end{bem}

\begin{bem}
	Es gelten die strikten Inklusionen
$ \N \subsetneq \N_0 \subsetneq \Z \subsetneq \Q \subsetneq \R \subsetneq \C $. 
 \end{bem} 

\begin{bem}
	Manche Quellen definieren die Menge der natürlichen Zahlen als $ \set{0,1,2,\ldots} $, es ist mittlerweile sogar die ISO-Norm 80000-2. Aktuell spielen aber die ISO-Normen in den mathematischen Texten keine so große Rolle. Vgl. auch den ISO-Standard 31-11, in dem man z.B. $\subset$ für die strikte Inklusion reserviert. 
\end{bem}


\begin{bem}
	Für Zahlenbereiche $B  \in \{\Z,  \Q, \R\}$ und ein $a \in \R$ benutzen wir die folgenden Bezeichnungen:
	\begin{align*}
			B_{\ge a}&  := \setcond{x \in B}{x \ge a},  &  B_{>a} & := \setcond{x \in B}{x > a}, 
			\\ B_{\le a} & := \setcond{x \in B}{x \le a}, &  B_{<a} & := \setcond{x \in B}{x < a}. 
	\end{align*} 
	Beispiel: $\Z_{\ge 2}$. 
\end{bem} 


\begin{bem}[Intervalle]
	Seien $ a,b \in \R $ mit $ a \leq b $. Dann können Intervalle wie folgt definiert werden:
	\begin{align*}
		[a,b] &:= \{ x \in \R : a \leq x \leq b \}\\
		(a,b) &:= \{ x \in \R : a < x < b \}\\
		(a,b] &:= \{ x \in \R : a < x \leq b \}\\
		[a,b) &:= \{ x \in \R : a \leq x < b \} \\
		[a,\infty) & := \setcond{x \in \R}{x \ge a} \\
		(a,\infty) &:= \setcond{x \in \R}{x > a} \\ 
		(-\infty,a] &:= \setcond{x \in \R}{x \le a} \\
		(-\infty,a) &:= \setcond{x \in \R}{x < a}
	\end{align*}
\end{bem}


\begin{bem}
		Die Erweiterung der  Zahlenbereiche zu immer größeren Bereichen ist durch den Wunsch nach einer (gewissen) Vollständigkeit motiviert.
\end{bem} 


\begin{defn}
Seien $ A,B $ Mengen. Dann heißt:
\begin{itemize}
\item $ A \cap B := \{ x : (x \in A) \wedge (x \in B) \} $ \textbf{Durchschnitt} von $ A $ und $ B $,
\item $ A \cup B := \{ x : (x \in A) \vee (x \in B) \} $ \textbf{Vereinigung} von $ A $ und $ B $,
\item $ A \setminus B := \{ x : (x \in A) \wedge (x \notin B) \} $ \textbf{Mengendifferenz} von $ A $ und $ B $,
\item $A \operatorname{\triangle} B := (A \setminus B) \cup (B \setminus A)$ \textbf{Symmetrische Differenz} von $A$ und $B$. 
\end{itemize}
\end{defn}

\begin{bem}
	Für eine Grundmenge $X$ wird die Menge $2^X$ aller Teilmengen von $X$ zu einer sogenannten \textbf{booleschen Algebra} der Teilmengen von $X$, indem man $2^X$ mit den Verknüpfungen $A \cap B$, $A \cup B$ und der unären Verknüpfung $\overline{A} := X \setminus A$ ausstattet. 
\end{bem} 

\begin{defn}
Seien $ A,B $ Mengen. $ A $ und $ B $ heißen genau dann \textbf{disjunkt}, wenn $ A \cap B = \emptyset $ gilt. In diesem Fall wird die Vereinigung von $A$ und $B$ eine  \textbf{disjunkte Vereinigung} genannt
und als $A \cupdot B$ bezeichnet. 
\end{defn}

\section{Tupel und Kreuzprodukte}

\begin{defn}[Paare und das Kreuzprodukt] 
	Für beliebige Objekte $ a,b $ kann man das \textbf{(geordnete) Paar} $ (a,b) $ definieren. Das Paar $(a,b)$ besteht aus der ersten Komponente $a$ und der zweiten Komponente $b$. 
	Die Gleichheit $ (a,b) = (c,d) $ von Paaren wird durch die Gleichheit $ a=c $ und $ b=d $ der jeweiligen Komponenten definiert.  Für Mengen $ A,B $ definiert man das \textbf{Kreuzprodukt} (wird auch das \textbf{kartesische Produkt} genannt) $ A \times B $ als die Menge
	\[
	A \times B := \{ (a,b) \,:\, a \in A, \ b \in B \}
	\]
	aller Paare bei denen die erste Komponente in $A$ und die zweite Komponente in $B$ liegt. 
\end{defn} 

\begin{defn}[Tupel und Kreuzprodukt] 
	Komplett analog zu (geordneten) Paaren definiert man auch (geordnete) Tripel $(a,b,c)$, (geordnete) Quadrupel $(a,b,c,d)$ und allgemeiner (geordnete) $n$-Tupel $(x_1,\ldots,x_n)$ mit $n \in \N_0$. Dementsprechend betrachtet man auch das Kreuzprodukt $A \times B \times C$ von drei Mengen, das Kreuzprodukt $A \times B \times C \times D$ von vier Mengen und allgemein das Kreuzprodukt 
	\[
	X_1 \times \ldots \times X_n := \{ (x_1,\ldots,x_n) \,:\, x_1 \in X_1, \ldots , x_n \in X_n \} 
	\]
	von Mengen $ X_1,\ldots,X_n $.
	
	Das Element $ x_i $ mit $ i \in \is{1}{n} $ im $ n $-Tupel $ (x_1,\ldots,x_n) $ heißt die $ i $-te \textbf{Komponente} des Tupels.
	
	Für eine Menge $ X $ führt man die Bezeichnung
	\begin{equation*}
		X^n := \underbrace{X \times \ldots \times X}_{n \:\text{mal}} = \{ (x_1,\ldots,x_n) : x_1,\ldots,x_n \in X \}
	\end{equation*}
	ein. 
\end{defn} 


\begin{aufg} Was stellen die folgenden Kreuzprodukte geometrisch dar? Zeichnen Sie diese Mengen:
	\begin{itemize}
		\item $ [0,1] \times [0,2] $, $ \{ 0 \} \times [0,2] $  und $ \{ 0,1 \} \times \{ 0,2 \} $ 
		\item $ [0,1]^3 $, $ [0,1]^2 \times \{ 0 \} $, $ [0,1]^2 \times \{ 1 \} $  und $ \{ 0 \}^2 \times [0,1] $. 
	\end{itemize}
	\textbf{Hinweis:} Die geometrische Darstellung von Teilmengen des $\R^n$ für $n \in \{2,3\}$ im kartesischen Koordinatensystem wird in den Kursen IT-2 und IT-3 nützlich sein. 
\end{aufg}

\begin{aufg}
	Beim bekannten Spiel \glqq Schiffe versenken\grqq\ lässt sich das Spielfeld als die Menge $\{1,\ldots,10\}^2$ beschreiben, weil jedes Kästchen des Spielfeldes durch die Angabe von einem Paar $(x,y)$ mit $x,y \in \{1,\ldots,10\}$ beschrieben werden kann. Eine Platzierung eines 4er-Schiffs lässt sich dann als eine vierelementige Teilmenge von $\{1,\ldots,10\}^2$ beschreiben. Z.B.: 
	\[ \{(1,2), (1,3),(1,4),(1,5)\}
	\] und 
	\[ \{(5,6),(6,6),(7,6),(8,6)\}
	\] sind zwei mögliche Platzierungen. Beschreiben Sie die Menge aller möglichen Platzierungen eines 4er-Schiffs mit Hilfe der mathematischen Bezeichnungen, die Sie oben gelernt haben. 
\end{aufg} 


\section{Abbildungen}


\begin{defn}
Seien $ X $ und $ Y $ Mengen. Eine \textbf{Abbildung} $ f $ von $ X $ nach $ Y $ ist eine Vorschrift, die jedem $ x \in X $ genau ein Element aus $ Y $ zuordnet. Das Element aus~$ Y $, das einem Element $x \in X $ zugeordnet wird, wird durch $ f(x) $ bezeichnet.

Wenn $ f $ eine Abbildung von $ X $ nach $ Y $ ist, dann notieren wir dies durch $ f : X \to Y $. Die Menge $ X $ heißt dann der \textbf{Definitionsbereich} von $ f $, und die Menge~$ Y $ der \textbf{Wertebereich} von $ f $.

Wenn der Wertebereich $Y$ von $f$ eine Teilmenge von $\R$ ist, so nennen wir $f : X \to Y$ eine \textbf{Funktion}. Bei der Beschreibung von Abbildungen benutzt man auch die Bezeichnung $x \mapsto f(x)$, um zu verdeutlichen, dass $x$ auf $f(x)$ abgebildet wird. 
\end{defn} 

\begin{bem} 
In manchen Quellen benutzt man den Begriff Funktion als ein Synonym zum Begriff Abbildung. 
\end{bem} 
% 16.10.2014

%Beispiel:
%\begin{enumeration}
%\item Kekse
%\item Nüsse
%\item Riegel
%\end{enumeration}
%
%\begin{table}[h]
%	\begin{tabular}{c|c|c}
%		$ X $ & $ Y $ & $ Y $ \\
%		\hline
%		1 & Kekse & Kekse \\
%		2 & Kekse & Nüsse \\
%		3 & Kekse & Riegel
%	\end{tabular}
%\end{table}

\begin{bsp}\ 
\begin{itemize}
	\item $ f : \R \to \R$ mit $ f(x) := x^2 -2x + 7 $ für alle $x \in \R$
	\item $ f : \R \setminus \{ 1 \} \to \R$ mit $ f(x) := \frac{1}{x-1}$ für alle $x \in \R \setminus \{ 1 \}$
	\item Die Vorzeichen-Funktion $ \sign: \R \to \R $ wird durch Fallunterscheidung als 
	\[
		\sign(x) := \begin{cases} 
				-1 &  \text{für} \ x < 0,
				\\ 0 & \text{für} \ x =0,
				\\ 1 & \text{für}  \ x > 0
			\end{cases} 
	\]
	definiert. 
	\item $g : \N \to \N$ mit $g(n) := 1$ falls $n \geq 10$ und $g(n) := 2$ falls $n \leq 50$ ist keine Abbildung
	\item Definitions- bzw. Wertebereiche von Abbildungen müssen keine Teilmengen der Zahlenbereiche sein:
	\begin{itemize}
	 \item Sei $ f : 2^{\N} \to \N$ mit $ f(A) := \min(A) $. Die Eingabe dieser Funktion ist keine Zahl sondern eine Menge, man hat z.B. $ f( \{ 2,7,43 \} ) = 2 $.
	\item Sei $ f : \N \to 2^\N$ mit $ f(k) := \is{1}{k}.$ Die Rückgabe dieser Abbildung ist keine Zahl sondern eine Menge, z.B. $f(3) = \{1,2,3\}$. 
	\end{itemize}
	\item $(x,y) \mapsto (x,-y)$ ist eine vertikale Spiegelung $\R^2 \to \R^2$. 
	\item $(x,y) \mapsto (-x,y)$ ist eine horizontale Spiegelung $\R^2 \to \R^2$. 
\end{itemize}
\end{bsp} 

\begin{bem}
Zwei Abbildungen $ f,g : X \to Y $ heißen gleich, falls $ f(x) = g(x) $ für alle $ x \in X $ gilt.

\underline{Wichtig:} Bei Gleichheit von zwei Abbildungen müssen insbesondere Definitions- und Wertebereich gleich sein, nicht nur die Zuweisungsvorschrift.
\end{bem} 

\begin{bem} 
	Für zwei Mengen $X$ und $Y$, 
	bezeichnet man als $ Y^X $  die Menge aller Abbildungen von $ X $ nach $ Y $.
\end{bem} 

\begin{aufg}
	Aus wievielen Abbildungen besteht die Menge $\{1,2,3\}^{\{1,2\}}$? Zählen Sie diese Abbildungen auf. Was ist  mit $\{1,2\}^{\{-1,0,1\}}$?
\end{aufg} 

\begin{defn}
Seien $ X,Y,A,B $ Mengen mit $ A \subseteq X $ und $ B \subseteq Y $ und sei $ f : X \to Y $ eine Abbildung. Dann heißt $ f(A) := \{ f(x) : x \in A \} $ das \textbf{Bild} von $ A $ bzgl. $ f $ und $ f^{-1}(B) := \{ x \in X : f(x) \in B \} $ das \textbf{Urbild} von $ B $ bzgl. $ f $.
\end{defn} 

\begin{bsp}
	Sei $ f : \R \to \R $ mit $ f(x) := x^2 $ für alle $ x \in \R $.
	\begin{itemize}
		\item $ f( [1,2] ) = [1,4] $. 
		\item $ f^{-1}( [1,4] ) = [1,2] \cup [-2,-1] $. 
		\item
		$ \begin{aligned}[t]
			f^{-1}( [-7,8] ) &= \{ x \in \R : f(x) \in [-7,8] \} \\
			&= \{ x \in \R : -7 \leq f(x) \leq 8 \} \\
			&= \{ x \in \R : -7 \leq x^2 \leq 8 \} \\
			&= \{ x \in \R : x^2 \leq 8 \} \\
			&= \{ x \in \R : |x| \leq \sqrt{8} \} \\
			&= [-\sqrt{8},\sqrt{8}]
		\end{aligned} $
\end{itemize}
\end{bsp}


\begin{defn}
Seien $ X,Y $ Mengen und sei $ f : X \to Y $. Dann heißt $ f $
\begin{itemize}
	\item \textbf{injektiv}, falls für alle $ x_1, x_2 \in X$ mit $ x_1 \neq x_2 $ die Bedingung $ f(x_1) \neq f(x_2) $ gilt.
	\item \textbf{surjektiv}, falls für jedes $ y \in Y $ ein $ x \in X $ mit der Eigenschaft $ f(x) = y $ existiert.
	\item \textbf{bijektiv}, falls $ f $ injektiv und surjektiv ist.
\end{itemize}
\end{defn}

\begin{bsp}
	Wir untersuchen die folgende Funktionen auf Bijektivität:\\[-2em]
	\begin{itemize}
		\item $ f : \R \to \R, f(x) := x^2 $
		für alle $ x \in \R $\\
		surjektiv ? nein, $ -1 \neq f(x) $ für alle $ x \in \R $\\
		injektiv ? nein, $ f(x) = f(-x) $ für alle $ x \in \R $
		\item $ f : \R \to \R_{\geq 0}, f(x) := x^2 $ für alle $ x \in \R $\\
		surjektiv ? ja\\
		injektiv ? nein (wie zuvor)
		\item $ f : \R \setminus \{ 0 \} \to \R, f(x) = \frac{1}{x} $ für alle $ x \in \R \setminus \{ 0 \}$\\
		surjektiv ? nein, 0 wird nicht angenommen\\
		injektiv ? ja
		\item $ f : \R \to \R, f(x) = 2x + 3 $ für alle $ x \in \R $\\
		bijektiv ? ja
	\end{itemize}
\end{bsp}


\begin{defn} 
Seien $ X,Y $ Mengen und sei $ f : X \to Y $ bijektiv. Die Abbildung, die jedem $ y \in Y $ das eindeutige $ x \in X $ mit $ f(x) = y $ zuordnet, heißt die \textbf{Umkehrabbildung} von $ f $ und wird durch $ f^{-1} : Y \to X $ bezeichnet.
\end{defn} 

\begin{aufg}
	Was ist die Umkehrfunktion von $ f : \R \to \R$ mit $ f(x) := 2x + 3 $ ?
\end{aufg}


\begin{defn}
Seien $ X,Y,Z $ Mengen, $ f : X \to Y $ und $ g : Y \to Z $ Abbildungen. Dann heißt $ g \circ f : X \to Z $ mit $ ( g \circ f )(x) := g( f(x) ) $ für alle $ x \in X $ die \textbf{Komposition} (oder Verkettung) von $ g $ und $ f $.
\end{defn}

\begin{bsp}
	Seien $ f : \R \to \R$ mit $ f(x) = 2x + 3 $ für alle $ x \in \R $ und $ g : \R \to \R $ mit $g(x) = x^2 $ für alle $ x \in \R $. Dann ist $ ( f \circ g )(x) = 2x^2 + 3 $ und $ ( g \circ f )(x) = (2x + 3)^2 $.
\end{bsp}


\begin{defn}
Sei $ X $ eine Menge. Dann heißt die Abbildung $ \id_X : X \to X $ mit $ \id_X(x) := x $ für alle $ x \in X $ die \textbf{identische Abbildung} auf $ X $. Man schreibt auch häufig $ \id $, wenn die Wahl von  $ X $ aus dem Kontext klar ist. 
\end{defn}

\begin{bem}
	Seien $ X,Y $ Mengen und sei $ f : X \to Y $ bijektiv. Dann gilt
	\begin{itemize}
		\item $ f \circ f^{-1} = \id_Y $,
		\item $ f^{-1} \circ f = \id_X $.
	\end{itemize}
\end{bem} 

\section{Vereinigung und Durchschnitt von Mengenfamilien}

\begin{defn}
Seien $ I,X $ Mengen und sei $ A : I \to 2^X $. Man schreibt auch in diesem Fall $ A_i $ statt $ A(i) $ für $ i \in I $, $ (A_i)_{i \in I} $ ist eine \textbf{Familie} (bzw. \textbf{Schar}) von Teilmengen von $ X $.

Für die Familie $ (A_i)_{i \in I} $ definiert man ihren \textbf{Durchschnitt} als 
\begin{align*}
	\bigcap_{i \in I} A_i &:= \setcond{x \in X}{x \in A_i \:\text{für alle}\: i \in I},
\end{align*}
und ihre \textbf{Vereinigung} als 
\begin{align*}
	\bigcup_{i \in I} A_i &:= \setcond{x \in X}{x \in A_i \:\text{für ein}\: i \in I}.
\end{align*}
\end{defn} 

\begin{bsp} 
	Die Familie $(T_a)_{a \in \R}$ mit 
	\[
		T_a := \setcond{ (x,y) }{x \in \R, \ y =  a^2 + 2  a (x-a)}
	\]
	ist die Familie aller Tangenten zur Parabel 
	\[
	 	P = \setcond{ (x,y) }{x \in \R, \ y = x^2}. 
	 \]
	 Für die Vereinigung dieser Familie gilt
	 \[
	 	\bigcup_{a \in \R} T_a  = \setcond{ (x,y)}{x \in \R, \ y \le x^2},
	 \]
	 die Vereinigung ist somit die Menge aller Punkte, die unterhalb oder auf der Parabel $P$ liegen. 
	 
	 Man betrachte noch die Familie $(H_a)_{a \in \R}$ der Halbräume 
	 \[
	 		H_a := \setcond{(x,y)}{x \in \R, \ y \ge a^2 + 2 a (x-a)}. 
	 \]
	 Hierbei ist $H_a$, die Menge aller Punkte die oberhalb oder auf der Tangente $T_a$ liegen. Für den Durchschnitt dieser Familie gilt 
	 \[
	 		\bigcap_{a \in \R} H_a := \setcond{ (x,y) }{y \ge x^2},
	 \]
	 dieser Durchschnitt ist somit die Menge aller Punkte, die oberhalb oder auf der Parabel $P$ liegen. 
\end{bsp} 

\section{Summen und Produkte}

\begin{defn} 
Eine Menge $ X $ heißt endlich, falls $ X = \emptyset $ oder falls eine bijektive Abbildung von $ \is{1}{n} $ nach $ X $ existiert mit $ n \in \N $. Der Wert $ n $ heißt die Anzahl der Elemente (\textbf{Kardinalität}) von $ X $ und wird durch $ |X| $ bezeichnet. Man setzt die Kardinalität von $ \emptyset $ gleich 0. Bei einer unendlichen Mengen $X$ setzt man $|X| := \infty$. 
\end{defn} 

\begin{bem}
$ |X| $ ist wohldefiniert, d.h. eine Menge kann nicht zwei unterschiedliche Kardinalitäten haben.
\end{bem} 

\begin{defn}
Sei $ X $ eine nichtleere endliche Menge. Dann kann $ X $ als $ X = \is{x_1}{x_n} $ dargestellt werden mit  $ x_i \neq x_j $ für alle $ i,j \in \is{1}{n} $ mit $i \ne j$. 

Für eine Abbildung $ f : X \to \R $ definiert man
\begin{align*}
	\sum\limits_{x \in X} f(x) &:= f(x_1) + \ldots + f(x_n),
\\
	\prod\limits_{x \in X} f(x) &:= f(x_1) \cdot \ldots \cdot f(x_n).
\end{align*}
Im Fall $ X = \emptyset $ definiert man für $ f : X \to \R $ die leere Summe $ \sum\limits_{x \in X} f(x) := 0 $ und das leere Produkt $ \prod\limits_{x \in X} f(x) := 1 $.

In der Summe $\sum\limits_{x \in X} f(x)$ heißt $f(x)$ \textbf{Summand} und im Produkt $\prod\limits_{x \in X} f(x)$ heißt $f(x)$ \textbf{Faktor}.  
\end{defn}

\begin{bem} 
	Die Summe und das Produkt über eine Menge $X$ sind wohldefiniert: die beiden Werte sind von der Nummerierung $x_1,\ldots,x_n$ der Elemente von $X$ unabhängig. Das liegt daran, dass $+$ und $\cdot$ beide kommutative Operationen sind. 
\end{bem} 


\begin{bem}
	$\sum\limits_{i=a}^b f(i)$ benutzt man als eine kurze Schreibweise für $\sum\limits_{i \in \{a,\ldots,b\}} f(i)$. Im entarteten Fall $a>b$ ist $\sum\limits_{i =a}^b f(i)$ eine Summe über die leere Menge, also die leere Summe.
\end{bem} 

\begin{bsp}
	Für jedes $n \in \N$ gilt $S:=\sum\limits_{i=1}^n i = \frac{1}{2} n (n+1).$ Die Abbildung $i \mapsto n+1 - i$ ist eine Bijektion von $\{1,\ldots,n\}$ nach $\{1,\ldots,n\}$. Daher gilt  $S = \sum\limits_{i=1}^n (n+1 - i)$. Daraus ergibt sich 
	\[
		2 S = S +  S = \sum_{i=1}^n i + \sum_{i=1}^n (n+1 - i) = \sum_{i=1}^n ( i + n + 1 -i) = \sum_{i=1}^n (n+1) = n (n+1). 
	\]
	Das ergibt die gewünschte Darstellung $S = \frac{1}{2} n (n+1)$. 
\end{bsp} 




\section{Prädikate und Quantoren}


\begin{defn}
Sei $ X $ Menge. Dann heißt $ P : X \to \{ \false,\true \} $ \textbf{Prädikat} auf $ X $. 
\end{defn} 

\begin{bem}
	Informell beschrieben ist ein Prädikat eine Aussage über ein variables Element $x$ aus $X$. Dabei hängt im Allgemeinen der Wahrheitswert der Aussage von der Wahl von $x$ ab.
	
	Alternativ lässt sich ein Prädikat auf $X$ als eine Eigenschaft eines variablen Elements $x \in X$ beschreiben, die entweder vorhanden oder nicht vorhanden ist.
	
	\underline{Zum Beispiel:} $P(n):=$ \glqq$n$ ist gerade\grqq\ als Eigenschaft einer natürlichen Zahl $n \in \N$. Diese Eigenschaft ist bei $n=123$ nicht vorhanden und bei $n=2020$ vorhanden, also $P(123) = \false$ und $P(2020) = \true$.
\end{bem}

\begin{bsp}
$ P : \N \to \{ \false,\true \}$ mit 
\[
	P(k) :=  \text{\glqq$k(k+1)$ ist durch  $ 3$  teilbar\grqq}. 
\]
\end{bsp} 

\begin{bsp}
$ P : \R^2 \to \{ \false , \true \}$ mit 
\[
	P(x,y) := \text{\glqq Es gilt $x^2 + y^2 \le 1$.\grqq}
\] 
ist die Eigenschaft \glqq der Punkt $(x,y)$ hat den Abstand höchstens $1$ zum Punkt $(0,0)$\grqq. 
\end{bsp} 

\begin{bem} 
	Für eine gegebene Menge $X$ hat man eine natürliche Bijektion 
	\[
		 \{ \false,\true \}^X \to 2^X.
	\] 
	Jedes Prädikat $P : X \to \{ \false,\true \}$ erzeugt die Menge $A:=\setcond{x \in X }{P(x)}$. Umgekehrt entspricht jede Menge $A \subseteq X$ dem Prädikat 
	\[
		P (x) := \text{\glqq Es gilt $x \in A$\grqq}. 
	\]
\end{bem} 


\begin{defn}
$ \forall\: x \in X : P(x) $ für ein Prädikat $ P $ auf eine Menge $ X $ steht für die Aussage \textquote{die Bedingung $ P(x) $ gilt für alle $ x \in X $}. Das Symbol $ \forall $ heißt \textbf{Allgemeinheitsquantor} oder \textbf{Allquantor} (Bedeutung: für $ \forall $lle). \\[10pt]
%
$ \exists\: x \in X : P(x) $ bezeichnet die Aussage \textquote{die Bedingung $ P(x) $ gilt für ein $ x \in X $}. Das Symbol $ \exists $ heißt \textbf{Existenzquantor} (Bedeutung: es $ \exists $xistiert).
\end{defn}

\begin{bem}
	Negieren von quantifizierten Aussagen erfolgt durch die folgende naheliegende Weise: 
	\begin{itemize}
		\item $ \overline{\forall\: x \in X : P(x)} \Leftrightarrow \exists\: x \in X : \overline{P(x)} $
		\item $ \overline{\exists\: x \in X : P(x)} \Leftrightarrow \forall\: x \in X : \overline{P(x)} $
	\end{itemize}
\end{bem}


\begin{bem}
	$ \forall $ und $ \exists $ lassen sich kombinieren. Seien $X, Y$ Mengen. Wenn man ein Prädikat $ P $ auf $ X \times Y $ hat, also eine Abbildung $P : X \times Y \to \{\false,\true\}$, so kann man dafür Aussagen wie
	\[
		 \forall\, x \in X \,\exists\: y \in Y : P(x,y)
	\]
	und
	\[\exists\, x \in X  \, \forall\, y \in Y : P(x,y) 
	\]
	einführen. 
\end{bem}


\begin{bem}
	In der vorigen Bemerkung ist die Reihenfolge des Quantifizierens relevant. Sei $X$ eine Menge von  Personen und $A$ eine Menge von Adressen im Stadtteil Sandow. Die Aussage 
	\[
		\forall\, x \in X \, \exists\, a \in A \, :\, x \ \text{wohnt unter der Adresse} \  a
	\]
	bedeutet, dass alle Personen aus $X$ irgendwo in Sandow wohnen. Die Aussage 
	\[
		\exists\, a \in A \, \forall\, x \in X \,:\, x \ \text{wohnt unter der Adresse} \ a
	\]
	bedeutet dagegen, dass alle Personen aus $X$ unter einer und derselben Adresse in Sandow wohnen (z.B. als Wohngemeinschaft).
	
	Man sieht, die zweite Aussage ist eine stärkere Bedingung.
	Das hei\ss t, dass die zweite Aussage die erste impliziert.
\end{bem} 

\begin{bsp}
	Hier ein Beispiel einer Definition aus der Analysis, die man kompakt mit Quantoren und Prädikaten einführen kann. 
	
	Sei $ (a_n)_{n \in \N} $ Folge reeller Zahlen (mit anderen Worten: $ a : \N \to \R $) und sei $ \alpha \in \R $. Dann heißt $ \alpha $ Grenzwert von $ (a_n)_{n \in \N} $, falls das Folgende gilt:
	\begin{equation*}
		\forall\: \epsilon \in \R_{>0} \:\exists\: N \in \R \:\forall\: n \in \N: \left( (n \geq N) \Rightarrow (|a_n - \alpha| < \epsilon) \right)
	\end{equation*}
\end{bsp} 

\section{Relationen}


\begin{defn}
Seien $ X,Y $ Mengen. Dann heißt eine Teilmenge $ R $ von $ X \times Y $ eine (binäre) \textbf{Relation} zwischen $ X $ und $ Y $. Bei $X = Y$, heißt $R$ eine (binäre) Relation auf $X$. 
\end{defn}

\begin{bem}
	Da man Prädikate auf $X \times Y$ mit Teilmengen von $X \times Y$ identifizieren kann, lassen sich Relationen auch als Prädikate auf $X \times Y$ auffassen. 
\end{bem} 

\begin{bem}
Wenn für $ x \in X $ und $ y \in Y $ die Bedingung $ (x,y) \in R $ gilt, so schreibt man $ x \, R \, y $. Das bedeutet: $x$ steht in der Relation $R$ zu $y$. 
\end{bem}

\begin{bsp}\ % hack to force itemize to new line
\begin{itemize}
	\item $ X = \{f_1,f_2,f_3,f_4\} $ - Menge von Fahrzeugen\\
	$ Y =  \{ \text{Ersatzrad}, \text{Radio}, \text{Navi}, \text{Automatik} \}$ - Menge von Features von Fahrzeugen
	\begin{center} 
	\begin{tabular}{l|c|c|c|c}
		& Ersatzrad & Radio & Navi & Automatik \\
		\hline
		$ f_1 $ & ja & ja & ja & ja \\
		$ f_2 $ & ja & ja & ja & nein \\
		$ f_3 $ & nein & nein & ja & ja \\
		$ f_4 $ & nein & ja & ja & nein
	\end{tabular}
	\end{center}  
	Diese Tabelle legt eine Relation zwischen $X$ und $Y$ fest. 
	\item $ \leq, <, \geq, > $ sind binäre Relationen auf $ \R$
	\item Sei $X$ Menge. Dann ist $ \subseteq $ eine binäre Relation auf $ 2^X $.
	\item Für $ a,b \in \N $ schreibt man $ a \mid b $, wenn $ b $ durch $ a $ ohne Rest teilbar ist. Dies ist eine binäre Relation auf $\N$. 
\end{itemize}
\end{bsp}


\begin{defn}
Sei $ \sim $ eine Relation auf der Menge $ X $. Dann heißt $ \sim $ \textbf{Äquivalenz\-relation}, falls: 
\begin{enumerate}
	\item $ \sim $ ist \textbf{reflexiv}, d.h. $ x \sim x $ für alle $ x \in X $.
	\item $ \sim $ ist \textbf{symmetrisch}, d.h. $ x \sim y $ ist äquivalent zu $ y \sim x $ für alle $ x,y \in X $.
	\item $ \sim $ ist \textbf{transitiv}, d.h. aus $ x \sim y $ und $ y \sim z $ folgt $ x \sim z $ für alle $ x,y,z \in X $.
\end{enumerate}
Für eine Äquivalenzrelation $ \sim $ auf der Menge $ X $ und ein $ x \in X $ heißt
\begin{equation*}
	[x]_\sim := \{ y \in X : x \sim y \}
\end{equation*}
die \textbf{Äquivalenzklasse} von $ x $ bzgl. $ \sim $. Die Menge aller Äquivalenzklassen von $ \sim $ ist
\begin{equation*}
	X/{\sim} := \{ [x]_\sim : x \in X \}. % {\sim} supresses space between to "/"
\end{equation*}
\end{defn} 

\begin{bsp}\
\begin{itemize}	
	\item Sei $ m \in \N $. Für $ a,b \in \Z $ sagt man, dass $ a $ kongruent zu $ b $ modulo $ m $ ist, falls $ a-b \in m\Z $, wobei $ m\Z := \{ mz : z \in \Z \} $. Äquivalent gilt also $m \mid (a-b)$.
	
	Schreibweise: $ a \equiv b \mod{m} $.
	
	Die Kongruenz modulo $ m $ ist eine Äquivalenzrelation auf $ \Z $.
	
	Die Äquivalenzklassen werden oft mit $[a]_m$ bezeichnet und heißen Restklassen modulo $m$.
	
	\item Sei $ \sim $ Relation auf $ \Z \times \N $, definiert durch $ (a,b) \sim (c,d) $ für $ a,c \in \Z, b,d \in \N $, wenn $ ad = bc $ gilt.
	
	Diese Relation ist eine Äquivalenzrelation (Hausaufgabe!).
	
	Formal kann man die Menge der rationalen Zahlen als die Menge der Äquiva\-lenz\-klassen bzgl. dieser Äquivalenzrelation umsetzen. 
\end{itemize}
\end{bsp}


\begin{defn}
	Eine Menge $X$ mit einer binären Relation $\preceq$ darauf heißt \textbf{Poset} (partiell geordnete Menge; engl.~\underline{p}artially \underline{o}rdered \underline{set}), wenn für alle $x,y,z \in X$ folgendes gilt: 
	\begin{itemize}
		\item $x \preceq x$ (Reflexivität)
		\item $x \preceq y, \ y \preceq z$ $\Rightarrow$ $x \preceq z$ (Transitivität)
		\item $x \preceq y, \ y \preceq x$ $\Rightarrow$ $x = y$ (Antisymmetrie)
	\end{itemize} 
	Ein Poset notieren wir mittels $(X,\preceq)$.\\
	Die binäre Relation $\preceq$ heißt in diesem Fall die \textbf{partielle Ordnung} auf $X$. 
\end{defn} 

\begin{defn}
	Wenn für ein Poset $(X,\preceq)$ für alle $x, y \in X$, die Bedingung $x \preceq y$ oder die Bedingung $y \preceq x$ erfüllt ist, so nennt man $(X,\preceq)$ eine \textbf{total geordnete Menge} und $\preceq$ eine \textbf{totale Ordnung} auf $X$. 
\end{defn} 

\begin{bsp} Beispiele von Posets:
	\begin{itemize}
		\item $\R$ mit $\leq$
		\item $2^X$ mit Inklusion 
		\item $\N$ mit Teilbarkeit
	\end{itemize} 
\end{bsp} 

\begin{defn}
	Für $n \in \N$ ist eine \textbf{$n$-stellige Relation} auf den Mengen $X_1,\ldots,X_n$ eine Teilmenge $R \subseteq X_1 \times \cdots \times X_n$. 
\end{defn} 

\begin{bsp}
	Betrachten wir eine Tabelle, in welcher die Besucher:innen eines Hotels durch die Angaben \textbf{Name, Zimmer, Checkin-Datum, Checkout-Datum} geführt werden. Ist $S$ die Menge aller Strings und $D$ die Menge aller Daten, so kann man die Tabelle als eine $4$-stellige Relation  $R \subseteq S \times S \times D \times D$ auffassen. Die Bedingung $(p,z,d_{in},d_{out}) \in R$, dass $(p,z,d_{in},d_{out})$ sich in der Relation $R$ befindet,  bedeutet, dass die Person $p$ am Tag $d_{in}$ im Zimmer $z$ untergebracht wurde und am Tag $d_{out}$ das Hotel verlassen hat. 
	
	Wie man an diesem Beispiel sieht, sind Tabellen eine Möglichkeit, Relationen $R$ durch eine Aufzählung (durch die Zeilen einer Tabelle) zu beschreiben. 
\end{bsp} 


\section{Beweisansätze} 

\subsection{Direkter Beweis} 

\begin{bem}
	Eine grobe Beschreibung eines direkten Beweises ist wie folgt. Ein direkter Beweis einer Implikation $a \Rightarrow b$ ist ein Beweis der auf Implikationen basiert, die von $a$ ausgehen und zu $b$ führen.
\end{bem} 

\begin{bem}
	Man kann für $x \in \R$ die Äquivalenz: 
	\[
			x^2 - 5 x + 6 = 0  \qquad \Longleftrightarrow \qquad x \in \{2,3\}
	\]
	als zwei Implikationen ausschreiben und anschließend 
	folgendermaßen direkt verifizieren. Es ist $x^2 - 5 x + 6 = (x-2) (x-3)$. Aus $(x-2) (x-3)=0$ folgt, dass $x-2=0$ oder $x-3=0$ gilt. Im ersten Fall erhält man aus $x-2=0$, dass $x=2$ ist. Im zweiten Fall erhält man aus $x-3 = 0$, dass $x=3$ ist. Folglich hat man $x \in \{2,3\}$. Umgekehrt: ist $x \in \{2,3\}$, so hat man im Fall $x=2$ die Gleichheiten $x^2 - 5 x + 6= 2^2 - 5 \cdot 2 + 6 = 4- 10 + 6 =0$ und im Fall $x=3$ die Gleichheiten $x^2 - 5 x + 6 = 3^2 - 5 \cdot 3 + 6 = 9 - 15 + 6 = 0$. Aus $x \in \{2,3\}$ folgt also $x^2 - 5 x + 6$. 
\end{bem} 


\subsection{Indirekte Beweise} 


\begin{bem}
	Ein Widerspruchsbeweis ist ein Beweis, bei dem man eine Aussage $a$, die man zeigen möchte, durch die Herleitung der Implikation $\overline{a} \Rightarrow \text{falsch}$ bestätigt bzw. 
	eine Implikation $a \Rightarrow b$, die man zeigen möchte, durch die Herleitung der Implikation $a \wedge \overline{b} \Rightarrow \text{falsch}$ bestätigt. 
	
	In beiden Fällen erhält man aus einer Annahme einen sogenannten Widerspruch, d.h., eine falsche Aussage. Den Widerspruch erhält man oft in der Form $c \wedge \overline{c}$ für eine Aussage $c$. 
\end{bem} 

\begin{lem}
	Für $t \in \N$ seien $p_1,\ldots,p_t \in \N$ Zahlen mit $p_i \ge 2$ für alle $i \in \{1,\ldots,t\}$ und sei $n := p_1 \cdots p_t + 1$. Dann ist $n$ durch keine der Zahlen $p_1,\ldots,p_t$ teilbar. 
\end{lem} 
\begin{proof} 
	Angenommen, $n$ wäre durch ein $p_i$ mit $i=1,\ldots,t$ teilbar. Da aber das Produkt $p_1 \cdots p_t$ durch $p_i$ teilbar ist, ist $1 = n - p_1 \cdots p_t$ ebenfalls durch $p_i$ teilbar. Wir haben also gezeigt, dass $1$ durch die ganze Zahl $p_i$, mit $p_i \ge 2$, teilbar ist. Das ist ein Widerspruch, der uns die Behauptung unseres Lemmas bestätigt. 
\end{proof} 

\begin{bem}
	Ein Beweis durch Kontraposition ist der Beweis der Implikation $a \Rightarrow b$ dadurch, dass man die Implikation $\overline{b} \Rightarrow \overline{a}$ bestätigt. Ein  Beweis durch Kontraposition und der Widerspruchsbeweis sind miteinander verwandt, denn einen Beweis durch Kontraposition kann man in einen Widerspruchsbeweis konvertieren. 
\end{bem} 

\begin{bem}
	Beweise durch Kontraposition und Widerspruch nennt man \emph{indirekt.} 
\end{bem} 

\begin{lem} \label{lem:a:a^3}
	Sei $a \in \N$. Dann sind die folgenden Aussagen äquivalent: 
	\begin{enuma}
		\item $a$ ist gerade. 
		\item $a^3$ ist gerade. 
	\end{enuma} 
\end{lem} 
\begin{proof} 
	Wir zeigen (a) $\Rightarrow$ (b) direkt. Ist $a$ gerade, so hat $a$ die Form $a = 2 k$ mit $k \in \N$. Somit ist $a^3 = (2 k)^3 = 8 k^3 = 2 \cdot 4 k^3$ ebenfalls gerade. 
	
	Die Implikation (b) $\Rightarrow$ (a) können wir durch die Kontraposition herleiten: wir zeigen also $\neg$ (a) $\Rightarrow$ $\neg$ (b). Wenn $a$ ungerade ist, so hat $a$ die Form $a = 2 k+1$ mit $k \in \N_0$. Somit ist $a^3  = (2k + 1)^3 = (2k)^3 + 3 (2k)^2 + 3 (2k) + 1 = 2 ( 4 k^3 + 6 k^2 + 3 k) + 1$ eine ungerade Zahl.  
\end{proof} 

\begin{thm}
	Die Zahl $\sqrt[3]{2}$ ist nicht rational. 
\end{thm} 
\begin{proof} 
	Angenommen, $\sqrt[3]{2}$ wäre rational. Dann hätte die Zahl die Form $\sqrt[3]{2} = \frac{a}{b}$ mit $a, b \in \N$. Darüberhinaus können wir annehmen, dass $a$ und $b$ nicht beide gerade sind, denn sonst kann man $a$ und $b$, solange sie beide gerade sind, durch $2$ teilen, wodurch sich $a$ und $b$ um den Faktor zwei verkleinern. Es ist klar, dass dieser Prozess nach endlich vielen Schritten terminiert. 
	
	Aus $\sqrt[3]{2} = \frac{a}{b}$ folgt $2 b^3  = a^3$. Es folgt also, dass $a^3$ gerade ist. Dann ist aber nach Lemma~\ref{lem:a:a^3} die Zahl $a$ gerade und hat somit die Form $a = 2 k$ mit $k \in \N$. Dann ist $2 b^3 = a^3 = (2k)^3 = 8 k^3$, woraus $b^3= 4 k^3$ folgt. Die Zahl $b^3$ ist also gerade. Nach Lemma~\ref{lem:a:a^3}, das wir nun auf die Zahl $b$ anwenden können, ist die Zahl $b$ ebenfalls gerade. Wir haben also gezeigt, dass $a$ und $b$ beide gerade sind. Unsere Annahme war aber, dass $a$ oder $b$ ungerade ist. Dieser Widerspruch zeigt, dass die Zahl $\sqrt[3]{2}$ nicht rational ist. 
\end{proof} 


\subsection{Vollständige Induktion} 

\begin{thm}[Vollständige Induktion, Version~1] \label{thm:ind}
	Sei $P$ ein Prädikat auf $\N$. Dann sind die folgenden Aussagen äquivalent: 
	\begin{enuma}
			\item $P(n)$ gilt für alle $n \in \N$. 
			\item  $P(1)$ gilt und, aus $P(n)$ folgt $P(n+1)$, für alle $n \in \N$. 
	\end{enuma} 
\end{thm} 
\begin{proof} 
	Die Implikation (a) $\Rightarrow$ (b) ist klar: $P(1)$ ist erfüllt und da $P(n)$ und $P(n+1)$ beide wahr sind, ist die Implikation $P(n) \Rightarrow P(n+1)$ für jedes $n$ eine wahre Aussage. 
	
	Nun zeigen wir (b) $\Rightarrow$ (a) durch Kontraposition. Angenommen, (a) ist nicht erfüllt. Dann gibt es ein $n \in \N$ für welches $P(n)$ falsch ist. Wir fixieren das kleinste solche $n \in \N$. Ist unser $n=1$ so, ist (b) nicht erfüllt, weil $P(1)$ nicht erfüllt ist. Ist $n>1$ so ist (b) nicht erfüllt, weil $P(n)$ falsch und $P(n-1)$ wahr ist, wodurch die Implikation $P(n-1) \Rightarrow P(n)$ nicht erfüllt ist. 
\end{proof} 

\begin{bem}
	Beim Verwenden von Theorem~\ref{thm:ind} unterteilt sich die Argumentation in die folgenden Schritte:
	\begin{itemize}
		\item \underline{Induktionsanfang (IA):} man verifiziert, dass $P(1)$ gilt. 
		\item \underline{Induktionsvoraussetzung (IV):} man macht die Annahme: sei $n \in \N$ und sei die Aussage $P(n)$ erfüllt. 
		\item \underline{Induktionsschritt (IS):} man folgert $P(n+1)$ aus der Induktionsvoraussetzung.  
	\end{itemize} 
\end{bem} 

\begin{thm}
	Für jedes $n \in \N$ gilt 
	\[
		\sum_{i=1}^n i = \frac{1}{2} n (n+1). 
	\]
\end{thm} 
\begin{proof} 
		Das Prädikat mit dem wir uns in dieser Aussage befassen ist die Gleichung $$\sum_{i=1}^n i = \frac{1}{2} n (n+1)$$ die von einem variablen $n \in \N$ abhängig ist. 
		
\noindent\underline{IA:} Diese Formel ist für $n=1$ erfüllt, denn $\sum_{i=1}^1 i = 1$ und $\frac{1}{2} 1 \cdot (1+1) = 1$. 
		
\noindent\underline{IV:} Sei nun $n \in \N$ ein beliebiger Wert, für welche die Formel $\sum_{i=1}^n  i = \frac{1}{2} n (n+1)$ erfüllt ist.
		
\noindent\underline{IS:} Wir zeigen, dass die Formel mit $n+1$ an der Stelle von $n$ ebenfalls erfüllt ist. Es gilt 
		\[
			\sum_{i=1}^{n+1} i = \sum_{i=1}^n i  + (n+1),
		\]
		da wir in der Summe den Summanden zum Index $i=n+1$ abspalten können. Nach der Induktionsvoraussetzung ist $\sum_{i=1}^n i = \frac{1}{2} n (n+1)$. Somit hat man 
		\[
			\sum_{i=1}^{n+1} i = \frac{1}{2} n (n+1) + (n+1) = \frac{1}{2} (n+1) (n+2). 
		\]
		Zusammenfassend: Unsere Formel gilt für $n=1$ und wenn unsere Formel für ein $n \in \N$ erfüllt ist, so ist sie auch mit $n+1$ an der Stelle von $n$ erfüllt. Aus Theorem~\ref{thm:ind} folgt, dass unsere Formel für jedes $n \in \N$ erfüllt ist. 
\end{proof} 


\begin{bsp}
	Sei $q \in \R \setminus \{1\}$ und $n \in \N_0$. Dann gilt $\sum_{i=0}^n q^i = \frac{q^{n+1} - 1}{q-1}$. Zeigen Sie diese Formel durch vollständige Induktion über $n$. Hier ist der Induktionsanfang $n=0$. 
\end{bsp} 

\begin{bsp}
	Finden Sie eine Formel für $\sum_{i=1}^n i q^i$ mit $q \in \R \setminus \{1\}$ und $n \in \N_0$ und beweisen Sie diese Formel durch Induktion. 
\end{bsp} 

\begin{bem}
	Durch Induktion lassen sich nicht nur Gleichungen herleiten. Es gibt viele verschiedene Situationen in der Mathematik (und insbesondere diskreter Mathematik), in denen man durch die Induktion Aussagen verifizieren kann.
	
	Bei der theoretischen Behandlung von Algorithmen nutzt man häufig Induktionsbeweise um die Korrektheit eines Algorithmus zu beweisen oder seine Laufzeit zu analysieren.
\end{bem} 

\begin{thm}
	$n \le 2^n$ gilt für alle $n \in \N$. 
\end{thm} 
\begin{proof} 
	Diese Ungleichung kann man mit der Verwendung Ihrer Schulkenntnisse aus der Analysis herleiten. Der folgende Beweis durch die Induktion ist aber elementarer. 
	
	Die Ungleichung gilt für $n=1$, denn $1 \le 2^1$. Sei nun $n \in \N$ ein Wert, für welchen $n \le 2^n$ gilt. Im Induktionsschritt sollen wir nun $n + 1 \le 2^{n+1}$ herleiten. Da wir $n \le 2^n$ voraussetzen, gilt $n+1 \le 2^n + 1$, daher reicht es zu verifizieren, dass $2^n + 1 \le 2^{n+1}$ erfüllt ist. Die letzte Ungleichung ist äquivalent zur Ungleichung $2^n \ge 1$, die für $n \in \N$ erfüllt ist. 
\end{proof} 

\begin{aufg} 
	Zeigen Sie, dass $100 n \le 2^n$ für alle $n \in \N$ mit $n \ge 10$ erfüllt ist. 
\end{aufg} 


\begin{thm}[Vollständige Induktion, Version~2]
	\label{thm:ind:ver2}
	Sei $P$ ein Prädikat auf  $\N$. Dann sind die folgenden Aussagen äquivalent: 
	\begin{enuma}
			\item $P(n)$ gilt für alle $n \in \N$. 
			\item es gilt $P(1)$, und für jedes $n \in \N$ gilt die Implikation 
			\[
				P(1) \wedge \cdots \wedge P(n)  \Rightarrow P(n+1).
			\] 
	\end{enuma} 
\end{thm} 
\begin{proof} 
	Es gibt zwei Weisen, diese Version der vollständigen Induktion herzuleiten. Zum einen kann man den Beweis von Theorem~\ref{thm:ind} sehr geringfügig modifizieren, um dieses Theorem zu beweisen. Zum anderen kann man Theorem~\ref{thm:ind} für das Prädikat $Q(n) := P(1) \wedge \cdots \wedge P(n)$ benutzen. 
\end{proof} 

\begin{thm}[Primfaktorzerlegung - Existenz]
	\label{thm:primfakt-ex}
	Für jedes $n \in \N$ existieren  Primzahlen $p_1,\ldots,p_t$ ($t \in \N_0$), deren Produkt gleich $n$ ist, d.h.:  
	\[
		n = \prod_{i=1}^t p_i.
	\] 
\end{thm} 

\begin{proof} 
	Die Behauptung \glqq es existieren $t \in \N_0$ Primzahlen $p_1,\ldots,p_t$ mit $n=\prod_{i=1}^t p_i$\grqq\ ist wahr für $n \in\{1,2\}$. Denn $n=1$ ist Produkt $1= \prod_{i=1}^0 p_i = 1$ und im Fall $n=2$ ist $2 = \prod_{i=1}^1 p_i$ mit $p_1:=2$.
	
	Sei nun $n \in \N$ mit $n \ge 3$ so, dass jede Zahl $a \in \{1,\ldots,n-1\}$ Produkt von endlich vielen Primzahlen ist (im Sinne der Behauptung). Ist $n$ Primzahl, so gilt die Behauptung mit $t=1$ und $p_1 = n$. Ist $n$ keine Primzahl, so besitzt $n$ einen Teiler $a \in \{2,\ldots,n-1\}$. Es folgt $n = a b$ mit $b = n / a \in \N$ und $b \le \frac{n}{2} \le n-1$.
	
	Die Anwendung der Induktionsvoraussetzung auf $a$ und $b$ ergibt, dass man $a$ sowie $b$ als Produkt von Primzahlen darstellen kann. Es gilt also 
	\begin{align*}
			a & = \prod_{i=1}^{r} u_i, 
		\\	b & = \prod_{i=1}^s  v_i
	\end{align*} 
	mit $r, s \in \N$ für gewisse Primzahlen $u_1,\ldots,u_r,v_1,\ldots,v_s$ (hier ist weder $r$ noch $s$ gleich $0$, denn $a,b \ge 2$). 
	Dann ist $n$ Produkt von $t = r+s$ Primzahlen $p_1,\ldots,p_t$ mit $p_i = u_i$ für $i \in \{1,\ldots,r\}$ und $p_i = v_{i-r}$ für $i \in \{r+1,\ldots,r+s\}$. 
\end{proof} 

\begin{bem}
	Kommentar zur  vorigen Behauptung: man hat $\prod_{i=1}^0 p_i = 1$ im Fall $t=0$ und $\prod_{i=1}^1 p_i = p_1$ im Fall $t=1$. 
\end{bem} 


\begin{bem}
	Im vorigen Beweis setzen wir das Muster $P(1) \wedge \cdots \wedge P(n) \Rightarrow P(n+1)$ aus Theorem~\ref{thm:ind:ver2} als $P(1) \wedge \cdots \wedge P(n-1) \Rightarrow P(n)$ um. Das heißt, in Theorem~\ref{thm:ind:ver2} werden die schon abgearbeiteten Werte als $1,\ldots,n$ bezeichnet und der nächste Wert als $n+1$. In unserer Umsetzung werden die abgearbeiteten Werte als $1,\ldots,n-1$ bezeichnet und der nächste Wert als $n$. 
\end{bem} 


\begin{bem}[Vollständige Induktion und Algorithmen]
	Induktionsbeweise haben oft eine algorithmische Interpretation und können zu iterativen oder rekursiven Algorithmen konvertiert werden. In diesem Fall kann man sich die folgende Umsetzung des vorigen Beweises als Algorithmus vorstellen. Nehmen wir an, wir wollen für alle Zahlen $1,\ldots,N$ für ein gegebenes $N \in \N$, $N \ge 2$, deren Zerlegung in Primfaktoren berechnen. Dann können wir folgendermaßen vorgehen. Wir führen für jedes $n \in \{1,\ldots,N\}$, eine Liste $L[n]$ der Primfaktoren von $n$ ein. Für $n=1$ ist dies eine leere Liste. Angenommen, die Liste sei bereits für die Zahlen $1,\ldots,n-1$ erzeugt. Dann können wir die Liste $L[n]$ anhand der bereits vorhandenen Listen $L[1],\ldots,L[n-1]$ generieren: Wir testen, ob $n$ Primzahl ist. Ist das der Fall so erzeugen wir $L[n]$ als Liste aus einer einzigen Zahl $n$. Ist $n$ keine Primzahl, so Faktorisieren wir $n$ als $n = ab$ mit $a,b \in \{2,\ldots,n-1\}$ und erzeugen dann $L[n]$ durch das Zusammenfügen der Listen $L[a]$ und $L[b]$. 
\end{bem} 

\begin{bem}\
	
	\lstinputlisting{Code/prime_factorizations.sage}
	
\end{bem} 

\subsection{Fallunterscheidung} 

\begin{bem}
Im Beweis von Theorem~\ref{thm:primfakt-ex} haben wir zwischen den Fällen, dass $n$ eine Primzahl ist und dass $n$ keine Primzahl ist, unterschieden. In jedem der beiden Fälle gaben wir ein eigenes Argument an, wieso $n$ in Primfaktoren zerlegbar ist.  

	Eine solche \textbf{Fallunterscheidung} ist ein verbreitetes Element in Beweisen in der Mathematik.
	Die Fallunterscheidung ist dabei weniger eine eigene Beweismethode, sondern mehr eine Art einen Beweis zu organisieren.
	Verschiedene Fälle werden dabei meist durch verschiedene Argumente/Beweisverfahren bearbeitet.
\end{bem} 


\begin{bsp}
	Sei $M$ die Menge der Nutzer:innen eines sozialen Netzwerks. Auf $M$ ist eine binäre Relation \glqq befreundet sein\grqq\ definiert, die symmetrisch ist. Wir zeigen, dass in jeder $6$-elementigen Teilmenge $P \subseteq M$ drei Nutzer:innen existieren, die entweder gegenseitig befreundet sind oder gegenseitig nicht befreundet sind. 
	
	Wir fixieren zuerst ein beliebiges $p \in P$. Bezüglich $p$ zerlegt sich die $5$-elementige Menge $P \setminus \{p\}$ in die Menge $A$ derjenigen $a \in P \setminus \{p\}$, die mit $p$ befreundet sind, und der Menge $B$ derjenigen $b \in P \setminus \{p\}$, die mit $p$ nicht befreundet sind. Da $A \cup B = P \setminus \{p\}$ insgesamt $5$ Elemente hat, hat eine der beiden Mengen $A$ oder $B$ mindestens $3$ Elemente: denn hätten $A$ und $B$ beide weniger als $3$ Elemente, so hätte $A \cup B$ insgesamt höchstens $4$ Elemente, ein Widerspruch. 

	\begin{itemize} 
		\item[] \emph{Fall~1:} $A$ hat mindestens $3$ Elemente, d.h., $p$ ist mit mindestens drei Personen aus  $P \setminus \{p\}$ befreundet. 
		\begin{itemize} 
			\item[] \emph{Fall~1a:} In $A$ findet man zwei befreundete Personen $a', a''$.
			
			Dann ist $\{p,a',a''\}$ eine $3$-elementige Teilmenge von $P$ mit Personen, die gegenseitig befreundet sind. 
			\item[] \emph{Fall~1b:} In $A$ sind  keine zwei Personen befreundet. Dann sind alle Personen aus $A$ nicht gegenseitig befreundet. In $A$ hat man also drei Personen die nicht gegenseitig befreundet sind. 
		\end{itemize} 
		\item[] \emph{Fall~2:} $B$ hat mindestens $3$ Elemente, d.h., $p$ ist mit mindestens drei Personen aus $P \setminus \{p\}$ nicht befreundet. 
		\begin{itemize} 
				\item[] \emph{Fall~2a:} In $B$ findet man zwei Personen $b',b''$, die nicht befreundet sind. Dann ist $\{p,b',b''\}$ eine $3$-elementige Teilmenge von $P$ mit Personen, die gegenseitig nicht befreundet sind. 
				\item[] \emph{Fall~2b:} In $B$ sind alle Personen gegenseitig befreundet. Dann hat man in $B$ drei Personen, die gegenseitig befreundet sind. 
		\end{itemize} 
	\end{itemize} 
	In jedem der möglichen Fälle, haben wir die Existenz von drei Personen in $P$ nachgewiesen, die entweder gegenseitig befreundet oder gegenseitig nicht befreundet sind. 
\end{bsp} 

\begin{bem}[Fallunterscheidung mit Computer] 
	In manchen Beweisen in der Mathematik benötigt man so viel Fallunterscheidung (etwa zur Abarbeitung einer endlichen Anzahl von Sonderfällen), dass man es nicht mehr schafft, all die Fälle manuell zu 
behandeln. Dann holt man sich oft den Computer zur Hilfe. 

\underline{Beispiel:} Angenommen, Sie haben einen Beweis dafür, dass jede natürliche Zahl $n \in \N$ mit $n \geq 1000$ geschrieben werden kann als Summe von vier Quadraten, das hei\ss t, es gibt $a,b,c,d \in \N_0$ mit
\[
n = a^2 + b^2 + c^2 + d^2. 
\]
Dann könnten Sie mit dem Computer die Fälle $n < 1000$ testen und werden feststellen, dass die Aussage für alle $n \in \N$ gilt.
\end{bem} 

\section{Schnupperstunde in Algebra} 

\subsection{Was ist Algebra?}

\begin{bem}
	Algebra ist die Theorie algebraischer Strukturen. Während man in der Schule mit einer relativ kleinen Anzahl algebraischer Strukturen wie $(\R,+,\cdot)$ oder dem Vektorraum $\R^3$ arbeitet, befasst man sich in Algebra mit verschiedenen Kategorien algebraischer Strukturen, wie z.B. Halbgruppen, Gruppen, Ringen, Körper und Vektorräumen. 

Man entwickelt auch Mittel,  neue/eigene algebraische Strukturen anzulegen. Wenn man diesen Prozess mit der Programmierung vergleicht, so ist der Prozess sehr ähnlich zur Entwicklung eigener Datenstrukturen (im Gegensatz zur Nutzung der standard\-¸mäßig vorhandenen Datenstrukturen). 
\end{bem} 

\begin{bem}[Algebraische Struktur] 
	Eine algebraische Struktur ist in der Regel eine Menge $A$, die mit einer oder mehreren Verknüpfungen ausgestattet ist. In den allermeisten Fällen sind die Verknüpfungen, die man betrachtet, binär: sie sind Abbildungen $\ast : A \times A \rightarrow A$.  Für solche Abbildungen schreibt man dann $a \ast b$ an der Stelle von $\ast(a,b)$. Sehr oft handelt es sich auch um die Verknüpfungen, für welche (zumindest) das Assoziativgesetz $a \ast (b \ast c) = (a \ast b) \ast c$, für alle $a,b,c \in A$, erfüllt ist. 
\end{bem}	
	
\begin{bem}[Polymorphismus in Algebra] 
	Man benutzt oft zum Bezeichnen der Verknüp\-fungen (bzw. der Verknüpfung) einer algebraischen Struktur die Symbole $+$ (Plus) und $\cdot$ (Mal). Hierbei meint man dann die Plus-Operation bzw. die Mal-Operation innerhalb der gegebenen algebraischen Struktur $A$. Das heißt, diese Operationen $+$ und/oder $\cdot$ innerhalb einer algebraischen Struktur $A$ müssen nicht unbedingt etwas mit den Operationen $+$ und $\cdot$ innerhalb der Menge $\R$ der reellen Zahlen zu tun haben. Das bedeutet: genau so, wie Symbole $a,b,c,d,\ldots$ in der Mathematik kontextabhängig sind (können verschiedene Bedeutung in verschiedenen Kontexten haben), sind auch die Bezeichnungen wie $+$ und $\cdot$  kontextabhängig (bzw. strukturabhängig) und können so, wie man es sich wünscht, eingeführt werden. Wenn man also $+$ in der Struktur $A$ hat, so ist das streng genommen $+_A$ -- die Plusoperation aus der Struktur $A$ -- man schreibt aber einfach nur $+$ und nimmt stillschweigend  an, dass es aus dem Kontext klar ist, welche Struktur $A$ gemeint ist. Die Nutzung derselben Bezeichnung für verschiedene Operationen nennt man in der Programmierung den Polymorphismus.
\end{bem} 

\begin{bsp}
	Für $n \in \N$ heißt die Menge $S_n$ aller bijektiven Abbildungen von $\{1,\ldots,n\}$ nach $\{1,\ldots,n\}$ mit der Multiplikation 
	\[(\sigma \cdot \tau )(i) := \sigma(\tau(i))
	\] die symmetrische Gruppe. Was (allgemein) eine Gruppe ist, wird in IT-2 diskutiert. 
\end{bsp} 

\begin{bsp} 
	Die algebraische Struktur $\F_2$, welche man als Menge  $\{0,1\}$ mit den binären Operationen 
\begin{align*}
\begin{array}{c|cc}
	+ & 0 & 1 \\
	\hline 
	0 & 0 & 1 \\
	1 & 1 & 0
\end{array}
& & \text{und} & & 
\begin{array}{c|cc}
\cdot & 0 & 1 \\
\hline 
0 & 0 & 0 \\
1 & 0 & 1
\end{array}
\end{align*} 
einführt, ist ein sogenannter binärer Körper. Die Bezeichnungen $+$, $\cdot$, $0$ und $1$, die wir hier verwenden, sind polymorph. 

Wir meinen $+_{\F_2}, \cdot_{\F_2}$, $0_{\F_2}$ und $1_{\F_2}$, schreiben aber in unserem Kontext von $\F_2$ vereinfachend $+, \cdot, 0, 1$. 

Der binäre Körper spielt in der Kodierungstheorie und der Kryptographie eine wichtige Rolle. 

\end{bsp} 


\subsection{Kommutativer Ring} 

\begin{defn}
	Eine Menge $R$ mit zwei binären Verknüpfungen $+, -$ und zwei verschiedenen ausgezeichneten Elementen $0, 1 \in R$ heißt \textbf{kommutativer Ring}, wenn für alle $a,b,c \in R$ folgendes erfüllt ist: 
	\begin{itemize}
		\item $a + b =b +a$ und $a \cdot b = b \cdot a$ 
		\item $a + 0 = a$ und $a \cdot 1 = a$ 
		\item $(a+b)+c = a+(b+c)$ und $a \cdot (b \cdot c) = (a \cdot b) \cdot c$
		\item Zu jedem $a$ gibt es ein eindeutiges Element aus $R$, das man als $-a$ bezeichnet, für welches $a+(-a)=0$ erfüllt ist. 
		\item $a \cdot (b+c) = a \cdot b + a \cdot c$
	\end{itemize} 
\end{defn}

\begin{aufg} 
	Ist $R$ kommutativer Ring mit $1$, dann gilt $a \cdot 0=0$ für alle $a \in R$. Zeigen Sie das. 
\end{aufg} 

\begin{bsp}\ 
\begin{itemize}
		\item $(\N,+,\cdot)$ kein Ring. 
		\item $(\N_0,+,\cdot)$ (immer noch) kein Ring. 
		\item $(\Z,+,\cdot)$ ein kommutativer Ring. 
		\item $(\Q,+,\cdot)$ ein kommutativer Ring. 
		\item $(\R,+,\cdot)$ ein kommutativer Ring. 
		\item $(\C,+,\cdot)$ ein kommutativer Ring. 
\end{itemize} 
\end{bsp} 

\subsection{Körper} 

\begin{defn}
	Eine Menge $K$ mit zwei binären Verknüpfungen $+$ und $\cdot$ heißt \textbf{Körper}, wenn $K$ bzgl. $+$ und $\cdot$ kommutativer Ring ist und darüberhinaus für jedes $a \in K \setminus \{0\}$ ein eindeutiges Element $a^{-1} \in K$ existiert, für welches $a \cdot a^{-1}  = 1$ gilt. 
\end{defn} 

\begin{bsp}\ 
\begin{itemize} 
	\item $(\F_2,+,\cdot)$
	\item Führen Sie auf einer dreielementigen Menge $\{0,1,a\}$ die Verknüpfungen $+$ und $\cdot$ so ein, dass die Menge mit diesen Verknüpfungen zu einem Körper wird. 
	\item $(\Z,+,\cdot)$ kein Körper, da in $\Z \setminus \{0\}$ nichts außer $-1$ und $1$ invertierbar ist. 
	\item $(\Q,+,\cdot)$ ein Körper. 
	\item $(\R, + ,\cdot)$ ein Körper. 
	\item $(\C, + ,\cdot)$ ein Körper. 
\end{itemize} 
\end{bsp} 

\begin{defn}
	Ein Körper $K$ heißt \textbf{algebraisch abgeschlossen}, wenn für jede Wahl von $d \in \N$ und alle $a_d \in K \setminus \{0\}, a_{d-1},\ldots,a_0 \in K$ die Gleichung 
	\[
	a_d x^d + a_{d-1} x^{d-1} + \cdots + a_0 = 0
	\]
	mindestens eine Lösung $x$ aus $K$ besitzt. Eine Gleichung wie oben nennt man \textbf{Polynomgleichung} vom Grad $d$ mit Koeffizienten in $K$. 
\end{defn} 

\begin{bsp}\ 
	\begin{itemize} 
		\item $\Q$ ist nicht algebraisch abgeschlossen, vgl. die Gleichung $x^2 - 2 = 0$, mit den Koeffizienten $ 1, 0 , -2 \in \Q$, die keine Lösung $x$ in $\Q$ besitzt. 
		\item $\R$ ist nicht algebraisch abgeschlossen, vgl. die Gleichung $x^2 + 1 = 0$ mit den Koeffizienten $1, 0, 1 \in \R$, die keine Lösung $x$ in $\R$ besitzt. 
	\end{itemize} 
\end{bsp} 

\begin{defn} 
	Sind $A$ und $B$ Mengen mit $A \subseteq B$ und $\ast_A : A \times A \to A$ und $\ast_B : B \times B \to B$ binäre Verknüpfungen, so nennt man $\ast_B$ \textbf{Erweiterung} von $\ast_A$ und $\ast_A$ \textbf{Einschränkung} von $\ast_B$ auf $A$, wenn $x \ast_A y = x \ast_B y$ für alle $a,b \in A$ erfüllt ist (mit anderen Worten: $\ast_B$ wirkt genau so wie $\ast_A$ innerhalb von $A$). 
\end{defn} 

\begin{defn}
	Sind $(F,+,\cdot)$ und $(K,+,\cdot)$ Körper mit $F \subseteq K$, bei denen $+$ und $\cdot$ von $K$ Erweiterungen von $+$ bzw. $\cdot$ auf $F$ sind, so nennt man den Körper $K$ eine \textbf{Erweiterung} des Körpers $F$. 
\end{defn} 

\begin{bsp}
	$\R$ ist Erweiterung von $\Q$. Es gibt aber viele Körper dazwischen. Zum Beispiel ist 
	\[
		\Q[\sqrt{2}] := \setcond{ a + \sqrt{2} b }{a,b \in \Q}
	\]
	ebenfalls ein Körper. Es gilt $\Q \varsubsetneq \Q[\sqrt{2}] \varsubsetneq \R$. 
	Wie sieht das Inverse eines Elements aus $\Q[\sqrt{2} ] \setminus \{0\}$ aus? 
\end{bsp} 

\begin{thm}
	Jeder Körper besitzt eine algebraisch abgeschlossene Körpererwei\-terung. 	
\end{thm} 

\begin{bem}
	Es gilt sogar eine stärkere Aussage: jeder Körper besitzt eine (in einem bestimmten Sinn) minimale algebraisch abgeschlossene Körpererweiterung. 
\end{bem} 

\subsection{Der Körper der komplexen Zahlen} 

\begin{defn} 
	Die Menge $\C$ der komplexen Zahlen führen wir als die Menge der formalen Ausdrücke der Form $x +  y \, \iu$ mit $x, y \in \R$ ein. Hierbei ist $\iu$ ein formales Element, für welches wir $\iu^2 := -1$ festlegen. Das Element $\iu$ nennt man die \textbf{imaginäre Einheit} oder die \textbf{Wurzel aus $-1$}. Die Menge der reellen Zahlen $\R$ wird als eine Teilmenge von $\C$ aufgefasst, indem man $x \in \R$ als $x +  y  \, \iu$ mit $y=0$ schreibt. 
	
	Nach diesen Festlegungen lassen sich die Operationen $+$ und $\cdot$ vom Körper $\R$ der reellen Zahlen auf $\C$ auf eine eindeutige Weise erweitern, wenn man fordert, dass  $\C$ mit den Operationen $+$ und $\cdot$ ein kommutativer Ring sein soll, vgl. dazu die Gesetze für einen kommutativen Ring.  (Wie wir in Kürze sehen werden, ist $(\C,+,\cdot)$ sogar ein Körper.) Die Addition und Multiplikation führen wir also auf die folgende Weise ein:
	\begin{align*}
			(x_1 + y_1 \iu ) + (x_2 +  y_2 \iu ) & := (x_1 +y_1) +  (y_1 + y_2) \iu
			\\ (x_1 + y_1 \iu) \cdot (x_2 + y_2 \iu) & := (x_1 x_2 - y_1 y_2) + (x_1 y_2 + x_2 y_1) \iu,
	\end{align*} 
für $x_1,x_2, y_1, y_2 \in \R$. 

Ist $z = x  + y \iu$ mit $x, y \in \R$, so führen wir den \textbf{Realteil} von $z$ als $\Re(z) :=x$ und den \textbf{Imaginärteil} von $z$ als $\Im(z):= y$ ein; die Zahl $\overline{z} = x - y \iu$ nennen wir \textbf{komplex konjugiert} zu $z$; den Wert $|z| = \sqrt{x^2 + y^2}$ nennen wir den \textbf{Betrag} von $z$. 
\end{defn} 

\begin{aufg}
	Berechnen Sie $(3+ 2 \iu) ( 5 - \iu)$, indem Sie eine Darstellung dieser Zahl als $x+y \iu$ mit $x, y \in \R$ bestimmen. 
\end{aufg} 

\begin{bem}
	In Algebra werden oft Strukturen formal nach \glqq eigenen Vorgaben\grqq\ eingeführt. Bei der Definition von komplexen Zahlen sieht man ein Beispiel dafür. 
\end{bem} 

\begin{thm}
		$\C$ ist ein algebraisch abgeschlossener Körper. 
\end{thm}
\begin{proof} 
	Dass $(\C,+,\cdot)$ ein kommutativer Ring ist, lässt sich direkt verifizieren (Aufgabe). 
	
	Um zu zeigen, dass $(\C,+,\cdot)$ sogar ein Körper ist, muss man verifizieren, dass jedes $z = x + y \iu$ mit $x, y \in \R$ mit $|z| \ne 0$ ein inverses Element in $\C$ besitzt. Es stellt sich heraus, dass man das inverse Element $z^{-1}$ als $z = \frac{1}{|z|^2} \bar{z}$ beschreiben kann. Mit der Verwendung der dritten binomischen Formel erhalten wir 
	
	\[
		 	z z^{-1} = \frac{z \bar{z} }{|z|^2} = \frac{ ( x+ y \iu) ( x - y\iu) }{x^2 + y^2}   = \frac{ x^2 - (y \iu)^2 }{x^2 + y^2} = \frac{x^2 - y^2 \iu^2}{x^2 + y^2} = \frac{x^2 + y^2}{x^2 + y^2} = 1. 
	\]
	
	Dass der Körper $(\C,+,\cdot)$ algebraisch abgeschlossen ist, ist ziemlich bemerkenswert. Bedenken Sie, dass wir nur die imaginäre Einheit $\iu$, also eine formale Lösung der Polynomgleichung $z^2 + 1=0$ in einem unbekannten $z$, eingeführt haben. Die Behauptung über die algebraische Abgeschlossenheit ist, dass wir durch diese Ergänzung für eine beliebige Polynomgleichung von einem positiven Grad (und mit Koeffizienten in $\C$) eine Lösung in $\C$ finden. Um diese Behauptung herzuleiten braucht man Wissen aus der Analysis (wir geben also an dieser Stelle keinen Beweis). 
\end{proof} 

\begin{aufg}
	Zeigen Sie $|u \cdot v| = |u| \cdot |v|$ für alle $u, v \in \C$. \textbf{Hinweis:} Am besten zeigt man $|u \cdot v|^2 = |u|^2  \cdot |v|^2$, um die Wurzeln zu vermeiden. 
\end{aufg} 

\subsection{Exkurs: Trigonometrie} 

\begin{thm}[Der Satz des Pythagoras]
	In einem rechtwinkligen Dreieck seien $a$ und $b$ die Längen der am rechten Winkel anliegenden Seiten und $c$ die Länge der dem rechten Winkel gegenüberliegenden Seite. Dann gilt $c^2 = a^2 + b^2$. 
\end{thm} 

\begin{kor} \label{kor:pythagoras}
	Der Abstand zwischen dem Punkt $(0,0) \in \R^2$ und dem Punkt $(x,y) \in \R^2$ ist gleich $\sqrt{x^2 + y^2}$. 
\end{kor} 
\begin{proof} 
 	Im Fall $x =0$ oder $y=0$ ist die Behauptung klar, da sich der Punkt $(x,y)$ auf einer der beiden Koordinatenachsen befindet. Im Fall $x \ne 0$ und $y \ne 0$ ist das Dreieck mit den Ecken $(0,0), (x,0), (x,y)$ rechtwinklig und die Längen der am rechten Winkel anliegenden Seiten sind $|x|$ und $|y|$. Nach dem Satz des Pythagoras ergibt das die Länge $\sqrt{|x|^2 + |y|^2} = \sqrt{x^2 + y^2}$ für den Abstand zwischen $(0,0)$ und $(x,y)$. 
\end{proof} 

\begin{defn}
	Sei $c \in \R^2$ und $\rho>0$. Dann heißt die Menge aller Punkte, die zum Punkt $c$ den Abstand $\rho>0$ haben, \textbf{Kreis} mit Zentrum in $c$ vom Radius $\rho>0$. 
\end{defn} 

\begin{bem}
	Aus Korollar~\ref{kor:pythagoras} folgt: die Menge 
	\[
		\setcond{(x,y) \in \R^2}{x^2 + y^2 = 1}
	\] ist Kreis vom Radius $1$ mit Zentrum in $(0,0)$. Diese Menge nennen wir im folgenden den \textbf{Einheitskreis}. 
\end{bem} 

\begin{bem}[Über Radianten und Grade] 
	Im alten Babylonien dachte man, das Jahr wäre 360 Tage lang (das stimmt nicht, wie wir jetzt wissen). Daher teilte man den Jahreskreis in 360 Teile auf, die den Tagen entsprechen sollten. Ein Grad steht daher für einen Tag im babylonischen Jahreskreis. Das zeigt, dass die Herkunft der Messung der Winkel in Graden nicht mathematisch ist. Sie ist anthropologisch: sie hängt mit dem Planeten Erde zusammen, auf dem wir uns befinden, und mit den Babylonier:innen, die bei der Bestimmung der Anzahl der Tage im Jahr sich ein wenig verschätzten. Dennoch hat sich die Messung mit 360 Graden für den vollen Winkeln bis jetzt erhalten. Das liegt vielleicht daran, dass einige für uns interessante Winkel mit Graden durch eine ganze Zahl darstellbar sind ($90^\circ$, $60^\circ$, $30^\circ$). 
	
	Die Messung mit Radianten ist eine dimensionslose Messung und sie ist intrinsisch mathematisch. Man nimmt einen Kreis mit dem Radius $1$ und misst Winkel durch die Längen der Bögen dieses Kreises. Dabei bezeichnet man die Länge einer Hälfte des Einheitskreises als $\pi$ und nennt die Zahl $\pi$ die Kreiszahl. Diese Zahl $\pi$ ist etwas größer als $3$ (das sieht man, wenn man in den Einheitskreis ein reguläres Sechseck einschreibt). 
	
	Ein Grad ist nichts Anderes als 
	\[
		1^\circ := \frac{\pi}{180}= \frac{2\pi}{360}.
	\] Wenn man Winkel in Radianten misst, kann man etwa $1{.}2$ Radianten aber auch einfach nur $1{.}2$ sagen, denn die Einheit Radiant ist dimensionslos. 
	
	An sich gibt es an der Messung der Winkel mit Graden nichts Falsches. Dieser Kommentar dient einfach nur dazu, darauf hinzuweisen, dass die Zahlen wie $360$ und $180$ in Bezug auf die Winkelmessung keine mathematische sondern eine anthropologische Natur haben.
\end{bem} 

\begin{bem}[Kosinus, Sinus und Co.] 
 Man betrachte eine kreisförmige Radrennbahn mit Zentrum im Punkt $(0,0)$ vom Radius $1$. Diese Bahn ist nach dem Satz des Pythagoras durch die Gleichung $x^2 + y^2 = 1$ beschrieben. Nun legen wir den Punkt $(1,0)$ dieser Bahn als den Startpunkt fest. Von diesem Punkt aus kann man nun Strecken einer beliebigen Länge zurücklegen. Wie lang die Strecke ist und ob man sich im Gegenuhrzeigersinn oder im Uhrzeigersinn bewegt wird durch eine Zahl $\alpha \in \R$ notiert. Der Betrag von $\alpha$ gibt die Länge der Strecke an, die man zurücklegen will. Das Vorzeichen von $\alpha$ gibt an, ob man sich im Gegenuhrzeigersinn oder im Uhrzeigersinn bewegt: bei einem positiven Vorzeichen + im Gegenuhrzeigersinn und bei einem negativen Vorzeichen - im Uhrzeigersinn. Für jedes $\alpha \in \R$ erhält man einen Punkt $(x,y)$, den man erhält, wenn man sich nach dem Zurücklegen der vorgegebenen Strecke in die vorgegebene Richtung, ausgehend vom Punkt $(1,0)$, bewegt. Die $x$-Komponente dieses Punkts nennt man den Kosinus von $\alpha$ und die $y$-Komponente den Sinus von~$\alpha$. Die Bezeichnungen dazu sind: 
 \begin{align*} 
 	x & = \cos \alpha
 	\\ y & = \sin \alpha 
 \end{align*}	
Ansonsten betrachtet man noch den Tangens 
\[
	\tan \alpha = \frac{\sin \alpha}{\cos \alpha}
\]
und den Kotangens 
 \[
 	\cot \alpha = \frac{\cos \alpha}{\sin \alpha}.
 \] 
 Ist $0 \le \alpha \le \pi$, so sagt man bei $x =\cos \alpha$, dass $\alpha$ Arcus Kosinus von $x$ ist. Man schreibt dann $\alpha = \arccos x$. Mit anderen Worten ist $\arccos$ die Umkehrfunktion vom Kosinus, wenn man den Kosinus als Funktion $[0,\pi] \to [-1,1]$ auffasst. 
 Ist $-\pi \le \alpha \le \pi$, so sagt man bei $y = \sin \alpha$, dass $\alpha$ Arcus Sinus von $y$ ist. Man schreibt dann 
 $\alpha = \arcsin y$. Mit anderen Worten ist $\arcsin$ die Umkehrfunktion der Einschränkung vom Sinus, wenn man den Sinus als Funktion $[-\pi,\pi] \to [-1,1]$ auffasst. 
\end{bem} 

\begin{bem}[Kosinus und Sinus im Taschenrechner] Wenn Studierende den Kosinus und Sinus (unter anderem für sehr einfache Werte $\alpha$) im Taschenrechner berechnen, so sieht man, dass es immer wieder dazu kommt, dass ihre Ergebnisse falsch sind. Das liegt daran, dass man in vielen Taschenrechnern eine Umschaltung zwischen Grad und Radianten hat. Ist der Taschenrechner auf Radianten eingestellt, so berechnet er den eigentlichen  Kosinus und Sinus, wie sie in der Mathematik (und in den meisten Programmiersprachen) zu finden sind. Ist der Taschenrechner auf Grade eingestellt, so berechnet er die Funktionen $t \mapsto \cos( \frac{\pi}{180}t)$ und $t \mapsto \sin( \frac{\pi}{180})t$ an der Stelle von $\cos$ und $\sin$. Übrigens: in Excel wird die Funktion $t \mapsto \frac{\pi}{180} t$, die oben in $\cos$ und $\sin$ eingesetzt wurde, das Bogenmaß von $t$ genannt. 
\end{bem} 

\begin{bem} 
	Die vielen Formeln, die man für den Kosinus und Sinus und andere trigonometrische Funktionen hat, lassen sich im Rahmen der linearen Algebra (IT-3) viel besser verstehen. 
\end{bem} 

\subsection{Eulersche Formel} 

\begin{thm} \label{thm:z:betr:arg}
	Jede komplexe Zahl $z \in \C$ besitzt eine Darstellung als 
	\[
		z = \rho ( \cos \phi + \iu \sin \phi )
	\]
	mit $\rho \in \R_{\ge 0}$ und $\phi \in \R$. Hierbei gilt $\rho = |z|$. Bei $z \ne 0$, ist $\phi$ eindeutig durch $z$ bis auf das Addieren eines ganzzahligen Vielfachen von $2 \pi$ definiert. 
\end{thm}

\begin{proof}  
	{\color{red} WARNUNG:} Der nachfolgende Beweis und unsere Definition von $\cos$ und $\sin$ entspricht nicht ganz den mathematischen Standards, solange wir den Begriff  Länge (eines Bogens) und Orientierung (einer Kurve), auf den  wir uns bei der Einführung von $\cos$ und $\sin$ beziehen, nicht mathematisch formal definiert haben. Wir verlassen uns also auf Intuition und darauf, dass man (später) den Begriff Länge mathematisch korrekt einführen kann (solche Begriffe führt man in der Analysis ein). Es gibt auch einen formalen nicht-geometrischen Zugang zum Kosinus und Sinus (dieser Zugang ist aber nicht wirklich intuitiv, sodass man dadurch nicht wirklich versteht, was Kosinus und Sinus eigentlich sind). 
	
	Da jede komplexe Zahl $z = x + y \iu$ eindeutig durch $x, y \in \R$ gegeben ist, kann man $z$ als einen Punkt $(x,y) \in \R^2$ visualisieren. Die Visualisierung von $\C$ auf diese Weise nennt man die gaußsche Zahlenebene. Dabei werden $1$ und $\iu$ als die zueinander senkrechten Vektoren $(1,0)$ und $(0,1)$ dargestellt. Man sieht, dass die Menge $K := \setcond{z \in \C}{|z|=1} = \setcond{ x+ \iu y}{x^2 + y^2 =1}$ als  der Einheitskreis mit Zentrum in $0 \in \C$ und dem Radius $1$ in der gaußschen Zahlenebene darstellbar ist. 
	
	\emph{Existenz:} Ist $z \ne 0$, so ist $z / |z|$ ein Punkt auf dem Kreis $K$ und so hat $z$ die Darstellung $ z / |z| = \cos \phi + \iu \sin \phi$ für ein $\phi \in \R$ nach unserer Beschreibung von $\cos$ und $\sin$. Es folgt also, dass $z = \rho ( \cos \phi + \iu \sin \phi)$ mit $\rho = |z|$ gilt. Im Fall $z= 0 \in \C$ kann man $\rho =0$ und ein beliebiges $\phi$ fixieren. 
	
	\emph{Eindeutigkeit:} Ist $z = \rho (\cos \phi + \iu \sin \phi)$ mit $\rho \in \R_{\ge 0}$ und $\phi \in \R$ so gilt $|z| = | \rho (\cos \phi + \iu \sin \phi) | = \rho | \cos \phi + \iu \sin \phi| = \rho \sqrt{ \cos^2 \phi + \sin^2\phi } = \rho$. Ist $z \ne 0$, so ist 
	$ z/ |z|$ der Punkt $\cos \phi + \iu \sin \phi$ auf dem Einheitskreis $K$. Der Punkt $\cos \phi + \iu \sin \phi$ auf dem Kreis $K$ ändert sich nicht, wenn man zum Wert von $\phi$  ein ganzzahliges Vielfaches von $2 \pi$ dazu addiert, weil der Kreis $K$ die Länge $2\pi$ hat. So besteht die Möglichkeit als $\phi$ einen Wert aus $[0,2 \pi)$ zu wählen. 
	
	 Da $K$ die Länge $2\pi$ hat, ist jeder Punkt eindeutig durch die Angabe eines solchen $\phi \in [0,2 \pi)$ gegeben. 
\end{proof} 

\begin{defn}[Definition der Exponentialfunktion durch die Euler-Formel]
	\label{def:euler:formel}
	Wir erweitern die Exponentialfunktion $e^x : \R \to \R$ in der Variablen $x \in \R$ zur \textbf{Exponentialfunktion} $e^z : \C \to \C$ in der Variablen $z \in \C$, indem wir 
	\[
			e^{x+ \iu y} := e^x ( \cos y + \iu \sin y)
	\]
	für alle $x, y \in \R$ festlegen. (Insbesondere gilt im Fall $x=0$ die Gleichung $e^{\iu y} = \cos y + \iu \sin y$ laut unserer Definition). 
\end{defn} 

\begin{bem}
	Jede Zahl $z \in \C$ besitzt eine Darstellung $z = \rho e^{\iu \phi}$ mit $\rho = |z| \in \R_{\ge 0}$ und $\phi \in \R$. Das ist die Umformulierung von Theorem~\ref{thm:z:betr:arg} in den neu eingeführten Bezeichnungen. 
\end{bem} 

\begin{defn} 
	In der (eindeutigen) Darstellung  der Zahl $z \in \C^2 \setminus \{0\}$ als $z = \rho e^{\iu \phi}$ mit $\rho \in \R_{>0}$ und $\phi = (-\pi,\pi]$  wird $\phi$ das \textbf{Argument} von $z$ genannt und als $\Arg(z)$ bezeichnet. 
\end{defn} 

\begin{aufg}
	Zeigen Sie, dass bei $z_k := \rho_k e^{\iu \phi_k}$ mit $\rho_1,\rho_2 \in \R_{\ge 0}$ und $\phi_1,\phi_2 \in \R$ die Gleichung $z_1 z_2 =  \rho_1 \rho_2 e^{\iu (\phi_1 + \phi_2)}$ erfüllt ist.  Benutzen Sie dafür die Formeln für $\cos(\alpha \pm \beta)$ und $\sin(\alpha \pm \beta)$. 
	
	Wenn Sie diese Formeln nicht kennen bzw. nicht gefunden haben, dann gibt es eine alternative Aufgabe: Leiten Sie aus der Tatsache, dass $z_1 z_2 = \rho_1 \rho_2 e^{\iu(\phi_1 + \phi_2)}$ gilt, Formeln für $\cos(\alpha \pm \beta)$ und $\sin(\alpha \pm \beta)$ her. 
	
	\textbf{Kommentar:} Formeln für $\cos(\alpha \pm \beta)$ und $\sin(\alpha \pm \beta)$ kann man im Rahmen der Linearen Algebra (IT-2) herleiten. 
\end{aufg} 

\begin{bem}[Merkhilfe für trigonometrische Formeln mit Hilfe von $e^{z} : \C \to \C$]
	Es stellt sich heraus, dass sich die Rechenregeln für $e^x : \R\to \R$ direkt auf die Rechenregel in $e^z : \C \to \C$ übertragen lassen. Das Übertragen der Rechenregel im Fall einer reellen Variablen auf den Fall einer komplexen Variablen basiert auf den (zahlreichen) trigonometrischen Formeln, die man kennt. 
	
	Umgekehrt gilt: die Rechenregel für $e^z : \C \to \C$ in Kombination mit der Euler-Formel \glqq speichern\grqq\ die trigonometrischen Formeln. Das bedeutet, dass man sich die trigonometrischen Formeln durch die Euler-Formel merken kann. 
	
	Stellen wir uns vor, wir haben die Formeln für $\cos 2 \alpha$ und $\sin 2 \alpha$ vergessen. Was wir tun können, ist Folgendes. Es gilt: 
	\[
			e^{2 \alpha \iu} = (e^{\iu \alpha} )^2. 
	\]
	Die Anwendung der Euler-Formel für die linke und rechte Seite ergibt dann 
	\[
		\cos 2 \alpha + \iu \sin 2 \alpha  = (\cos \alpha + \iu \sin \alpha)^2. 
	\]
	Für die rechte Seite können wir die zweite binomische Formel anwenden. Das ergibt: 
	\[
		\cos 2 \alpha + \iu \sin 2 \alpha  = \cos^2 \alpha + 2\sin \alpha \cos \alpha  \iu + \iu^2 \sin^2 \alpha. 
	\]
	Nach einer Vereinfachung der rechten Seite erhalten wir 
	\[
	\cos 2 \alpha + \iu \sin 2 \alpha  = (\cos^2 \alpha - \sin^2 \alpha ) + 2\sin \alpha \cos \alpha \iu
\]
Auf diese Weise kommen wir zu den Formeln
\begin{align*}
		\cos 2 \alpha  & = \cos^2 \alpha - \sin^2 \alpha, 
		\\ \sin 2 \alpha & = 2 \sin \alpha \cos \alpha. 
\end{align*} 
\end{bem} 

\begin{aufg} 
	Welche trigonometrischen Identitäten speichern die folgenden Formeln? 
	\begin{align*}
			e^{\iu \alpha} e^{- \iu \alpha} & =1, 
			\\ e^{\iu (\alpha + \beta)} & = e^{\iu \alpha} e^{\iu \beta},
			\\ e^{- \iu \alpha} & = (e^{\iu \alpha})^{-1}. 
	\end{align*} 
\end{aufg} 

\begin{bem}
	A priori ist nicht klar, wieso unsere Definition der Exponentialfunktion $e^z : \C \to \C$ sinnvoll ist. Um zu verstehen, dass das die einzige richtige Weise ist, diese Funktion zu definieren, braucht man Wissen aus der Analysis (IT-3), und zwar braucht  man Potenzreihen dafür. 
\end{bem} 

\begin{defn} 
	In Anlehnung an Definition~\ref{def:euler:formel} erweitern wir $\sin, \cos : \R \to \R$ zu Funktionen $\sin, \cos : \C \to \C$, indem wir 
	\begin{align*}
			\cos z  & : = \frac{e^{\iu z} + e^{-\iu z}}{2}
			\\ \sin z & := \frac{e^{\iu z } - e^{ -\iu z}}{2 \iu}
	\end{align*} 
	festlegen. 
\end{defn} 

\begin{bem}
	Die Euler-Formel ermöglicht einen Übergang von den trigonometrischen Funktionen zur Exponentialfunktion. So ein Übergang, der oft die Berechnung  erleichtert, wird gerne in der Physik/Elektrotechnik benutzt, wenn man sich mit Schwingungen und Wellen beschäftigt. 
\end{bem} 

\begin{aufg}\ 
	Schlagen Sie nach, wie die inversen trigonometrischen Funktionen Arcus Kosinus, Arcus Sinus und Arcus Tangens definiert sind. Der Wertebereich  vom Arcus Kosinus ist $[0,2\pi]$. Der Wertebereich vom Arcus Sinus ist dagegen in $[-\pi,\pi]$. Begründen Sie, warum die Wertebereiche so gewählt sind. \textbf{Hinweis:} Die inversen trigonometrischen Funktionen sollen eigentlich die trigonometrischen Funktionen umkehren. Was braucht man für Eigenschaften für die Umkehrbareit? Sind diese Eigenschaften für die trigonometrischen Funktionen erfüllt? 
\end{aufg} 

\begin{aufg}\
	\begin{itemize}
			\item Sei $n \in \N$. Was sind alle Lösungen der Gleichung $z^n =1$ in einer komplexwertigen Unbekannten $z$? Wenn Sie Schwierigkeiten im Fall eines allgemeinen $n$ haben, versuchen Sie zuerst die Lösungen in den Fällen $n \in \{1,2,4\}$ zu finden. 
			\item Bestimmen Sie alle Lösungen der Gleichung $z^2 + 2 z +  2 = 0$ in einer komplexwertigen Unbekannten $z$. 
			\item Wie kann das Konjugieren $z \mapsto \overline{z}$ geometrisch als eine Transformation  auf der komplexen Ebene $\C$ beschrieben werden? 
	\end{itemize} 
\end{aufg} 


\section{Asymptotische Notation}

\subsection{O, $\Omega$ und $\Theta$}

\begin{bem}
Bei der Analyse von Algorithmen und in der Analysis redet man oft von der Größenordnung von Funktionen. Eine praktische Ausdrucksweise dafür ist die sogenannte asymptotische Notation (oder auch Landau-Notation).
\end{bem} 

\begin{defn}[$O$-Notation]  
Seien $f, g: \N \to \R$ Funktionen. 
Man schreibt $f(n) = O(g(n))$, wenn eine Konstante $c>0$ und ein $n_0 \in \N$ existiert, so dass $|f(n)| \le c |g(n)|$ für alle $n \ge n_0$ gilt. 
\end{defn} 

\begin{bem} 
Die Bezeichnung $f(n)=O(g(n))$ steht für \glqq $f(n)$ hat die Größenordnung höchstens $g(n)$ bis auf eine multiplikative Konstante\grqq\ und man sagt \glqq$f(n)$ ist in Groß-O von $g(n)$\grqq.

Die Schreibweise $f(n) = O(g(n))$ ist streng genommen nicht ganz korrekt, in der Literatur aber sehr verbreitet. Die korrekte Schreibweise wäre $f(n) \in O(g(n))$, d.h., $f(n)$ liegt in der Menge aller Funktionen der Größenordnung höchstens $g(n)$.

In der Literatur verwendet man oft $O(g(n))$ als eine Schreibweise für eine anonyme Funktion der Größenordnung höchstens $g(n)$. In diesem Kurs spielen die Beträge in der Definition von $O(g(n))$ in der Regel keine Rolle, weil wir beim Anwenden der asymptotischen Notationen fast ausschließlich nichtnegative Funktionen benutzen. 
\end{bem} 

\begin{defn}[$\Omega$-Notation] 
Für zwei Funktionen $f, g: \N \to \R$ schreibt man $f(n) = \Omega(g(n))$, wenn eine Konstante $c>0$ und ein $n_0 \in \N$ existieren, so dass $|f(n)| \ge c |g(n)|$ für alle $n \ge n_0$ gilt.

In diesem Fall: Die Größenordnung von $f(n)$ ist mindestens $g(n)$, bis auf eine multiplikative Konstante und man sagt \glqq$f(n)$ ist in Groß-Omega von $g(n)$\grqq. 
\end{defn} 

\begin{defn}[$\Theta$-Notation] 
Für zwei Funktionen $f, g: \N \to \R$ schreibt man $f(n) = \Theta(g(n))$, wenn sowohl $f(n) = O(g(n))$ als auch $f(n) = \Omega(g(n))$ gelten.

In diesem Fall: Die Größenordnung von $f(n)$ ist genau $g(n)$ bis auf eine multiplikative Konstante, und man sagt \glqq$f(n)$ ist in Groß-Theta von $g(n)$\grqq.
\end{defn} 


\begin{bem}
Die asymptotischen Notationen $O(g(n))$, $\Omega(g(n))$ und $\Theta(g(n))$ (und ihre weiteren Varianten) werden oft auch Landau-Symbole genannt.
\end{bem} 

\begin{bsp}
	Sei $f : \N \to \R$ definiert durch $f(n):=\sqrt{2 n + 5 } - 10$. Es gilt $f(n) = \Theta(\sqrt{n})$, denn einerseits ist $\sqrt{2n + 5} - 10 \le \sqrt{2n + 5} \le \sqrt{ 7n} = \sqrt{7} \sqrt{n}$ für alle $n \in \N$, woraus $f(n) = O(\sqrt{n})$ folgt. Andererseits ist $\sqrt{2n + 5} - 10 \ge \sqrt{n} - 10 \ge \frac{1}{2} \sqrt{n}$ für alle $n \ge 400$, woraus $f(n) = \Omega(\sqrt{n})$ folgt.  
\end{bsp}

\begin{aufg}
	Sind die folgenden asymptotischen Abschätzungen richtig?
	\begin{itemize}
		\item $n! = O(n^n)$
		\item $n^n = \Omega(n!)$
		\item $n! = O(2^n)$
		\item $n^n = O(n!)$
	\end{itemize}
\end{aufg}


\begin{bem}
	Seien $f_1,f_2,g_1,g_2 : \N \to \R$ Funktionen, wobei $g_1,g_2$ nicht-negativ sind und $f_i(n) = O(g_i(n))$, für $i=1,2$, vorausgesetzt wird. Dann gilt
	\[
	f_1(n) + f_2(n) = O(g_1(n)+g_2(n)) = O(\max\{g_1(n),g_2(n)\}),
	\]
	und
	\[
	f_1(n) \cdot f_2(n) = O(g_1(n)\cdot g_2(n)).
	\]
\end{bem}

\subsection{$o$ und $\omega$}

\begin{defn}[$o$-Notation] 
Bei $g : \N \to \R$ steht $o(g(n))$ für die Menge aller Funktionen $f: \N \to \R$ mit der Eigenschaft, dass für \textbf{jedes} $c>0$ ein $n_0 \in \N$ existiert derart, dass $|f(n)| \le c |g(n)|$ für alle $n \in \N$ mit $n \ge n_0$ erfüllt ist.

Wir schreiben dann $f(n) = o(g(n))$ beziehungsweise $f(n) \in o(g(n))$, und sagen \glqq $f(n)$ liegt in Klein-o von $g(n)$\grqq.
\end{defn} 

\begin{defn}[$\omega$-Notation]
Die Bezeichnung $\omega(g(n))$ steht für die Menge aller Funktionen $f : \N \to \R$ mit der Eigenschaft, dass für \textbf{jedes} $c>0$ ein $n_0 \in \N$ existiert derart, dass $|f(n)| \ge c|g(n)|$ für alle $n \in \N$ mit $n \ge n_0$ erfüllt ist.

Wir schreiben dann $f(n) = \omega(g(n))$ beziehungsweise $f(n) \in \omega(g(n))$, und sagen \glqq $f(n)$ liegt in Klein-$\omega$ von $g(n)$\grqq.
\end{defn} 
