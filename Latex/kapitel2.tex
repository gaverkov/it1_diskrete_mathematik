\chapter{Kombinatorik} 

\section{Vereinigung und Kreuzprodukt von zwei Mengen} 

\begin{bem}
	Die Haupfrage der Kombinatorik ist ``Wie viele Elemente hat meine endliche Menge?''. Etwas formaler geht es um die Formeln für die Anzahl der Elemente verschiedener endlicher Mengen, welche man in der diskreten Mathematik gerne benutzt. 
\end{bem} 

\begin{lem} \label{lem:vereinigung:zwei}
	Seien $A, B$ endliche disjunkte Mengen. Dann ist $|A \cupdot B| = |A| + |B|$. 
\end{lem} 

\begin{proof}
	Ist $A$ oder $B$ leer, so gilt die Formel trivialerweise. Etwa, für $B = \emptyset$ gilt 
	\[
		|A \cup B| = | A \cup \emptyset| = |A| = |A| + 0 = |A| + | \emptyset| = |A| + |B|. 
	\]
	
	 Sonst nummerieren wir alle Elemente von $A$ als $a_1,\ldots,a_s$ und $B$ als $b_1,\ldots,b_t$. Das heißt, $A$ ist eine $s$-Elementige Menge $A = \{a_1,\ldots,a_s\}$ und $B$ ist eine $t$-elementige Menge $B = \{b_1,\ldots,b_t\}$ mit $s,t \in \N$. Es gilt $a_i \ne a_j$ für $i,j \in \{1,\ldots,s\}$ mit $i \ne j$   und $b_i \ne b_j$ für  $i,j \in \{1,\ldots,t\}$ mit $i \ne j$. Da $A$ und $B$ disjunkt sind gilt auch $a_i \ne b_j$ für alle $i \in \{1,\ldots,s\}$ und $j \in \{1,\ldots,t\}$. 
	 
	 Somit ist 
	 $A \cupdot B = \{a_1,\ldots,a_s,b_1,\ldots,b_t\}$, sodass wir $A \cupdot B = \{c_1,\ldots,c_n\}$ mit $n = s + t$ und 
	 \[
	 	c_i = \begin{cases}
	 		 a_i, & \text{für} \ i \in\{1,\ldots,s\}
	 		 \\b_{i-s}  &  \text{für} \ i \in \{s+1,\ldots,n\}
	 		\end{cases}.
 	\] 
 	Hierbei sind $c_1,\ldots,c_n$ nach der Konstruktion paarweise verschieden. Das zeigt, dass $A \cupdot B$ genau $n = s+t$ Elemente hat. 
\end{proof}

\begin{lem} \label{lem:disjunkte:vereinigung}
	Seien $A_1,\ldots,A_n$ ($n \in \N$) endliche paarweise disjunkte Mengen. Dann ist 
	\[
		\left| \bigcup_{i=1}^n A_i \right| = \sum_{i=1}^n |A_i|. 
	\]
\end{lem} 

\begin{proof} 
	Wir beweisen die die Formel durch Induktion über $n$. 
	Die Formel ist rivial für $n=1$, denn $\bigcup_{i=1}^1 A_i = A_1$ und $\sum_{i=1}^n |A_i|$ ist $|A_1|$. 
	Sei $n \in \N$ mit $n \ge 2$ gegeben und sei die Formel im Fall von $n-1$ an der Stelle von $n$ Mengen bereits verifiziert. Da die Mengen $A_1 \cup \cdots \cup A_{n-1}$ und $A_n$ paarweise disjunkt sind, erhalten wir durch die Anwendung von Lemma~\ref{lem:vereinigung:zwei} zu diesen beiden Mengen, dass 
	\[
		| A_1 \cup \cdots \cup A_n| = |A_1 \cup \cdots \cup A_{n-1} | + |A_n| 
	\]
	erfüllt ist. Aus der Induktionsvoraussetzung folgt, dass 
	\[
		| A_1 \cup \cdots \cup A_{n-1} | = \sum_{i=1}^{n-1} |A_i|
	\]
	erfüllt ist. Somit ist 
	\[
		| A_1 \cup \cdots \cup A_n| = \sum_{i=1}^{n-1} |A_i|  + |A_n| = \sum_{i=1}^n |A_i|. 
	\]
\end{proof} 

\begin{lem} \label{lem:inkl:exkl:2}
	Seien $A$ und $B$ endliche Mengen. Dann gilt 
	\[
		|A \cup B| = |A| + |B| - |A \cap B|. 
	\]
\end{lem}

\begin{proof}
		Wir können $A \cup B$ als dijsunkte Vereinigung von $A$ und $B \setminus A$ darstellen. Die Anwendung von Lemma~\ref{lem:vereinigung:zwei} zu $A$ und $B \setminus A$ ergibt 
		\[
			| A \cup B| = |A \cupdot (B \setminus A)| = |A | + |B \setminus A|. 
		\]
		Die Menge $B$ ist disjunkte Vereinigung von $B \setminus A$ und $A \cap B$. Die Anwendung von Lemma~\ref{lem:vereinigung:zwei} zu $A \cap B$ und $B \setminus A$ ergibt
		\[
			|B| = |B \setminus A| + |A \cap B|. 
		\]
		Aus den beiden Gleichungen, die wir auf diese Weise herleiten, folgt dann 
		\[
			| A \cup B|  = |A | + |B \setminus A| = |A| + (|B| - | A \cap B|) = |A| + |B| - | A \cap B|. 
		\]
\end{proof} 

\begin{bem}
	Die Intuition hinter Lemma~\ref{lem:inkl:exkl:2} ist: wir zählen alle Elemente in $A$ sowie $B$ ab. Dadurch werden die Elemente in $A \cap B$ doppelt abgezählt. Wir sollen also die Anzahl der Elemente in $A \cap B$ abbziehen, um auf die Anzahl der Elemente in $A \cup B$ zu kommen. 
\end{bem} 


\begin{bsp}
	Wie viele dreistellige Zahlen gibt es, bei denen zwei benachbarte Stellen gleich sind? 
	
	Wir identifizieren die dreistelligen Zahlen mit den  den Tripeln $(s_1,s_2,s_3)$ mit $s_1 \in \{1,\ldots,9\}$ und $s_2,s_3 \in \{0,\ldots,9\}$. Daher arbeiten wir innerhalb der Menge
	\[
			X = \{1,\ldots,9\} \times  \{0,\ldots,9\} \times \{0,\ldots,9\}
	\]
	Dann entspricht die Menge 
	\[
		A := \setcond{ (s_1,s_2,s_3) \in X}{s_1=s_2}
	\]
	der Menge der dreistelligen Zahlen, bei denen die $100$er und die $10$er Stelle gleich sind, und die Menge
	\[
		B:= \setcond{ (s_1,s_2,s_3)}{s_2=s_3}
	\]
	der Menge der dreistelligen Zahlen, bei denen die $10$er und die $1$er Stelle gleich sind. Wir sind also an der Anzahl der Elemente in der Vereinigung $A \cup B$ interessiert. Wegen 
	\[
			| A \cup B| = |A| + |B|  - |A \cap B|
	\]
	reicht es aus, die Anzahl der Elemente in $A$, $B$ und $A \cap B$ zu bestimmen. 	
	Der Durchschnitt 
	\[	
		A \cap B = \setcond{ (s_1,s_2,s_3) \in X}{s_1=s_2=s_3}
	\]
	entspricht der Menge der dreistelligen Zahlen, bei denen alle drei Stellen gleich sind. Die Menge $A$ hat genau $9 \cdot 10$ Elemente. Das lässt sich folgendermaßen verifizieren:  
	\begin{itemize} 
		\item es gibt für die $100$er Stelle $9$ Möglichkeiten hat;
		\item unabhängig von der $100$er Stelle gibt es für die $1$er Stelle $10$ Möglichkeiten
		\item Die $10$er Stelle bei den Zahlen, die den Elementen aus $A$ entsprechen, ist gleich der $100$er Stelle. 
	\end{itemize} 
	Ganz analog sieht man, dass die Menge $B$ ebenfalls genau $9 \cdot 10$ Elemente hat. Die Menge $A\cap B$ hat $9$ Elemente. Es folgt: 
	\[
		|A \cup B| = 90 + 90 - 9 = 171. 
	\]	
	Hier ein Python-Code, der all die Möglichkeiten auflistet: 
	\lstinputlisting{Code/komb171.py}
\end{bsp} 


\begin{bsp}
	In einer Klasse haben $15$ Kinder eine Playstation, $10$ Kinder eine Xbox und $3$ Kinder die beiden genannten Geräte. Wie viele Kinder haben mindestens eines der beiden Geräte? 
	
	Sei $P$ die Menge der Kinder, die eine Playstation haben, $X$ die Anzahl der Kinder, die eien Xbox haben. Dann gilt: 
	\[
		|P \cup X| = |P| + |X| - |P \cap X| = 15 + 10 - 3 = 22. 
	\]
\end{bsp} 

\begin{bem}[Kombinatorik und Wahrscheinlichkeiten] 
	Die Berechnung von Wahrscheinlichkeiten reduziert sich im Fall von endlichen Wahrscheinlichkeitsräumen oft zur Berechnung von Quotienten 
	\begin{align*}
			\frac{\text{der Anzahl der günstigens Fälle}}{\text{die Anzahl aller Fälle}}
	\end{align*} 
	uns somit zur Lösung von kombinatorischen Aufgaben, vgl. auch \cite{Tit19}. 
\end{bem} 

\begin{aufg}
	Alice wirf zwei Spielwürfel und Bob wirft einen Spielwürfel. Mit welcher Wahrscheinlichkeit ist die Augenzahl bei Bob gleich einer der Augenzahlen bei Alice? 
\end{aufg} 

\begin{lem} \label{lem:anzahl:prod:2}
	Seien $A$ und $B$ endliche Menge. Dann gilt: 
	\[
		|A \times B| = |A| \cdot |B|. 
	\]
\end{lem}
\begin{proof} 
	Ist $A$ oder $B$ leer, so sind die linke sowie rechte Seite der Gleichung gleich $0$. Ansonsten stellen wir die Menge $A \times B$ kann als disjunkte Vereinigung $\bigcup_{a \in A} \{a\} \times B$ da. Lemma~\ref{lem:disjunkte:vereinigung} ergibt
	\[
		|A \times B| = \left| \bigcup_{a \in A} \{a \} \times B \right| = \sum_{a \in A} | \{a \} \times B |.
	\]
	Für ein beliebiges festes $a \in A$ kann nun die Anzahl der Elemente in $\{a\} \times B$ bestimmt werden. Diese Anzahl ist $|B|$, da die Abbildung $f_a : B \to \{a\} \times B$ mit $f_a(b) := (a,b)$ bijektiv ist: denn hat man zwei verschiedene Elemente $b',b'' \in B$ so sind auch $(a,b')$ und $(a,b'')$ verschieden (Injektivität) und hat man ein beliebiges Element aus $\{a\} \times B$ fixiert, etwa $(a,b)$ mit $b \in B$, so erhält man dieses Element als $f_a(b) = (a,b)$. 
\end{proof} 

\begin{bsp}
	In einem Kinosaal hat man $10$ Reihen mit $16$ Plätzen pro Reihe. Jeder Platz kann also durch die Angabe der Reihe und die Nummer des Platzes in der Reihe als das Paar $(r,n)$ mit $r \in \{1,\ldots,10\}$ und $n \in \{1,\ldots,16\}$ notiert werden. Die Anzahl der Plätze ist 
	\[
			| \{1,\ldots,10\} \times \{1,\ldots,16\} |  = | \{1,\ldots,10\} | \times | \{1,\ldots,16\}| = 10 \cdot 16 = 160. 
	\]
	Dieses Beispiel ist genauso einfach wie unser Lemma~\ref{lem:anzahl:prod:2}, das die Idee hinter dem Beispiel ganz allgemein erfasst. 
\end{bsp} 
 

\section{Kartesisches Produkt} 

\begin{thm} \label{thm:kreuz:produkt}
	Seien $A_1,\ldots, A_n$ endliche Mengen ($n \in \N$). Dann ist 
	\[
		|A_1 \times \cdots \times A_n|  = \prod_{i=1}^n |A_i|. 
	\]
\end{thm} 
\begin{proof}
	Wir beweisen die Gleichung durch Induktion über $n$.
	Für $n=1$ erhalten wir eine triviale Identät. Sei die Gleichung für $n-1$ Mengen für ein $n \in \N$ mit $n \ge 2$ efüllt. Dann erhält man wegen 
	\[
			A_1 \times \cdots \times A_n = A_1 \times (A_2 \times \cdots \times A_n)
	\]
	durch die Anwendung von Lemma~\ref{lem:anzahl:prod:2} die Gleichung 
	\[
		|A_1 \times \cdots \times A_n | = |A_1| \cdot |A_2 \times \cdots \times A_n|. 
	\]
	Anschließend erhalten wir aus der Induktionsvoraussetzung 
	\[
		|A_2 \times \cdots \times A_n| = \prod_{i=2}^n |A_i|,
	\]
	woraus sich die gewünschte Gleichung für die Mengen $A_1,\ldots,A_n$ ergibt. 
\end{proof} 

\begin{bem}[Identifikation  in der Mathematik: das Selbe oder das Gleiche?]
	In der Mathematik wird oft eine stillschweigende Identifikation von Objekten vorgenommen. Genauer sind  zwischen manchen paaren von Mengen natürliche Bijektionen vorhanden, auf deren Basis diese Mengen identifiziert werden. Zum Beispiel kann man für eine Menge $X$ die kartesische erste Potenz $X^1$ mit $X$ identifizieren, weil man ein einelementiges Tupel $(x)$ mit $x \in X$ mit $x \in X$ identifizieren kann. Man geht also davon aus, dass $(x)$ das Selbe wie $x$ ist. Beim Programmieren dagegen erfolgt eine solche Identifikation nicht immer automatisch: man soll dann Objekte verschiedener Datentypen explizit in einander konvertieren. 
	
	Im vorigen Beweis haben wir $(a_1,(a_2,\ldots,a_n))$, ein Paar, bei dem die zweite Komponente ein $(n-1)$-Tupel ist, stillschweigend mit $(a_1,\ldots,a_n)$ identifiziert. 
\end{bem} 

\begin{kor} \label{kor:product} 
	Sei $A$ endliche Menge und $n \in \N$. Dann gilt $|A^n| = |A|$. 
\end{kor}
\begin{proof}
	Die Behauptung folgt durch die Anwendung von Theorem~\ref{thm:kreuz:produkt} im Fall, dass alle Mengen $A_1,\ldots,A_n$ gleich $A$ sind. 
\end{proof}  

\section{Abbildungen} 

\begin{thm}
	Seien $X,Y $ endliche Mengen. Dann gilt $|Y^X| = |Y|^{|X|}$, wobei man bei $X= Y = \emptyset$, $|Y|^{|X|} = 0^0  =1$ setzt. 
\end{thm}
\begin{proof}
	Im entarteten Fall $X = \emptyset$ gibt es nur eine Abbildung von $\emptyset$ nach $Y$. 
	Sei $X \ne\emptyset$ und sei $n := |X|$, sodass wir alle Elemente von $x$ als $x_1,\ldots,x_n$ indexieren können. Dann entspricht jede Abbildung $f : X \to Y$ einem $n$-Tupel $(f(x_1),\ldots,f(x_n)) \in Y^n$, wobei zwei verschiedene Abbildungen von $X$ nach $Y$ zwei verschiedene Tupel erzuegen. 
	Umkgekehrt definiert jedes $n$-Tupel $(y_1,\ldots,y_n)$ die Abbildung $f: X \to Y$ mit $f(x_i)  = y_i$ für alle $i \in \{1,\ldots,n\}$. Man sieht also, dass die Abbildung 
	\[
			Y^X \to Y^n,
	\]
	die einem $f : X \to Y$ das Tupel $(f(x_1),\ldots,f(x_n))$ zuordnet, bijektiv ist. Wir erhalten mit der Verwendung von Korollar~\ref{kor:product}
	\[
			|Y^X| = |Y^n| = |Y|^n = |Y|^{|X|}. 
	\]
\end{proof} 

\begin{bsp}
	Wieviele Möglichkeiten gibt es drei verschiedene Aufgaben unter vier Personen zu verteilen, wenn jede Aufgabe genau einer Personen zugeordent wird? 
	
	Eine Zuordnung der Aufgaben den Personen ist eine Abbildung aus einer $3$-elementigen Menge $X$ von  Aufgaben in die $4$-elementige Menge $Y$ der Personen. Wir zählen also die Abbildungen aus $Y^X$. Die Anzahl ist $|Y^X| = |Y|^{|X|} = 4^3 = 64$. 
\end{bsp} 

\begin{bem}\
	\lstinputlisting{Code/Abb.py} 
	Demo dazu: 
	\lstinputlisting{Code/Abb_demo.py}
\end{bem}

\section{Injektive und bijektive Abbildungen} 

\begin{defn}[Fakultät]
	Für $n \in \N_0$ ist \textbf{$n$ Fakultät} als 
	\[n ! := \prod_{i=1}^n i
	\] definiert. Insbesondere gilt $0!=1!=1$. 
\end{defn} 

\begin{defn}
	Für $n, k \in \N_0$ definieren wir die \textbf{fallende Faktorielle} von $n$ der Länge $k$ als
	\[
		n^{\underline{k}} := n \cdot \ldots \cdot (n-k+1).
	\]
	Man hat insbesondere $n^{\underline{0}}=1$. 
\end{defn} 

\begin{defn}
	Für Mengen $X, Y$ bezeichnen wir als $\Inj(X,Y)$ die Menge aller injektiven und als $\Bij(X,Y)$ die Menge aller bijektiven Abbildungen von $X$ nach $Y$. 
\end{defn} 

\begin{thm} \label{thm:inj:abbildungen}
	Seien $X,Y$ endliche Mengen. Dann ist $|\Inj(X,Y) | = |Y|^{\underline{|X|}}$. 
\end{thm} 
\begin{proof} 
	Wenn es eine injektive Abbildung $f$ von $X$ existiert $Y$ exisitert, so gilt $|X| \le |Y|$, denn $f(X)$ ist eine Teilmenge von $Y$, die genau so viele Elemente wie $X$ hat. Das bedeutet, dass die Formel iim $|X| > |Y|$ erfüllt ist, weil in diesem Fall die linke sowie rechte Seite gleich $0$ ist. Wir beweisen die Formeln im Fall $|X| \le |Y|$ durch Induktion über $|X|$. Für $X = \emptyset$ gibt es genau eine injektive Abbildung von $X= \emptyset$ nach $Y$. Also gilt die Formel für $|X|=0$. 
	Nun betrachten wir $X$ und $Y$ mit $k=|X|$ und $k \le |Y|$ und nehmen an, dass die Formel im Fall $k-1=|X| \le |Y|$ bereits verifiziert wurde. Wir fixieren ein beliebiges $a \in X$. Jede injektive Abbildung $f$ von $X$ nach $Y$ bildet das fixierte $a$ auf eines der Elemente aus $Y$ ab. Wir können also die Menge $\Inj(X,Y)$ als disjunkte Vereinigung 
	\[
			\Inj(X,Y) = \bigcup_{b \in Y} \Inj_{a,b}(X,Y),
	\]
	mit 
	\[
		\Inj_{a,b} (X,Y) := \setcond{ f \in \Inj(X,Y)}{f(a) = b}. 
	\]
	Nach Lemma~\ref{lem:disjunkte:vereinigung} gilt 
	\[
			|\Inj(X,Y)| = \sum_{b \in Y} |\Inj_{a,b}(X,Y)|
	\]
	Jede Abbildung $f  \in \Inj_{a,b}(X,Y)$ erzeugt die Abbildung $\tilde{f} : X \setminus \{a\} \to Y \setminus \{b\}$. Da $f$ injektiv ist, ist $\tilde{f}$ ebenfalls injektiv. Auf diese Weise haben wir die Abbildung 
	\[
				f \mapsto \tilde{f} 
	\]
	von $\Inj_{a,b}(X,Y)$ nach $\Inj(X \setminus \{a\}, Y \setminus \{b\})$ erstellt. Die Abbildung $f$ ist offensichtlich eine Bijektion, sodass man $|\Inj_{a,b}(X,Y)| = |\Inj(X \setminus \{a\},Y \setminus \{b\})|$ hat. Nach der Induktionsvoraussetzung ist 
	\[
		|\Inj(X \setminus \{a\}, Y \setminus \{b\} | = |Y \setminus \{b\}|^{\underline{|X \setminus \{a\}}} = (|Y| -1)^{\underline{|X|-1}}.
	\] 
	Es folgt 
	\begin{align*}
	|\Inj(X,Y)| & = \sum_{b \in Y} |\Inj_{a,b}(X,Y)| 
		\\ & = \sum_{b \in Y} (|Y| -1)^{\underline{|X|-1}} 
		\\ & = |Y| \cdot (|Y| -1)^{\underline{|X|-1}}
		\\ & = |Y|^{\underline{|X|}}. 
	\end{align*}
\end{proof} 

\begin{bsp}
		In einer Deutsch-Stunde sollen  $3$ von $25$ Schüler:innen ein Gedicht vortragen. Wie viele Möglichkeit gibt es dafür, wenn man die Reihenfolge, in der man vorträgt, berücksichtigt? Wir zählen injektive Abbildungen von $\{1,2,3\}$ in die Menge der $25$ Schüler:innen. Die Antwort: 
		\[
			25^{\underline{3}} = 25 \cdot 24 \cdot 23. 
		\]
\end{bsp} 

\begin{bem}\
\lstinputlisting{Code/Inj.py} 
\end{bem}

\begin{kor}
	 Seien $X$ und $Y$ endliche Mengen der gleichen Kardinalität $n$. Dann gilt $|\Bij(X,Y)| = n!$
\end{kor} 
\begin{proof}
	Haben endliche Mengen $X$ und $Y$ die gleiche Kardinalität, so gilt die Gleichheit $\Bij(X,Y ) = \Inj(X,Y)$. Die Inklusion $\Bij(X,Y ) \subseteq \Inj(X,Y)$ ist trivial, weil jede Abbildung nach der Definition injektiv ist. Umgekehrt: Ist $f: X \to Y$ injektiv, so hat $Y \setminus f(X)$ genau $|Y| - |f(X)| = |Y| - |X| =0$ Elemente. Das bedeutet, $Y = f(X)$, sodass $f$ auch surjektiv ist. Das zeigt die Inklusion $\Inj(X,Y) \subseteq \Bij(X,Y)$. 
	
	Nach dieser Bemerkung folgt die Behauptung direkt aus Theorem~\ref{thm:inj:abbildungen}. 
\end{proof} 

\begin{bsp}
	Wie viele Möglichkeiten gibt es, $10$ verschiedene Bücher in einem Regal anzuordnen? Wir zählen bijektive Abbildungen von $\{1,\ldots,10\}$ in die Menge von $10$ Büchern. Die Antwort: 
	\[
			10!
	\]
\end{bsp} 

\begin{bem} Der Demo-Code 
\lstinputlisting{Code/Inj_demo.py}
ergibt $6$ Möglichkeiten. 
\end{bem} 

\section{Teilmengen} 

\begin{defn}
Für eine Menge $X$ xund $k \in \N_0$ bezeichnen wir als $\binom{X}{k}$ die Menge aller $k$-elementigen Teilmengen von $X$. 
\end{defn}

\begin{bsp}
	\[
	\binom{\{1,2,3,4\}}{2} = \Bigl\{ \{1,2\},\{1,3\}, \{1,4\}, \{2,3\}, \{2,4\} ,\{3,4\} \Bigr\}.
	\]
\end{bsp} 


\begin{defn}
	Für $n,k \in \N_0$ wird der \textbf{Binomialkoeffizient  $n$ über $k$} als 
	\[
		\binom{n}{k}:= \frac{n^{\underline{k}}}{k!}  = \frac{n \cdot \ldots \cdot (n-k+1)}{k!}.
	\]
	definiert. 
\end{defn} 

\begin{bem}
	Im Fall $0 \le k \le n$ und $n,k \in \N_0$ hat man $\binom{n}{k} = \frac{n!}{k! (n-k)!}$. 
\end{bem} 


\begin{thm} \label{thm:k:elem:teilmengen} 
		Sei $X$ endliche Menge und sei $k \in \N_0$. Dann gilt: 
		\[
				\left| \binom{X}{k} \right| = \binom{|X|}{k}. 
		\]
\end{thm} 
\begin{proof} 
	In den Fällen $k > |X|$ werden die beiden Seiten der Formel gleich $0$. Wir nehmen also $k \le |X|$ an. Ist $k=0$, werden die beiden Seiten der Formeln gleich $1$. Wir nehmen also $0 < k \le |X|$ an. 
	
	Wir zählen die injektiven Abbildungen von $I:=\{1,\ldots,k\}$ nach $X$ auf die folgende  Weise auf. Bei jeder injektiven Abbildung $f : I \to X$ ist das Bild $f(I)$ eine $k$-elementige Teilmenge von $X$. Also kann man die injektiven Abbildungen $f: I \to X$  nach der Wahl von $f(I)$ gruppieren. Mit anderen Worten ist $\Inj(I,X)$ disjunkte Vereinigung
	\[
		\Inj(I,X)  = \bigcup_{B \in \binom{X}{k} } \setcond{f \in \Inj(I,X)}{f(I) = B}. 
	\]
	Aus Lemma~\ref{lem:disjunkte:vereinigung} folgt
	\[
		|\Inj(I,X) | = \sum_{B \in \binom{X}{k} } | \setcond{f \in \Inj(I,X)}{f(I) = B}|. 
	\]
	Jeder ijectiven Abbildung $ f: I \to X$ mit einem vorgeschriebenen Bild $B$ 
	die bijektive Abbildung $\tilde{f} : I \to B$ mit $\tilde{f}(i) = f(i)$ zuordnen kann, und die Abbildung $f \mapsto \tilde{f}$ von $\setcond{f \in \Inj(I,X)}{f(I) = B}$ nach $\Bij(I,B)$ ist bijektiv. 
	Wir erhalten also 
	\[
			| \setcond{f \in \Inj(I,X)}{f(I) = B}| = |\Bij(I,B)| = k!. 
	\]
	Das ergibt
	\[
		|\Inj(I,X) | = \left| \binom{X}{k} \right| \cdot k!
	\]
	Nach Theorem~\ref{thm:inj:abbildungen} hat man 
	\[
		|\Inj(I,X)| = |X|^{\underline{k}}. 
	\]
	Es folgt: 
	\[
		\left| \binom{X}{k} \right| = \frac{|\Inj(I,X) |}{k!} = \frac{|X|^{\underline{k}}}{k!} = \binom{|X|}{k!}. 
	\]
\end{proof} 

\begin{bem}[Doppeltes Abzählen] 
	Im Beweis des  Theorems~\ref{thm:k:elem:teilmengen} haben wir injektive Abbildungen zwischen zwei festgelegten Mengen auf eine andere Weise als im Beweise des Theorems~\ref{thm:inj:abbildungen} abgezählt. Aus den beiden Weisen abzuzählen ergab sich dann im Beweis des Theorems~\ref{thm:inj:abbildungen} für die Anzahl der $k$-elementigen Teilmengen einer gegebenen Menge. Solchen Beweisansatz nennt man in der Kombinatorik \textbf{Doppeltes Abzählen}. 
\end{bem} 

\begin{aufg}
	Für alle $n,k \in \N_0$ gilt $\binom{n}{k} = \binom{n}{n-k}$. Verifizieren Sie das direkt arithmetisch und kombinatorisch, indem man eine Bijektion zwischen 
	den Mengen $\binom{X}{k}$ und $\binom{X}{|X|-k}$ für eine $n$-elementige Teilmenge $X$ erstellt. 
\end{aufg} 


\begin{thm}
	Sei $X$ endliche Menge. Dann gilt $|2^X|=2^{|X|}$. 
\end{thm} 
\begin{proof} 
	Es gibt verschiedene Ansätze zum Beweis dieser Formel. Zum Beispiel kann man eine natürliche Bijektion zwischen  $2^X$ und $\{0,1\}^X$ erstellen und dann $|\{0,1\}^X| = |\{0,1\}|^{|X|} = 2^{|X|}$ nutzen (Aufgabe). 
	
	Wir präsentieren hier einen Beweis durch Induktion über $|X|$. Hat $X$ $0$ Elemente, so gilt $2^X = 2^\emptyset = \{\emptyset\}$. Somit ist $|2^X| = 1 = 2^0 = 2^{|X|}$. Sei $X$ Mengen mit $n$ Elementen, mit $n \ge 1$, und sei die Formel für Mengen $X$ mit $n-1$ Elementen bereits verifiziert. Wir fixieren ein $a \in X$. Die Teilmengen $A$ von $X$ zerlegen sich nach den Bedingungen $a \in A$ und $a \in A$ in zwei disjunkte Mengen. Es gilt also
	\begin{align*}
			2^X & = \setcond{A}{A \subseteq X, \ a \in A} \cupdot \setcond{A}{A \subseteq X, \ a \not\in A}
			\\ & = \setcond{B \cup \{a\}}{B \subseteq X \setminus \{a\}} \cupdot 2^{X \setminus \{a\}}. 
	\end{align*}
	Das ergibt
	\[
		|2^X| = |\setcond{B \cup \{a\}}{B \subseteq X \setminus \{a\}}| + | 2^{X \setminus \{a\}}|. 
	\]
	Die Abbildung $B \mapsto B \cup \{a\}$ ist eine Bijektion von $2^{X \setminus \{a\}}$ nach $\setcond{B \cup \{a\}}{B \subseteq X \setminus \{a\}}$. Es folgt: 
	\[
		|2^X| = 2 \cdot |2^{X \setminus \{a\}}|
	\]
	Da $X \setminus \{a\}$ eine $(n-1)$-elementige Menge ist, folgt nach der Induktionsovraussetzung $|2^{X \setminus \{a\}}| = 2^{|X \setminus \{a\}|} = 2^{|X|-1}$. Wir erhalten somit $2^{|X|} = 2 \cdot 2^{|X|-1} = 2^{|X|}$. 
\end{proof} 

\begin{thm}[Binomischer Lehrsatz] 
	Für $x, y \in \R$ und $n \in \N$ gilt 
	\[
			(x+y)^n = \sum_{i=0}^n \binom{n}{i} x^i y^{n-i}. 
	\]
\end{thm} 
\begin{proof}
	Es gibt einen direkten kombinatorischen Beweis dieser Formel. Beim Auflösen der Klammern im Ausdruck 
	\[
			(x+y)^n = \underbrace{(x+y) \cdots \ldots \cdot (x+y)}_{n \ \text{mal}}
	\]
	erhält man $2^n$ Terme, indem man aus jeder der $n$ Klammern $x$ oder $y$ wählt und die so gewählten Werte miteinander multipliziert. Wenn wir durch $I$ die Nummern der Klammern bezeichnen, aus denen der Wert $x$ gewählt wurde, erhalten wir 
	\[
			(x+y)^n = \sum_{I \subseteq \{1,\ldots,n\}} x^{|I|} y^{n - |I|}. 
	\]
	Nun können wir die Teilmengen von $\{1,\ldots,n\}$ nach ihrer Kardinalität Gruppieren. Wir erhalten 
	\begin{align*}
			(x+y)^n & =\sum_{i=0}^n \sum_{I \in \binom{\{1,\ldots,n\}}{i}} x^i y^{n-i} 
			\\ & = \sum_{i=0}^n \left| \binom{\{1,\ldots,n\}}{i} \right| x^i y^{n-i}. 
			\\ & = \sum_{i=0}^n \left| \binom{n}{i} \right| x^i y^{n-i}
	\end{align*} 
	
	wird aus jeder der $n$ Klammern entweder $x$ oder $y$ gewählt und a
\end{proof} 

\begin{thm} \label{pascals:triangle} 
	Für alle $n \in \N$ und $i \in \N_0$ hat man $\binom{n}{i}  = \binom{n-1}{i}  + \binom{n-1}{i-1}$. 
\end{thm} 
\begin{proof} 
	Wir beschränken uns auf den nichttrivialen Fall $1 \le i \le n$. 
	Man kann einen rein arithmetischen Beweis anhand der expliziten Formel für die Binomailkoeffizienten geben. Alternativ geben wir hier einen kombinatorischen Beweis. Sei $X$ eine $n$-elementige Menge. Wir fixieren ein Element $p\in X$ in dieser Menge. Die Menge $\binom{X}{i}$ der $i$-elementigen Teilmengen ist disjunkte Vereinigung 
	\[
		\binom{X}{i} = \setcond{ A \in \binom{X}{i}}{p \in A} \cupdot \setcond{A \in \binom{X}{i}}{p \not\in A}. 
	\]
	Hierbei ist 
	\[
			\setcond{A \in \binom{X}{i}}{p \not\in A} = \binom{X \setminus \{p\}}{i}
	\]
	und $\setcond{ A \in \binom{X}{i}}{p \in A}$ kann bijektiv auf 
	\[
			\binom{X \setminus \{p\}}{i-1}
	\]
	abgebildet werden, indem man $A$ auf $A \setminus \{p\}$ abbildet. Auf diese Weise erhält man 
	\begin{align*}
		\binom{n}{i} = \left| \binom{X}{i} \right| & = \left| \binom{X \setminus \{p\}}{i} \right| + \left| \binom{X \setminus \{p\}}{i-1}\right| 
		\\ & = \binom{n-1}{i} + \binom{n-1}{i-1}. 
	\end{align*} 
\end{proof} 

\begin{bem}
	Die Formel in Theorem~\ref{pascals:triangle} wird oft als das sogenannte pascalsche Dreieck visualisiert. 
\end{bem} 

\begin{aufg}
	Beweisen Sie den binomischen Lehrsatz durch Induktion über $n$ mit der Verwendung von Theorem~\ref{pascals:triangle}. 
\end{aufg} 

\section{Das Prinzip der Inklusion-Exklusion} 

\begin{defn}
	Seien $A,X$ Mengen mit $A \subseteq X$. Dann nennen wir die Funktion $1_A : X \to \R$ mit 
	\[
		1_A(x): = \begin{cases}
				1 &  \text{für} \ x \in A, 
			\\	0 & \text{für} \ x \not\in A
			\end{cases} 
	\]
	die \textbf{charakteristische Funktion} von $A$ auf der Grundmenge $X$. (In der Regel ist die Wahl der Grundmenge aus dem Kontext klar, daher ist $X$ nicht direkt in der Bezeichnung $1_A$ vorhanden.)
\end{defn} 

\begin{bem}
	Sind $A, X$ endliche Menge mit $A \subseteq X$ so ist $|A| = \sum_{x \in A} 1_A(x)$. Darüber hinaus gilt $1_{X \setminus A} = 1_X - 1_A$. 
\end{bem}

\begin{lem}
	Seien $A_1,\ldots,A_n$ ($n \in \N$) endliche Teilmengen einer endlichen Menge $X$. Dann gilt 
	\[
			1_{A_1 \cap \cdots \cap A_n}  = 1_{A_1} \cdot \ldots \cdot 1_{A_n}. 
	\]
\end{lem} 
\begin{proof}
	Sei $x \in X$. 
	Liegt $x$ im Durchschnitt der Mengen $A_1,\ldots,A_n$ so gilt $1_{A_1 \cap \cdots \cap A_n} (x) =1$  aber auch $1_{A_i}(x)=1$ für jedes $i \in \{1,\ldots,n\}$. Die Auswertung der Funktionen auf der linken und rechten Seite auf $x$  ergibt somit den Wert $1$> 
	Ist $x$ nicht im Durchschnitt der Mengen $A_1,\ldots,A_n$ so gilt $1_{A_1 \cap \cdot A_n}(x) =0$ aber auch $1_{A_i}(x)=1$ für ein $i \in \{1,\ldots,n\}$. Die Auswertung der Funktionen auf der linken rechten Seite der Formel ergibt somit den Wert $0$. 
\end{proof} 

\begin{lem} \label{lem:product} 
	Sei $n \in \N$. Dann gilt für alle $t_1,\ldots,t_n \in \R$ die Gleichung 
	\[
		(1 - t_1) \cdot \ldots \cdot  (1-t_n) = \sum_{I \subseteq \{1,\ldots,n\}} (-1)^{|I|} \prod_{i \in I} t_i.  
	\]
	(Deutung der rechten Seite: die Summe über alle Teilmengen $I$ von $\{1,\ldots,t\}$ und das Produkt aller $t_i$ mit $i \in I$). 
\end{lem} 
\begin{proof} 
	Die Gleichung folgt durch das Ausmultiplizieren. Aus jeder der $n$ Klammern au fer linken Seite der Formel kann  beim Ausmultiplizieren unabhängig der Term $1$ oder der Term $-t_i$ gewählt werden. Die Menge $I$ kodiert also die Wahl der Terme in den Klammern durch die Angabe der Klammern, in denen man den Term $-t_i$ gewählt hat. Man hat also 
	\[
			(1 - t_1) \cdot \ldots \cdot  (1-t_n) = \sum_{I \subseteq \{1,\ldots,n\}}  \prod_{i \in I} (-t_i)  = \sum_{I \subseteq \{1,\ldots,n\}} (-1)^{|I|} \prod_{i \in I} t_i.
	\]
	Einen formaleren Beweis kann man zum Beispiel durch Induktion über $n$ führen (Aufgabe). 
\end{proof} 

\begin{thm}[Das Prinizip der Inklusion-Exklusion]
	Seien $A_1,\ldots,A_n$ ($n \in \N$) endliche Mengen. Dann gilt 
	\[
		| A_1 \cup \cdots \cup A_n| = \sum_{\emptyset \ne J \subseteq \{1,\ldots,n\}} (-1)^{|J|+1} \left| \bigcap_{j \in J} A_i \right| 
	\]
\end{thm} 
\begin{proof} 
	Wir setzen $X := A_1 \cup \cdots \cup A_n$ und betrachten die charakteristischen Funktionen von $A_1,\ldots,A_n$ auf der Grundmenge $X$. Dann gilt die Identität
	\[
		  (1-1_{A_1} )  \cdots (1 -1_{A_n}) = 0
	\]
	auf der Menge $x$; denn für jedes $x \in X$ hat man $x \in A_i$ für ein $i \in \{1,\ldots,n\}$, woraus sich $1 - 1_{A_i}(x) = 0$ ergibt. 
	
	Aus Lemma~\ref{lem:product} folgt dann 
	\[
		 0 = \sum_{I \subseteq \{1,\ldots,n\}} (-1)^{|I|} \prod_{i \in I} 1_{A_i} = \sum_{i \in I} (-1)^{|I| } \, 1_{\bigcap_{i \in I} A_i},
	\]
	wobei wir hier $\bigcup_{i \in \emptyset} A_i$ als $X$ interpretieren. 
	Daraus folgt: 
	\begin{align*}
		0  & = \sum_{x \in X} \sum_{i \in I} (-1)^{|I| } \, 1_{\bigcap_{i \in I} A_i}(x)
		\\ & = \sum_{I \subseteq \{1,\ldots,n\} } (-1)^{|I| } \sum_{x \in X} 1_{\bigcap_{i \in I} A_i}(x)
		\\ & = \sum_{I \subseteq \{1,\ldots,n\}} (-1)^{|I| } \left| \bigcap_{i \in I} A_i \right|
	\end{align*}
	Wir spalten in der vorigen Summer den Summanden zu $I = \emptyset$ ab und erhalten 
	\[
		0  = |X| + \sum_{\emptyset \ne I \subseteq \{1,\ldots,n\}} (-1)^{|I| } \left| \bigcap_{i \in I} A_i \right|,
	\]
	was zur Formel in der Behauptung des Theorems äquivalent ist. 
\end{proof} 

\section{Multimengen} 

\begin{defn}
Ist $X$ eine Menge, so ist eine \textbf{Multimenge} $M$ über der Grundmenge $X$ durch die Angabe der \textbf{Vielfachheitsabbildung} $\mu_M : X \to \N_0$ gegeben. Die Summe
\[
	\sum_{x \in X} \mu_M(x)
\]
nennt man die Kardinalität der Multimenge $M$ und bezeichnet diesen Wert als $|M|$.  Wir nennen die Menge $\setcond{ x \in X}{\mu_M(x)>0}$ den \textbf{Träger} von $M$.

Zwei Multimengen $M$ und $N$ nennen wir gleich, wenn sie den selben Träger $T$ haben, und für jedes $x \in T$ die Vielfachheiten von $x$ in $M$ und $N$ übereinstimmen.
\end{defn} 

\begin{bem}[Angabe von Multimengen durch die Aufzählung]
	Bei der Angabe der Multimengen durch die Aufzählung benutzen wir die Bezeichnung $\{x_1,\ldots,x_n\}_{\ne}$. So ist z.B. 
	\[
			M = \{1,1,1,1,2,2,2,3\}_{\ne}
	\]
	eine Multimenge über $\{1,2,3\}$ mit $\mu_M(1) =4, \mu_M(2)=3$ und $\mu_M(3)=1$. Wir benutzten in der Bezeichnung für Multimengen das untergestellte $\ne$, um explizit darauf hinzuweisen, dass es sich hierbei nicht um eine Menge handelt. In vielen Quellen benutzt man aber genau die gleiche Bezeichnung wie bei Mengen. 
\end{bem}

\begin{defn}
	Für $k \in \N_0$ bezeichnen wir als $\mbinom{X}{k}$ die Menge aller $k$-elementigen Multimengen, deren Träger in  $X$ enthalten ist. 

	Für $n,k \in \N_0$ führen wir die Bezeichnung $\mbinom{n}{k} := \binom{n+k-1}{k}$ ein. 
\end{defn} 

\begin{lem} \label{lem:multimengen}
	Seien $n \in \N$ und $k \in \N_0$. Dann gilt: 
	\[
		\bigl| \setcond{ (z_1,\ldots,z_n) \in \N_0^n}{z_1 + \cdots + z_n = k} \bigr| = \mbinom{n+k-1}{k}. 
	\]
\end{lem} 
\begin{proof} 
	Die Abbildung $(z_1,\ldots,z_n) \mapsto (y_1,\ldots,y_n):=(z_1+1,\ldots,z_n+1)$ ist eine Bijektion von 
	\[
			Z:=\setcond{ (z_1,\ldots,z_n) \in \N_0^n}{z_1 + \cdots + z_n = k}
	\]
	nach 
	\[
			Y:= \setcond{ (y_1,\ldots,y_n) \in \N^n}{y_1 + \cdots + y_n = n+ k}. 
	\]
	Es reicht also aus, die Anzahl der Elemente in $Y$ zu bestimmen. Eine Wahl von $(y_1,\ldots,y_n)$ lässt sich so veranschaulichen. Man zeichnet $n+k$ Punkte von links nach rechts. Zwischen diesen Punkt gibt es $n+k-1$ Lücken. Setzt man in genau $n-1$ Lücken Striche, so zerlegen sich Punkte in $n$. Bezeichnet man als $y_i$ die Anzahl der Punkte in $i$-te Gruppe, so erhält $(y_1,\ldots,y_n) \in Y$. Umgekehrt erzeugt jedes $(y_1,\ldots,y_n) \in Y$ eine Angabe, welche der $n+k-1$ Lücken mit Strichen gefüllt werden sollen. Daraus ergibt sich 
	\[
			|Z| = |Y|= \binom{n+k-1}{n-1} = \binom{n+k-1}{k}. 
	\]
	Die informelle Begründung mit Punkten, Lücken und Strichen kann mathematisch formal beschrieben werden. Es handelt sich um die Bijektion 
	\[
		(y_1,\ldots,y_n) \mapsto (y_1,y_1+y_2,\ldots, y_1+ \cdots +y_{n-1})
	\] von $Y$ nach 
	\(
			X:=\binom{\{1,\ldots,n+k-1\}}{n-1}. 
	\)
	(Aufgabe: Überprüfen Sie, dass es tatsächlich eine Bijektion ist.) 
\end{proof} 

\begin{thm}
	Sei $X$ endliche Menge und sei $k \in \N_0$. Dann gilt 
	\[
			\left| \mbinom{X}{k} \right| = \mbinom{|X|}{k}. 
	\]
	Mit anderen Worten: es gibt genau $\binom{n+k-1}{k}$ Multimengen mit $k$ Elementen, deren Träger in $X$ enthalten ist. 
\end{thm}
\begin{proof}
	Es reicht den Fall $|X| >0$, $k>0$ zu betrachten. Sei $n:= |X|$. Wir nummerieren die Elemente in $X$ und schreiben $X$ als $\{x_1,\ldots,x_n\}$. Jede Multimenge $M \in \mbinom{X}{k}$ ist eindeutig durch die Angabe der Werte $z_i := \mu_M(x_i)$ mit $i \in \{1,\ldots,k\}$ gegeben. Daher folgt die Behauptung aus Lemma~\ref{lem:multimengen}. 
\end{proof} 

