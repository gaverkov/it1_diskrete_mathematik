\chapter{Boolesche Algebra}

\section{Boolesche Funktionen und Formeln} 

\begin{defn}
	Eine \textbf{Boolesche Variable} ist eine Variable mit den Werten aus 
	\[
		\{\false,\true\}.
	\] Wir benutzen die Standard-Identifizierung von $\false \mapsto 0, \true \mapsto 1$, die uns ermöglichst die kürzere Schreibweise mit $0$ und $1$ zu nutzen. 
\end{defn} 

\begin{defn}
	Für $n \in \N$ nennen wir $f: \{0,1\}^n \to \{0,1\}$ \textbf{boolesche Funktion} in $n$ Variablen. 
\end{defn} 

\begin{bem} Es gibt genau vier boolesche Funktionen in einer Variablen: 
	\begin{center}
			\begin{tabular}{c|cccc}
				$x$ & $0$ & $x$  & $\overline{x}$ & $1$ 
				\\ \hline 
				$0$ & $0$ & $0$ &  $1$ & $1$
				\\ $1$ & $0$ & $1$ & $0$ & $1$ 
			\end{tabular} 
	\end{center} 
\end{bem} 

\begin{bsp}\
	\begin{enuma} 
		\item Die \textbf{Mehrheitsfunktion}:
		\[
		f (x_1,\ldots,x_n) := \begin{cases} 
			1 & \text{wenn mindestens $n/2$ der Variablen}
			\\ & 		\text{ $x_1,\ldots,x_n$ gleich $1$ gesetzt sind}, 
			\\ 0 & \text{sonst}. 
		\end{cases} 
		\]
		\item Die \textbf{Parity-Funktion}: 
		\[
		f (x_1,\ldots,x_n) := \begin{cases} 
			1 & \text{wenn ungerade viele Variablen}
			\\ & 		\text{ $x_1,\ldots,x_n$ gleich $1$ gesetzt sind}, 
			\\ 0 & \text{sonst}. 
		\end{cases} 
		\]
		\item Sei $V$ endliche Menge. Dann lässt sich jeder Graph $G = (V,E)$ mit der Knotenmenge $V$ als die characteristische Funktion $1_E \in \{0,1\}^{\binom{V}{2}}$ der Menge der Kanten kodieren. Verscheidene Eigenschaften von Graphen mit der Knotenmenge $V$ können dann als boolesche Funktionen auf $\{0,1\}^{\binom{V}{2}}$ beschrieben weren. Etwa 
		\[
			f(x) := \begin{cases} 1 & \text{der Graph $(V,E)$ mit $1_E=x$} 
				\\ & \text{ist zusamenhängend}, 
				\\0  & \text{sonst}.
				\end{cases} 
		\]
		\item Der Operator
		\[
			  x \, ?\, y \, :\, z = \begin{cases}
			  			y, & \text{für} \ x =1,
			  			\\ z, & \text{für} \  x=0,
			  	\end{cases} 
		\]
		der in C/C++ und Java verfügbar ist, ist im Fall $y,z \in \{0,1\}$ eine boolesche Funktion in drei Variablen. 
	\end{enuma}
\end{bsp}

\begin{defn}
	Eine \textbf{boolesche Formel} ist ein Ausddruck der aus endlich vielen booleschen Variablen, den Verknüpfungen $\neg$, $\vee$, $\wedge$ und Konstanen $0,1$ konstruiert ist. Eine boolesche Formel, die auf den Variablen  $x_1,\ldots,x_n$ $(n \in \N$) definiert ist, ergibt durch das Auswerten die boolesche Funktion $f : \{0,1\}^n \to \{0,1\}$ zu dieser Formel. 
\end{defn} 

\begin{bem}
In den booleschen Formel schreiben wir oft auch $a \cdot b$ oder $a b$ an der Stelle von $a \wedge b$ und setzten dabei die Priorität von $\cdot$ höher als die Priorität von $\vee$. 
\end{bem} 

\begin{defn}
Zwei Formeln nennen wir identisch, wenn sie dieselbe boolesche Funktion definieren. 
\end{defn} 

\begin{defn}
	Die selber boolesche Funktion kann durch verschiedene Formeln gegeben werde. Etwa, 
	\[
		x \, \overline{y}\vee y= x \vee y.  
	\]
\end{defn} 

\begin{bem}
	Wenn zwei Formeln $F$ und $G$ identisch sind, so bezeichnet man das in manchen Quellen als $F \equiv G$, wir nutzen aber hier (wie in \cite{Lov20}) die Bezeichnung $F = G$ (die wir ja auch sonst, etwa für die  Gleichheit von Funktionen benutzen). 
\end{bem} 

\begin{defn} 
Wir nennen eine boolesche Formel in $n$ Variablen \textbf{erfüllbar}, wenn sie eine eine boolesche Funktion $f  : \{0,1\}^n \to \{0,1\}$ definiert, die nicht identisch gleich $0$ ist. Das bedeutet, dass die Formel bei einer Belegung von Variablen zu $1$ ausgewertet wird. 
\end{defn} 

\begin{defn} 
Wir nennen eine boolesche Formel in $n$ Variablen eine \textbf{Tautologie}, wenn sie eine boolsche Funktion $f: \{0,1\}^n \to \{0,1\}$, die identisch gleich $1$ ist. Das bedeutet, dass die Formel bei jeder Belegung von Variablen zu $0$ ausgewertet wird. 
\end{defn} 


\begin{prop}
	Für alle $a,b,c \in \{0,1\}$ gelten die folgenden Gleichungen: 
	{ \small 
		\begin{align*}
			a \wedge b & = b \wedge a &  a \vee b & = b \vee a & & \text{Kommutativgesetze} 
			\\ (a \wedge b) \wedge c & = a \wedge (b \wedge c) &  (a \vee b) \vee c& = a \vee (b \vee c) & & \text{Distributivgesetze} 
			\\ a \wedge a & = a  & a \vee a & = a  & & \text{Idempotenzgesetze} 
			\\ a \wedge (b \vee c) & = (a \wedge b) \vee (a \wedge c) &  a \vee (b \wedge c) & = (a \vee b) \wedge (a \vee c)  && \text{Distributivgesetze} 
			\\ a \wedge 1 & = a  & a \vee 0 & = a  & & \text{Neutralitätsgesetze} 
			\\ a \wedge 0 & = 0 &  a \vee 1 & = 1 & & \text{Extremalgesetze} 
			\\ \overline{\overline{a}} & = a & & & & \text{Doppelnegationsgesetz}			
			\\ \overline{a \wedge b} & = \overline{a} \vee \overline{b}  & \overline{a \vee b} & = \overline{a} \wedge \overline{b} & & \text{De Morgansche Gesetze} 
			\\ a \wedge \overline{a} & = 0  & a \vee \overline{a} & = 1 & & \text{Komplementärgesetze}
			\\ \overline{0} & = 1 & \overline{1} & = 0 & & \text{Dualitätsgesetze} 
			\\ a \vee (a \wedge b) & = a & a \wedge (a \wedge b) & = a & & \text{Absorptionsgesetze}
		\end{align*} 
	}
\end{prop} 





\section{Darstellung durch DNF} 

\begin{defn}
	Ist $x$ eine boolesche Variable, so nennen wir die Formeln $x$ und $\overline{x}$ \textbf{Literale}. 
	
	Wir nennen Disjunktion endlich vieler Literale eine Elementardisjunktion (kurz ED) und Konjunktion endlich vieler Literale eine Elementarkonjunktion (kurz EK). 
	
	Des Weiteren nennen wir Konjunktion endlich vieler EDs eine konjunktive Normalform (kurz KNF) und Disjunktion endlich vieeler EDs eine dijsunktive Normalform. 
\end{defn} 


\begin{bem}
	Für eine boolesche Variable $x$, benutzen wir die Bezeichnung 
	\[
	x^a := \begin{cases} 
		x & \text{für} \ a=1,
		\\				\overline{x} & \text{für} \ a=0
	\end{cases} 
	\]
	mit $a \in \{0,1\}$, um die beiden Literale $x$ und $\overline{x}$ zu beschreiben. 
	Es bemerkt, dass der Wert von $x^a$ genau dann $1$ ist, wenn $x=a$ gilt. 
\end{bem} 

\begin{thm}
	Jede boolesche Funktion $ f: \{0,1\}^n \to \{0,1\}$ kann durch eine DNF beschrieben werden. 
\end{thm} 
\begin{proof}
	Sei 
	\[
	W := \setcond{ (a_1,\ldots,a_n) \in \{0,1\}^n}{f(a_1,\ldots,a_n) = 1}. 
	\]
	Es gilt: 
	\[
	f(x_1,\ldots,x_n) = \bigvee_{ (a_1,\ldots,a_n) \in W} x_1^{a_1} \cdots x_n^{a_n}. 
	\]
	Es reichte zu zeigen, dass die rechte Seite genau dann wahr ist, wenn $(x_1,\ldots,x_n) \in W$ gilt. Die rechte Seite ist genau dann wahr, wenn einer der EKs wahr ist. Die EK $x_1^{a_1} \cdots x_n^{a_n}$ ist aber genau dann wahr wenn $x_i = a_i$ für alle $i \in\{1,\ldots,n\}$ gilt. Das zeigt die Behauptung. 
\end{proof} 

\begin{defn}
	Sei $f : \{0,1\}^n \to \{0,1\}$. Dann definieren wir die duale Funktion $f^\ast : \{0,1\}^n \to \{0,1\}$ zu $f$ durch 
	\[
	f^\ast(x_1,\ldots,x_n) := \overline{ f(\overline{x_1},\ldots,\overline{x_n})}. 
	\]
\end{defn} 

\section{Dualität der booleschen Funktionen und Formeln} 

\begin{defn}
	Als duale Formel zu einer booleschen Formel $F$ definieren wir die Formel $F^\ast$ die aus $F$ entsteht, indem in der Formel die Operatioen $\wedge$ und $\vee$ sowie die Konstanten $0$ und $1$ vertauscht. Mit anderen Worten ensteht die duale Formeln nach durch die folgenden Regeln:
	\begin{itemize} 
		\item $L^\ast := L$ für jedes Literal $L$, 
		\item $0^\ast = 1$ und $1^\ast = 0$ 
		\item $(A \vee B)^\ast = A^\ast \wedge B^\ast$ und $(A \wedge B)^\ast = A^\ast \vee B^\ast$.
		\item $(\overline{A})^\ast = \overline{A}$
	\end{itemize}  
\end{defn} 

\begin{thm}
	Sei $F$ boolesche Formel. Dann ist die duale Funktion zur Funktion von $F$ gleich der Funktion zur dualen Formeln $F^\ast$. 
\end{thm}
\begin{proof} 
	Die Behauptung folgt aus der Tatsache, dass für die Operation $f \mapsto f^\ast$ für boolesche Funktionen $f,g : \{0,1\}^n \to \{0,1\}$ die folgenden Regeln gelten: 
	\begin{align*}
		0^\ast & = 1
		\\ 1^\ast & = 0
		\\ (x_i)^\ast & = x_i
		\\ (\neg{f})^\ast & = \neg{f^\ast}
		\\ (f \vee g)^\ast & = f ^\ast\wedge g^\ast
		\\ (f \wedge g)^\ast & = f ^\ast\vee g^\ast
	\end{align*} 
	Diese Regeln entsprechen genau den Regeln, die man benutzt, wenn man eine boolsche Formel dualisiert. 
\end{proof} 



\section{Darstellung durch eine DNF}


