\chapter{Boolesche Algebra}

\section{Boolesche Funktionen und Formeln} 

\begin{defn}
	Eine \textbf{boolesche Variable} ist eine Variable mit den Werten aus 
	\[
		\{\false,\true\}.
	\] 
\end{defn} 

\begin{bem}
Wir benutzen die Standard-Identifizierung $\{\false,\true\} \to \{0,1\}$ mit 
\begin{align*}
	\false & \mapsto 0 & \true & \mapsto 1,
\end{align*}
welche uns ermöglicht, die kürzere Schreibweise mit $0$ und $1$ zu nutzen. 
\end{bem}

\begin{defn}
	Für $n \in \N$ nennen wir $f: \{0,1\}^n \to \{0,1\}$ \textbf{boolesche Funktion} in $n$ Variablen. 
\end{defn} 

\begin{bem} Es gibt genau vier boolesche Funktionen in einer Variablen: 
	\begin{center}
			\begin{tabular}{c|cccc}
				$x$ & $0$ & $x$  & $\overline{x}$ & $1$ 
				\\ \hline 
				$0$ & $0$ & $0$ &  $1$ & $1$
				\\ $1$ & $0$ & $1$ & $0$ & $1$ 
			\end{tabular} 
	\end{center} 
\end{bem} 

\begin{bsp}\
	\begin{enuma} 
		\item Die \textbf{Mehrheitsfunktion}:
		\[
		f (x_1,\ldots,x_n) := \begin{cases} 
			1 & \text{wenn mindestens $n/2$ der Variablen}
			\\ & 		\text{ $x_1,\ldots,x_n$ gleich $1$ gesetzt sind}, 
			\\ 0 & \text{sonst}. 
		\end{cases} 
		\]
		\item Die \textbf{Parity-Funktion}: 
		\[
		f (x_1,\ldots,x_n) := \begin{cases} 
			1 & \text{wenn ungerade viele Variablen}
			\\ & 		\text{ $x_1,\ldots,x_n$ gleich $1$ gesetzt sind}, 
			\\ 0 & \text{sonst}. 
		\end{cases} 
		\]
		\item Sei $V$ endliche Menge. Dann lässt sich jeder Graph $G = (V,E)$ mit der Knotenmenge $V$ als die characteristische Funktion $1_E \in \{0,1\}^{\binom{V}{2}}$ der Menge der Kanten kodieren. Verscheidene Eigenschaften von Graphen mit der Knotenmenge $V$ können dann als boolesche Funktionen auf $\{0,1\}^{\binom{V}{2}}$ beschrieben weren. Etwa 
		\[
			f(x) := \begin{cases} 1 & \text{der Graph $(V,E)$ mit $1_E=x$} 
				\\ & \text{ist zusamenhängend}, 
				\\0  & \text{sonst}.
				\end{cases} 
		\]
		\item Der Operator
		\[
			  x \, ?\, y \, :\, z = \begin{cases}
			  			y, & \text{für} \ x =1,
			  			\\ z, & \text{für} \  x=0,
			  	\end{cases} 
		\]
		der in C/C++ und Java verfügbar ist, ist im Fall $y,z \in \{0,1\}$ eine boolesche Funktion in drei Variablen. 
	\end{enuma}
\end{bsp}

\begin{defn}
	Eine \textbf{boolesche Formel} ist ein Ausddruck der aus endlich vielen booleschen Variablen, den Verknüpfungen $\neg$, $\vee$, $\wedge$ und Konstanen $0,1$ konstruiert ist. Eine boolesche Formel, die auf den Variablen  $x_1,\ldots,x_n$ $(n \in \N$) definiert ist, ergibt durch das Auswerten die boolesche Funktion $f : \{0,1\}^n \to \{0,1\}$ zu dieser Formel. 
\end{defn} 

\begin{bem}
In den booleschen Formel schreiben wir oft auch $a \cdot b$ oder $a b$ an der Stelle von $a \wedge b$ und setzten dabei die Priorität von $\cdot$ höher als die Priorität von $\vee$. 
\end{bem} 

\begin{defn}
Zwei Formeln nennen wir identisch, wenn sie dieselbe boolesche Funktion definieren. 
\end{defn} 

\begin{defn}
	Die selber boolesche Funktion kann durch verschiedene Formeln gegeben werde. Etwa, 
	\[
		x \, \overline{y}\vee y= x \vee y.  
	\]
\end{defn} 

\begin{bem}
	Wenn zwei Formeln $F$ und $G$ identisch sind, so bezeichnet man das in manchen Quellen als $F \equiv G$, wir nutzen aber hier (wie in \cite{Lov20}) die Bezeichnung $F = G$ (die wir ja auch sonst, etwa für die  Gleichheit von Funktionen benutzen). 
\end{bem} 

\begin{defn} 
Wir nennen eine boolesche Formel in $n$ Variablen \textbf{erfüllbar}, wenn sie eine eine boolesche Funktion $f  : \{0,1\}^n \to \{0,1\}$ definiert, die nicht identisch gleich $0$ ist. Das bedeutet, dass die Formel bei einer Belegung von Variablen zu $1$ ausgewertet wird. 
\end{defn} 

\begin{defn} 
Wir nennen eine boolesche Formel in $n$ Variablen eine \textbf{Tautologie}, wenn sie eine boolsche Funktion $f: \{0,1\}^n \to \{0,1\}$, die identisch gleich $1$ ist. Das bedeutet, dass die Formel bei jeder Belegung von Variablen zu $0$ ausgewertet wird. 
\end{defn} 


\begin{prop}
	Für alle $a,b,c \in \{0,1\}$ gelten die folgenden Gleichungen: 
	\footnotesize
		\begin{align*}
			a \wedge b & = b \wedge a &  a \vee b & = b \vee a & & \text{Kommutativgesetze} 
			\\ (a \wedge b) \wedge c & = a \wedge (b \wedge c) &  (a \vee b) \vee c& = a \vee (b \vee c) & & \text{Distributivgesetze} 
			\\ a \wedge a & = a  & a \vee a & = a  & & \text{Idempotenzgesetze} 
			\\ a \wedge (b \vee c) & = (a \wedge b) \vee (a \wedge c) &  a \vee (b \wedge c) & = (a \vee b) \wedge (a \vee c)  && \text{Distributivgesetze} 
			\\ a \wedge 1 & = a  & a \vee 0 & = a  & & \text{Neutralitätsgesetze} 
			\\ a \wedge 0 & = 0 &  a \vee 1 & = 1 & & \text{Extremalgesetze} 
			\\ \overline{\overline{a}} & = a & & & & \text{Doppelnegationsgesetz}			
			\\ \overline{a \wedge b} & = \overline{a} \vee \overline{b}  & \overline{a \vee b} & = \overline{a} \wedge \overline{b} & & \text{De Morgansche Gesetze} 
			\\ a \wedge \overline{a} & = 0  & a \vee \overline{a} & = 1 & & \text{Komplementärgesetze}
			\\ \overline{0} & = 1 & \overline{1} & = 0 & & \text{Dualitätsgesetze} 
			\\ a \vee (a \wedge b) & = a & a \wedge (a \wedge b) & = a & & \text{Absorptionsgesetze}
		\end{align*} 
\end{prop} 





\section{Darstellung durch DNF} 

\begin{defn}
	Ist $x$ eine boolesche Variable, so nennen wir die Formeln $x$ und $\overline{x}$ \textbf{Literale}. 
	
	Wir nennen Disjunktion endlich vieler Literale eine Elementardisjunktion (kurz ED) und Konjunktion endlich vieler Literale eine Elementarkonjunktion (kurz EK). 
	
	Des Weiteren nennen wir Konjunktion endlich vieler EDs eine konjunktive Normalform (kurz KNF) und Disjunktion endlich vieeler EDs eine dijsunktive Normalform. 
\end{defn} 


\begin{lem}
	Für alle $x, a \in \{0,1\}$ gilt
	\begin{equation} \label{equiv:x:no:x}
	x \leftrightarrow a = \begin{cases} 
		x & \text{für} \ a=1,
		\\				\overline{x} & \text{für} \ a=0
	\end{cases} 
	\end{equation}
	und 
	\begin{equation} \label{neg:equiv} 
		\overline{x \leftrightarrow a} = x \leftrightarrow \overline{a}. 
	\end{equation}
\end{lem} 
\begin{proof} 
	Man kann alle vier Möglichkeiten für $(x,a) \in \{0,1\}^2$ durchprobieren. 
\end{proof} 

\begin{thm}
	Jede boolesche Funktion $ f: \{0,1\}^n \to \{0,1\}$ kann durch eine DNF beschrieben werden. 
\end{thm} 
\begin{proof}
	Wir betrachten die Belegungen von $(x_1,\ldots,x_n)$, auf denen $f$ zu $1$ ausgewertet wird: 
	sei 
	\[
	W := \setcond{ (a_1,\ldots,a_n) \in \{0,1\}^n}{f(a_1,\ldots,a_n) = 1}. 
	\]
	Es gilt:
	\begin{equation}
		\label{dnf}
		f(x_1,\ldots,x_n) = \bigvee_{ (a_1,\ldots,a_n) \in W} \ \, \bigwedge_{i=1}^n \, (x_i \leftrightarrow a_i). 
	\end{equation} 
	Um das zu sehen argumentieren wir wie folgt. Für alle $x_1,\ldots,x_n \in \{0,1\}$ gilt:
	\begin{align*}
		& & 	& f(x_1,\ldots,x_n) \ \text{ist wahr}  
	\\	& \Leftrightarrow &  & (x_1,\ldots,x_n) \in W
	\\	& \Leftrightarrow & & \text{es gibt ein $(a_1,\ldots,a_n) \in W$ mit $x_i=a_i$ für alle $i=1,\ldots,n$} 
	\\  & \Leftrightarrow & & \text{es gibt ein $(a_1,\ldots,a_n) \in W$, für welches $\bigwedge_{i=1}^n (x_i \leftrightarrow a_i)$ wahr ist}
	\\  & \Leftrightarrow & & \bigvee_{ (a_1,\ldots,a_n) \in W} \ \, \bigwedge_{i=1}^n \, (x_i \leftrightarrow a_i) \ \text{ist wahr}. 
	\end{align*} 

	Wegen \eqref{equiv:x:no:x} ist der Ausdruck $x_i \leftrightarrow a_i$ in \eqref{dnf} je nach dem Wert von $a_i$ entweder $x_i$ oder $\overline{x_i}$. Mit dieser Interpretation ist die rechte Seite von \eqref{dnf} eine DNF.
\end{proof} 

\begin{bem} \label{dnf:rezept}
	\eqref{dnf} ist ein Rezept der Konstruktion einer Darstellung als DNF für ein gegebenes $f$. Für jede Belegung $(a_1,\ldots,a_n)$, bei der $f$ wahr wird, füge die Elementarkonjunktion hinzu deren Literale für jedes $i=1,\ldots,n$ folgendermaßen  gewählt werden: $x_i$ bei $a_i=1$ und $\overline{x_i}$ bei $a_i=0$. 
\end{bem} 


\section{Dualität der booleschen Funktionen und Formeln} 

\begin{defn}
	Sei $f : \{0,1\}^n \to \{0,1\}$. Dann definieren wir die duale Funktion $f^\ast : \{0,1\}^n \to \{0,1\}$ zu $f$ durch 
	\[
	f^\ast(x_1,\ldots,x_n) := \overline{ f(\overline{x_1},\ldots,\overline{x_n})}. 
	\]
\end{defn} 


\begin{defn}
	Als duale Formel zu einer booleschen Formel $F$ definieren wir die Formel $F^\ast$ die aus $F$ entsteht, indem in der Formel die Operatioen $\wedge$ und $\vee$ sowie die Konstanten $0$ und $1$ vertauscht. Mit anderen Worten ensteht die duale Formeln nach durch die folgenden Regeln:
	\begin{itemize} 
		\item $L^\ast := L$ für jedes Literal $L$, 
		\item $0^\ast = 1$ und $1^\ast = 0$ 
		\item $(A \vee B)^\ast = A^\ast \wedge B^\ast$ und $(A \wedge B)^\ast = A^\ast \vee B^\ast$.
		\item $(\overline{A})^\ast = \overline{A}$
	\end{itemize}  
\end{defn} 

\begin{thm}
	Sei $F$ boolesche Formel. Dann ist die duale Funktion zur Funktion von $F$ gleich der Funktion zur dualen Formel $F^\ast$. 
\end{thm}
\begin{proof} 
	Die Behauptung folgt aus der Tatsache, dass für die Operation $f \mapsto f^\ast$ für boolesche Funktionen $f,g : \{0,1\}^n \to \{0,1\}$ die folgenden Regeln gelten: 
	\begin{align*}
		0^\ast & = 1
		\\ 1^\ast & = 0
		\\ (x_i)^\ast & = x_i
		\\ (\neg{f})^\ast & = \neg{f^\ast}
		\\ (f \vee g)^\ast & = f ^\ast\wedge g^\ast
		\\ (f \wedge g)^\ast & = f ^\ast\vee g^\ast
	\end{align*} 
	Diese Regeln entsprechen genau den Regeln, die man benutzt, wenn man eine boolsche Formel dualisiert. 
\end{proof} 



\section{Darstellung durch KNF}

\begin{thm}
	Jede boolesche Funktion $f : \{0,1\}^n \to \{0,1\}$ kann durch eine KNF gegeben werden. 
\end{thm} 
\begin{proof}
		Wir starten mit der Darstellung von $f^\ast$ als 
		\[
			f^\ast(x_1,\ldots,x_n) = \bigvee_{f^\ast(a_1,\ldots,a_n)=1} \ \, \bigwedge_{i=1}^n \, (x_i \leftrightarrow a_i).
		\]
		Das ergibt 
		\[
			\overline{ f(\overline{x_1},\ldots,\overline{x_n})} = \bigvee_{f(\overline{a_1},\ldots,\overline{a_n}) = 0} \ \, \bigwedge_{i=1}^n  \, (x_i \leftrightarrow a_i). 
		\]
		Nun negieren wir die linke und rechte Seite. Die beiden De Morgansche Gesetze ergeben die Darstellung 
		\[	
				f(\overline{x_1},\ldots,\overline{x_n}) = \bigwedge_{f(\overline{a_1},\ldots,\overline{a_n})=0} \ \, \bigwedge_{i=1}^n  \, (\overline{x_i} \leftrightarrow a_i). 
		\]
		Substitution von $x_i$ durch $\overline{x_i}$ ergibt 
		\[
				f(x_1,\ldots,x_n) = \bigwedge_{f(\overline{a_1},\ldots,\overline{a_n})=0} \ \, \bigwedge_{i=1}^n  \, (x_i \leftrightarrow a_i)
		\]
		Substitution von $a_i$ durch $\overline{b_i}$ ergibt
		\begin{equation} \label{knf} 
			f(x_1,\ldots,x_n) = \bigwedge_{f(b_1,\ldots,b_n)=0} \ \, \bigwedge_{i=1}^n  \, (x_i \leftrightarrow \overline{b_i})
		\end{equation}
		Wegen \eqref{equiv:x:no:x} kann $x_i \leftrightarrow \overline{b_i}$ je nach dem Wert von $b_i$ entweder durch $x_i$ oder durch $\overline{x_i}$ ersetzt werden. Mit dieser Interpretation ist die rechte Seite von \eqref {knf} eine KNF. 
\end{proof} 

\begin{bem}
	\eqref{knf} ist ein Rezept zur Konstruktion einer Darstellung als KNF für ein gegebenes $f$. Für jede Belegung $(b_1,\ldots,b_n)$, bei der $f$ falsch wird, füge die folgende Elementardijsunktion hinzu, deren Literarale für jedes $i=1,\ldots,n$, folgendermaßen fixiert werden: $x_i$ bei $b_i=0$ und $\overline{x_i}$ bei $b_i=1$. Dieses Rezept ist die ``Dualisierung'' des Rezepts \eqref{dnf:rezept} zur Konstruktion, das dadurch entsteht, dass man die Rollen von $0$ und $1$ vertauscht. 
\end{bem} 


\section{Test der Erfüllbarkeit von KNF mit jeweils höchstens zwei Literalen pro ED}


