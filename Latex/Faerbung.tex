\section{Schwere Rechenprobleme für Graphen} 

\subsection{Färbung} 

\begin{defn}
	Sei $G = (V,E)$ ein Graph und $C$ eine $k$-elementige Menge.
	Dann heißt eine Abbildung $ f : V\to C$ eine \textbf{$k$-Färbung von $G$}, wenn $f(u) \ne f(v)$ für alle $\{u,v\} \in E$ gilt. 
	
	Das minimale $k$, für das $G$ eine $k$-Färbung besitzt, nennt man die \textbf{chromatische Zahl} von~$G$ und bezeichnet diesen Wert als $\chi(G)$. 
\end{defn} 

\begin{prop}[Brute-Force-Färbung] \label{brute-force-faerbung}
	Es gibt einen Algorithmus, der für einen (als Adjazenz- oder Kantenliste) gegebenen Graphen $G=(V,E)$ und ein $k \in \N$ in der Zeit $O(|V| \cdot |E| \cdot k^{|V|})$ entscheidet, ob $G$ eine $k$-Färbung besitzt. 
\end{prop} 
\begin{proof} 
	Die Fälle $k=1$ oder $E = \emptyset$ sind direkt einsichtig. 
	Seien also $k \ge 2$ und $|E| \geq 1$.
	Es gibt $k^{|V|}$ Abbildungen $V \to \{0,\ldots,k-1\}$, die man alle algorithmisch aufzählen kann. Zur Aufzählung kann man diese Abbildungen als Darstellungen der Zahlen aus $\{0,\ldots,k^{|V|} -1\}$ im Stellenwertsystem zur Basis $k$ auffassen und all diese Darstellungen von der Darstellung von $0$ beginnend durch das sukzessive Inkrementieren aufzählen. Eine Inkrementierung benötigt $O(|V|)$ Elementaroperationen, sodass man beim Aufzählen auf insgesamt $O(k^{|V|} |V|)$ Elementaroperationen kommt. Für jede Abbildung kann durch das Iterieren über alle Kanten überprüft werden, ob die Abbildung eine $k$-Färbung ist. 
\end{proof} 


\begin{bem}
	Der Algorithmus aus Proposition~\ref{brute-force-faerbung} ist kein Polynomialzeit-Algorithmus: denn im Fall $k \ge 2$ hat die Laufzeit des Algorithmus die Ordnung mindestens $
k^{|V|} \ge  2^{|V|}$. Das heißt, es ist kein effizienter Algorithmus. 
	
	Im Spezialfall $k=2$, gibt es aber einen effizienten Algorithmus.
\end{bem} 

\begin{prop}
		Es gibt einen Algorithmus der für einen als Adjazenzliste gegebenen Graphen $G=(V,E)$ in der Zeit $\Theta(|V|+|E|)$ entscheidet, ob $G$ eine $2$-Färbung besitzt. 
\end{prop} 
\begin{proof}
	Graphen mit $2$-Färbung sind genau die bipartiten Graphen. Ob ein Graph bipartit ist, kann mit Hilfe einer Tiefen- oder Breitensuche entschieden werden.
	
	\underline{Mit einer Tiefensuche:} Zunächst können wir annehmen, dass $G$ zusammenhängend ist. Wenn er mehrere Zusammenhangskomponenten hat, dann führen wir die folgende Vorgehensweise auf jeder seiner Zusammenhangskomponenten einzeln aus (Bestimmen der Komponenten geht mittels Breitensuche).
	
	Nennen wir die beiden Teile der eventuellen Bipartition rot und blau.
	\begin{itemize}
	 \item 	Einen Knoten, aus dem die Suche gestartet wird, färben wir rot oder blau (die Wahl der Farbe für den Startknoten ist nicht wichtig).
	 \item Bei einem Knoten, der bereits gefärbt ist, müssen alle Nachbarn mit der jeweils anderen Farbe gefärbt werden: die noch nicht gefärbten Nachbarn von einem roten Knoten färbt man blau, die nicht gefärbten Nachbarn von einem blauen Knoten färbt man rot.
	 \item Das Färben der Nachbarn erfolgt in der Tiefensuche beim Sondieren der Kanten.
	 \item Wenn man beim Sondieren der Nachbarn eines Knotens u einen Knoten entdeckt, der mit der selben Farbe wie u gefärbt ist, ist der Graph nicht bipartit, da wir dann einen Kreis in~$G$ gefunden haben, der ungerade viele Kanten enthält.
	\end{itemize}
\end{proof} 

\begin{bem}
	Es ist kein effizienter Test der $3$-Färbbarkeit bekannt. Man vermutet, dass es keinen solchen Test gibt. 
	In der Theorie und den Anwendungen gibt es Tausende von Rechenproblemen, die zur Entscheidung der $3$-Färbbarkeit \emph{äquivalent} sind, wobei man in diesem Zussammenhang die Äquivalenz in einem bestimmten genau definierbaren Sinn einführt. Das bedeutet: Die $3$-Färbbarkeit gehört zu den sogenannten $\mathrm{NP}$-vollständigen Problemen. 
\end{bem} 


\condclearpage
\subsection{Existenz von Hamilton-Kreisen und das TSP}

\begin{defn}
Sei $G = (V,E)$ ein Graph oder $D = (V,A)$ ein Digraph.
\begin{itemize}
 \item Ein \textbf{Hamilton-Kreis} in $G$ bzw.~$D$ ist ein Zyklus im (Di)Graphen, der jeden Knoten in~$V$ genau einmal enthält.
 \item Ein \textbf{Euler-Kreis} in~$G$ bzw.~$D$ ist ein Zyklus im (Di)Graphen, der jede Kante in~$E$ bzw.~$A$ genau einmal enthält.
\end{itemize}
\end{defn}

\begin{lem}
\label{lem:existenz-euler-kreis}
Ein stark zusammenhängender Digraph $D$ bzw.~ein zusammenhängender Graph~$G$ enthält genau dann einen Euler-Kreis, wenn für jeden Knoten in~$D$ der Eingangsgrad gleich dem Ausgansgrad ist bzw.~jeder Knoten in~$G$ geraden Grad hat.

Existiert ein Euler-Kreis, so kann er in der Zeit $O(|A|)$ bzw.~$O(|E|)$ berechnet werden.
\end{lem}
\begin{proof}
Wir betrachten einen ungerichteten Graphen~$G$ (die Behandlung für Digraphen ist analog).
Sei zunächst $K$ ein Euler-Kreis in $G$. Da $K$ alle Kanten von $G$ genau einmal enthält, ist der Grad eines jeden Knotens in $G$ gleich dem Grad eines jeden Knotens in $K$.
In einem Kreis gibt es aber nur gerade Knotengrade.

Umgekehrt, sei nun $G$ ein Graph mit ausschlie\ss lich geraden Knotengraden.
Wir verfolgen einen iterativen Ansatz um einen Euler-Kreis schrittweise aus Kreisen in $G$ zusammenzusetzen:
\begin{itemize}
 \item Beginne an einem beliebigen Knoten $v$ in $G$ und verfolge solange Kanten, bis man wieder zu $v$ zurückkehrt (dies ist möglich durch die geraden Knotengrade in jedem Knoten).
 \item Dies erzeugt einen Kreis $K_1$ in $G$, dessen Kanten wir aus $G$ entfernen und dann den Graphen $G_1$ erhalten.
 \item Da alle Knotengrade in $K_1$ gleich $2$ sind, hat $G_1$ wieder nur gerade Knotengrade.
 \item Wie oben erzeugen wir einen Kreis $K_2$ in $G_1$ und setzen ihn mit $K_1$ entlang eines gemeinsamen Knotens zum Zyklus $Z$ zusammen.
 \item Iteratives Vorgehen erzeugt auf diese Weise Graphen $G_1,G_2,\ldots,G_k$, wobei der letzte Graph $G_k$ keine Kanten mehr enthält und der letzte erzeugte Kreis $K_{k+1}$ den bis dahin zusammengebauten Zyklus $Z$ zu einem Euler-Kreis vervollständigt.
\end{itemize}
%
Dieser algorithmische Ansatz kann in linearer Zeit $O(|E|)$ umgesetzt werden (Algorithmus von Hierholzer).
\end{proof}

\begin{bem} 
Das \textbf{Rundreiseproblem} oder \textbf{Problem der Handlungsreisenden} (engl.~\emph{traveling saleperson problem (TSP)}) hat zur Aufgabe, einen kürzesten Hamilton-Kreis zu finden.
Genauer:
\begin{center}
\textit{Gegeben ein zusammenhängender gewichteter Graph~$G=(V,E,w)$,\\ finde einen Hamilton-Kreis $H$ in $G$ minimalen Gewichts.}
\end{center}
Das Bestimmen eines minimalen Hamilton-Kreises findet Anwendung zum Beispiel:
\begin{itemize}
 \item in der DNA-Sequenzierung,
 \item beim Design integrierter Schaltkreise,
 \item bei der Maschinensteuerung bei Herstellung von Leiterplatten,
 \item bei Tourenplanung in Verkehrsnetzen.
\end{itemize}
\end{bem}

\begin{bem}
Es ist kein polynomieller Algorithmus bekannt, der das Rundreiseproblem löst.
Weiterhin wird vermutet, dass es keinen solchen Algorithmus geben kann.

Da man die Aufgaben in der Praxis trotzdem lösen möchte, braucht man zum Beispiel sogenannte \textbf{Approximationsverfahren}, die eventuell nicht einen minimalen Hamilton-Kreis bestimmen, aber einen, der \glqq fast\grqq\ minimal ist.
\end{bem} 

\begin{defn}
Sei $G=(V,E,w)$ ein gewichteter Graph. Das Rundreiseproblem bzw.~die Gewichtsfunktion~$w$ auf $G$ hei\ss t \textbf{euklidisch}, falls $w$ die sogenannte \textbf{Dreiecksungleichung} erfüllt.
Das hei\ss t, falls für je drei Knoten $i,j,k \in V$, die mit den Kanten $ij, ik, kj \in E$ ein Dreieck in $G$ bilden, die Ungleichung
\[
w(i,j) \leq w(i,k) + w(k,j)
\]
gilt.
\end{defn}

\begin{bem}[Spannbaum-Heuristik für das TSP]\,\\
\underline{Idee einer Approximationslösung für das Rundreiseproblem:} Gegeben sei $G=(V,E,w)$.
\begin{itemize}
 \item Bestimme einen minimalen Spannbaum $T$ in $G$.
 \item Erzeuge den Digraphen $T'$ aus~$T$, der für jede Kante $e \in T$ ein Paar von antiparallelen Kanten $e',e''$ enthält mit Gewichten $w(e')=w(e'')=w(e)$.
 \item Bestimme einen Euler-Kreis $K$ in $T'$ (zum Beispiel \glqq au\ss en herum\grqq).
 \item Erzeuge einen Hamilton-Kreis aus $K$ indem man entlang $K$ läuft und bereits besuchte Kanten überspringt und den Weg dabei \glqq abkürzt\grqq.
\end{itemize}
Die genaue Ausführung kann durch eine Nachfolger-Funktion, die der Kreis $K$ definiert, umgesetzt werden:
Angenommen, die Knoten in $G$ seien $V = \{1,2,\ldots,n\}$. Dann ist $\cc{succ}_K[i]$ der Nachfolgeknoten von $i$ in~$K$.
\end{bem}


\begin{algorithm}[H]
\caption{$\cc{Rundreise-MSB}(G)$}
\begin{algorithmic}[1]
 \STATE $T :=$ minimaler Spannbaum von $G$
 \STATE $T' :=$ Digraph zu $T$ mit antiparallelen Kanten gleichen Gewichts
 \STATE $K :=$ Euler-Kreis in $T'$
 \STATE $\cc{succ}_K :=$ Array von Nachfolgern der Knoten entlang von $K$
 \STATE $p := 1$, \ $q := \cc{succ}_K[1]$, \ $R := \{pq\}$, \ markiere Knoten $1$
 \WHILE{$q \neq 1$}
  \STATE $p := q$, \ markiere Knoten $p$
  \STATE \COMMENT{Abk\"urzen falls n\"otig}
  \WHILE{($q$ ist markiert \AND $q \neq 1$)}
   \STATE $q := \cc{succ}_K[q]$
  \ENDWHILE
  \STATE $R := R \cup \{pq\}$
 \ENDWHILE
 \RETURN $R$
\end{algorithmic}
\end{algorithm}


\begin{thm}
Sei $G=(V,E,w)$ ein gewichteter Graph mit euklidischer nicht-negativer Gewichtsfunktion, also $w(i,j) \geq 0$ für alle $i,j \in V$.
Sei $d^\ast$ das Gewicht eines minimalen Hamilton-Kreises in~$G$ und~$d$ das Gewicht des durch $\cc{Rundreise-MSB}(G)$ berechneten Hamilton-Kreises.
Dann gilt
\[
d \leq 2 \cdot d^\ast.
\]
Darüberhinaus hat der Algorithmus die Laufzeit $O(|E| \log |V|)$.
\end{thm}
\begin{proof}
Beginnen wir mit der Laufzeitanalyse: Mit dem Prim-Algorithmus finden wir einen minimalen Spannbaum $T$ in $G$ in Zeit $O(|E| \log |V|)$. Das Aufstellen von $T'$ erfolgt in Zeit $O(|V|)$, da $T$ genau $|V|-1$ Kanten enthält. Der Euler-Kreis $K$ in $T'$ kann in linearer Zeit $O(|V|)$ gefunden werden, und dieser wird dann in ebenfalls linearer Zeit traversiert um einen Hamilton-Kreis zu extrahieren.
Insgesamt erhalten wir somit die angegebene Laufzeit von
\[
O(|E| \log |V|) + O(|V|) + O(|V|) + O(|V|) = O(|E| \log |V|).
\]
Beweisen wir nun die behauptete Approximationsgüte des durch $\cc{Rundreise-MSB}(G)$ berechneten Hamilton-Kreises.
Sei dazu $T$ ein minimaler Spannbaum in $G$ und $K$ der berechnete Euler-Kreis in $T'$.
Nach Definition von $T'$ gilt dann
\[
w(K) = \sum_{e \in T} \left( w(e') + w(e'') \right) = 2 \cdot w(T).
\]
Sei weiterhin $R$ die Rundreise / der Hamilton-Kreis aus $\cc{Rundreise-MSB}(G)$.
Da $w$ euklidisch ist, also die Dreiecksungleichung erfüllt, ist jedes \glqq Abkürzen\grqq entlang von $K$ ein echtes Abkürzen bzgl.~der Gewichtsfunktion~$w$ und es gilt $w(R) \leq w(K)$.

Sei nun $R^\ast$ eine minimale Rundreise / Hamilton-Kreis in $G$, also mit $w(R^\ast)$ minimal unter allen Rundreisen in $G$.
Die Rundreise $R^\ast$ hat genau $|V|$ Kanten und nach dem Entfernen einer beliebigen Kante aus $R^\ast$ erhält man einen Spannbaum $T^\ast$ in $G$.
Nach Annahme an $T$ gilt damit $w(T^\ast) \geq w(T)$.
Da $w(i,j) \geq 0$ gilt für alle Kanten $ij \in E$, gilt weiterhin $w(R^\ast) \geq w(T^\ast)$.
Kombinieren wir nun alle abgeleiteten Ungleichungen, so erhalten wir
\[
\frac{d}{d^\ast} = \frac{w(R)}{w(R^\ast)} \leq \frac{w(K)}{w(T^\ast)} = \frac{2 \cdot w(T)}{w(T^\ast)} \leq \frac{2 \cdot w(T)}{w(T)} = 2,
\]
wie behauptet.
\end{proof}

\condclearpage
\subsection{Modellierung als lineare ganzzahlige Optimierungsaufgaben}

IM AUFBAU