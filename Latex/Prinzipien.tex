\chapter*{Einleitung}

\section{Naturwissenschaftliche Prinzipien}

\begin{bem}\ 
	\begin{itemize}
		\item Nur begründete bzw. verifizierte Annahmen. 
		\item Überprüfung von Behauptungen durch Beobachtungen/Experimente und Argumente. 
		\begin{itemize}
			\item Schwerpunkt bei Mathe: Argumente 
			\item Experimente bei Mathe meistens billig: Beispiele, Skizzen 
		\end{itemize} 
		\item Die Autorität zählt nicht (``Aber mein Lehrer hat gesagt, dass das so richtig ist''), es zählen validierte Behauptungen. 
		\item Begriffe/Bezeichnungen werden möglichst eindeutig festgelegt (bei Mathe - extrem eindeutig),  so dass man möglichst keinen Spielraum für eine Interpretation haben soll. (``War das die wahre Liebe?'' - kann man die Liebe eindeutig definieren? Wenn nicht, dann kann man so eine Art Frage endlos diskutieren, ohne Ergebnis. ). ``Die Erde ist rund'' - was heißt genau ``die Erde''? Was heißt genau ``rund''? Begriffsklärung sehr wichtig. 
		\item Naturwissenschaften (Reale Welt und Modelle dazu) und  Strukturwissenschaften (Modelle). 
		\item Grundlagenwissenschaften: wenn man oft genug ``Warum?'' fragt, kommt man zu grundlegenden Fragen, deren Antworten einen bleibenden Wert und -- auf Dauer -- breite Anwendungsmöglichkeiten haben. Wichtige Frage: Warum? Warum gibt uns die $pq$-Formel das richtige Ergebnis? Die Antwort ist für die Mathe-Gemeinschaft interessanter als die Formel selbst. Das Interessante in der Mathematik ist genau das, was man in den Formeltafeln nicht findet. 
		\item Modelle vs. Situationen in der realen Welt: 
		\begin{itemize}
			\item Verschiedene Modelle für die selbe Situation möglich (diskrete Zeit, kontinuirliche Zeit usw.) $\rightsquigarrow$ Modelle beschreiben die reale Situation nur annähernd. Wenn wir ein Modell verstanden haben, heißt es noch nicht, dass wir die reale Situation dazu komplett verstanden haben. Das ergibt eine natürliche Unterteilung (Expert:innen mit Schwerpunkt bei Modellen, Expert:innen mit Schwerpunkt bei der realen Welt). 
			%
			\item Gleiche Modelle für verschiedene Situationen möglich (Wachstum von Bakterien, Wachstum in der Wirtschaft usw.) $\rightsquigarrow$ Wir können gleiche Modelle in sehr unterschiedlichen Kontexten einsetzen. Daher ist die Studie der Modelle an sich (nicht gebunden an die konkrete reale Situation) oft sinnvoll. In Mathematik/Informatik macht man genau das. 
		\end{itemize} 
	\end{itemize} 
\end{bem}


\section{Mathematische Prinzipien} 

\begin{bem}\ 
	\begin{itemize}
		\item Für eine mathematische Theorie legt man Grundbegriffe und Grundbezeichnungen (eindeutig) fest. 
		\item Auf der Basis der bereits vorhandenen Begriffen und Bezeichnungen führt man immer neue Begriffe und Bezeichnungen (eindeutig) ein. 
		\item Mit Hilfe von vorhandenen Begriffen und Bezeichnungen werden Aussagen formuliert, die dann durch Argumentation bestätigt ($=$ bewiesen) oder widerlegt werden. 
		\item Wahre mathematische Beweise sind widerspruchsfrei und vollständig. 
		\item In der Theorie interessiert man sich vor allem für noch offene Aussagen, die aktuell weder bestätigt noch widerlegt worden sind. 
		\item Mathematik ist ungleich Rechnen. Rechnen, ohne dass man versteht, was die Rechenschritte bedeuten, ist keine mathematische Tätigkeit. Wichtig ist der Sinn hinter den Rechenschritten. Erkennt man beim Rechnen den Sinn nicht, dann ist man ein menschlicher Computer. Dabei soll man bedenken, dass echte Computer  weniger Fehler und  etwas schneller beim Rechnen sind :) 
		\item Mathematik ist keine Ansammlung von Formeltafeln oder Aufzählung von mathematischen Aussagen. Formeltafeln sind Nebenprodukt davon, was man verstanden hat. Mathematische Aussagen sind ebenfalls ein Produkt, was in Mathematik viel mehr zählt sind die Begründungen ($=$ Beweise). 
	\end{itemize} 
\end{bem} 

\section{Voraussetzungen} 

Eine der Voraussetzung für die Teilnahme an diesem Kurs (und den Kursen IT-2 und IT-3) ist Abiturwissen in Mathematik. Hier sind einige Testaufgaben, die helfen könnten, Ihren Grad der Vorbereitung einzuschätzen. 

\begin{aufg}
		Wie viele Lösungen hat das folgende System der Bedingungen?
		\[
			\left\{
				\begin{array}{rl}
						\sin(x+y) & = 0 
						\\ \sin(x-y) & = 0
						\\ -4 \le & x \le 4
						\\ -4 \le & y \le 4
				\end{array} 
				\right.
		\]
\end{aufg} 

\begin{aufg} 
	Lösen Sie das System 
	\[
		\left\{
		\begin{array}{rl}
			e^x + e^{2y} & = 5
			\\ e^x - 3 e^{2y} & = 3
		\end{array} \right. 
	\]
\end{aufg} 

\begin{aufg}
	Die Gleichung $a x + b =0$ in einer unbekannten $x \in \R$ hat für $a \ne 0, b \in \R$ genau eine Lösung, für $a =0, b \ne 0$ keine Lösung und für $a = 0, b= 0$ unendlich viele Lösungen. 
	
	Analysieren Sie analog die Gleichung $a x^4 + b x^2 + c =0$ in einer Unbekannten $x \in \R$.Wie viele reelle Lösungen hat die Gleichung $a x^4 + b x^2 + c =0$ in Abhängigkeit von der Wahl der reellen Koeffizienten $a,b,c \in \R$? 
\end{aufg} 