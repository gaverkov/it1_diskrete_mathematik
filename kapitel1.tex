\chapter{Mathematische Grundlagen}

\section{Aussagen und logische Verknüpfungen}


\begin{defn}
Eine Aussage ist ein Satz (eine Folge von Zeichen mit mathematischer Bedeutung), die einen eindeutigen Wahrheitswert (entweder falsch oder wahr) hat. Den Wahrheitswert kodiert man oft mit Zahlen $0$ (falsch) und $1$ (wahr). 
\end{defn} 


\begin{bsp}\ 
\begin{itemize}
	\item $ 2 < 1 $ (falsch)
	\item $ 2 = 1 $ (falsch)
	\item $ 2 > 1 $ (wahr)
	\item $2$ ist eine Primzahl (wahre Aussage, Primzahl definiert). 
	\item $2$ ist eine schöne Zahl (keine Aussage, es sei denn, die Eingeschaft einer Zahl schön zu sein, wurde definiert). 
	\item Es gibt unendlich viele Primzahlen $n$, für welche $n+2$ ebenfalls eine Primzahl ist. (eine Aussage, Wahrheitswert ist noch nicht geklärt). 
	\item Die Gleichung $x^2+1 =0$ hat keine Lösungen. (an sich keine Aussage, es sei denn, ein Kontext war vorher gegeben, in dem die Rolle von $x$ geklärt wuirde.)
	\item Die Gleichung $x^2+1=0$ hat keine reellwertigen Lösungen (wahre Aussage). 
\end{itemize}
\end{bsp} 

\begin{defn}
Seien $ A $ und $ B $ Aussagen. Dann definiert man anhand von $ A $ und $ B $ die folgenden Aussagen:
\begin{itemize}
	\item $ A \wedge B $ Konjunktion (\glqq und\grqq) ist genau dann wahr, wenn $A$ und $B$ beide wahr sind. 
	\item $ A \vee B $ Disjunktion (\glqq oder\grqq) ist genau dann falsch, wenn $A$ und $B$ beide falsch sind. 
	\item $ A \Rightarrow B $ Implikation ist genau dann falsch, wenn $A$ wahr und $B$ falsch ist. 
	\item $ A \Leftrightarrow B $ Äquivalenz ist genau dann wahr, wenn die Wahrheitswerte von $A$ und $B$ gleich sind. 
	\item $ A \:\dot{\vee}\: B $ ausschließende Disjunktion ist genau dann wahr, wenn die Wahrheitswerte von $A$ und $B$ verschieden sind. 
	\item $ \neg A $ (wir auch als $ \bar{A} $ bezeichnet), Negation (Verneinung) ist genau dann wahr, wenn $A$ falsch ist. 
\end{itemize}
\end{defn}


\begin{bsp}\
	\begin{itemize}
		\item Seien $ x,y \in \R$. Dann gilt die Implikation: $x = y \Rightarrow x^2 = y^2 $ (wahr)
		\item Seien $x,y \in \R$. Dann gilt die Implikation $ x,y \in \R, x^2 = y^2 \Rightarrow x = y $ (falsch für $x=1$ und $y=-1$)	
	\end{itemize}
\end{bsp}

\begin{bem}
	Alternativbezeichnungen für $\Rightarrow$ und $\Leftrightarrow$ sind $\rightarrow$ bzw. $\leftrightarrow$. 
\end{bem} 

\begin{bem}
	Wenn man in Mathe-Argumenten eine Folge von Implikationen benutzt, so schreibt man auch oft kurz so etwas wie $A \Rightarrow B \Rightarrow C$. Damit meint man  $(A \Rightarrow B) \wedge (B \Rightarrow C)$, d.h., aus $A$ folgt $B$ und aus $B$ folgt $C$. Das Gleiche auch für $\Leftrightarrow$. 
\end{bem} 


\begin{bem}
	Die Aussagenlogik ist die Studie der logischen Vernknüpfungen von Aussagen. Dabei spielt die Natur in den Formeln verwendeten Aussagen, die man mit Symbolen bezeichnet, etwa $a,b, c, d, \ldots$, an sich keine Rolle. Alles, was zählt, ist der Wahrheitswert. Daher kann man auch $a,b,c,d \ldots$ als Variablen aus $\{0,1\}$ auffassen, ohne dass sich an der Studie was ändert. Mehr über die Aussagenlogik erfahren wir später in diesem Kurs. 
\end{bem} 

\clearpage
\section{Mengen und Mengenoperationen}


\begin{defn}[Intuitive ``Definition'' einer Menge] 
	Eine Menge $X$ ist durch die Eindeutige Angabe definiert, welche Objekte Elemente der Menge sind. Man schreibt in diesem Fall $x \in X$ dafür, dass das Objekt $x$ Element der Menge $X$ ist, und $x \not\in X$ dafür, dass $x$ kein Element der Menge $X$. 
	
	Mit anderen Worten: für die Angabe einer Menge $X$ soll für jedes Objekt $x$ geklärt sein, ob für dieses Objekt $x \in X$ oder $x\not \in X$ gilt. 
\end{defn} 

\begin{bem}[Zur korrekten Definition einer Menge]
	Unsere Definition der Menge ist etwas intuitiv und  somit streng genommen keine echte Definition); sie reicht aber vorerst für unsere Zwecke völlig aus. Die genaue Definition einer Menge ist durch das Axiomensystem von Zermelo-Fraenkel gegeben. Dieses System legt Folgendes fest: 
	\begin{itemize}
		\item die Existenz der leeren Mengen, 
		\item die Bedingung für die Gleichheit von zwei Mengen
		\item die Möglichkeit Mengenfamilien zu vereinigen, 
		\item die Existenz einer sogenannten Potenzmenge für eine beliebige Menge
		\item Fundierungsaxiom (ist etwas technisch)
		\item die Möglichkeit Mengen, durch eine Bedingung zu definieren. 
		\item Ersetzungsaxiom (ist etwas technisch)
	\end{itemize} 
	Zu den obigen Axiomen nimmt man noch zusätzlich das sogenannte Auswahlaxiom dazu. Das Axiomensystem, das auf diese Weise entsteht, wird als ZFC (Zermelo-Fraankel axioms plus Axom of Choice) abgekürzt. 
\end{bem} 


\begin{bsp}[Definition einer Menge durch endliche/unendliche Auflistung] 
Eine Weise, Mengen zu definieren, ist durch die Auflistung ihrer Elemente. Dabei stehen die geschweiften Klammern für Mengen, die drei Punkte bedeuten \glqq usw\grqq.
\begin{itemize}
	\item $ \{1,2,5,7\} $ - Menge aus den vier Elementen $1,2,5$ und $7$. 
	\item $ \{1\} $ - Menge aus einem einzigen Element $1$. 
	\item $ \{1,\{2,5\},\{6\}\} $ - Menge aus drei Elementen, von denen zwei Elementen selber Mengen sind. 
	\item $ \{1,2,3,\ldots\} $ - Menge der aller positiven ganzen Zahlen. 
\end{itemize}
\end{bsp}

\begin{defn}
Seien $ A $ und $ B $ Mengen. Dann heißt $ A $ eine \textbf{Teilmenge} von $ B $, wenn jedes Element von $ A $ auch Element von $ B $ ist;  wir verwenden in diesem Fall die Bezeichnung $A \subseteq B$ und nennen die Relation $\subseteq$ \textbf{Inklusion}. 

Wenn $A$ Teilmenge von $B$ aber $B$ keine Teilmengen von $A$ ist, so sagt man, dass $A$ eine \textbf{echte Teilmenge} von $B$ ist; wir verwenden in diesem Fall die Bezeichnung $A \varsubsetneq B$ und nennen die Relation $\varsubsetneq$ eine \textbf{echte} bzw. \textbf{strikte Inklusion}. 
\end{defn}

\begin{bem}
In einigen mathematischen Quellen bezeichnet man die Inklusion als $ \subset $ und nicht als $ \subseteq $. Es ist schwer zu sagen, welche Bezeichnung in der Mehrheit der Quellen benutzt wird. Es gibt aber auch Quellen, in denen $ \subset $ die strikte Inklusion bezeichnet. Durch die Nutzung $\subseteq$ vermeiden wir potenzielle Ambiguitäten. 
\end{bem} 

\begin{defn}
	 Mengen $ A $ und $ B $ heißen \textbf{gleich}, wenn $ A \subseteq B $ und $ B \subseteq A $ gilt. 
\end{defn} 


\begin{bem}[Definition einer Menge durch eine Bedingung]
Eine sehr verbreite Weise, Mengen zu definieren, ist durch Bedingungen, nach dem Format 
\[
 \{ AUSDRUCK : BEDINGUNG \}.
\] Der Doppelpunkt bedeutet \glqq sodass\grqq, \glqq mit der Bedingung\grqq. In manchen Quellen wird ein Strich an der Stelle des Doppelpunktes benutzt. 
\end{bem} 

\begin{aufg}\ 
\begin{itemize}
\item Bestimmen Sie, welche der Zahlen $1,\ldots, 100$ Elemente der Menge 
\[
	 \{ k^2 : k \in \N, k \: \text{ungerade} \} 
\]
sind. 
\item Wie viele Elemente hat die Menge $\setcond{x \in \R}{x^2- 5 x + 6} $? Welche Elemente sind es genau? 
\end{itemize}
\end{aufg}

\begin{defn}
Die \textbf{leere Menge} ist die Menge, die keine Elemente enthält; sie wird als $ \emptyset $ bzw. $\varnothing$ bezeichnet. 
\end{defn} 

\begin{defn}[Potenzmenge]
Sei $ X $ eine Menge. Dann ist die \textbf{Potenzmenge} von $ X $ die Menge aller Teilmengen von $ X $; für diese Menge benutzen wir die Bezeichnung $ 2^X $. Nach unserer Definition ist $2^X$ als  
\[
	 2^X := \{ A : A \subseteq X \} .
\]
gegeben. 
\end{defn} 

\begin{bem} Hier und im folgenden verwenden wir die Gleichung $:=$ mit Doppelpunkt, wenn es um neue Bezeichnungen geht, die festgelegt werden. Format: 
	\[
		\text{NEUE BEZEICHNUNG} := \text{BEDEUTUNG DER BEZEICHNUNG}
	\]
	Die Nutzung vom Doppelpunkt in diesem Fall ist kein Muss; man kann auch durch den Begleittext verdeutlichen, dass man eine neue Bezeichnung einführt. 
\end{bem} 

\begin{aufg}
	Wenn $X$ genau $n \in \N$ Elemente hat, wie viele Elemente hat $2^X$? Was wäre Ihre Begründung dazu? 
\end{aufg} 

\begin{bem}
	Eine weitere Bezeichnung fúr die Potenzmengen, die man in der Literatur benutzt, ist $\mathcal{P}(X)$. Ich persönlich finde $2^X$ einleuchtender (zumindest im Kontext der Kombinatorik, die im folgenden Kapitel diskutiert wird). 
\end{bem} 


\begin{bem}
	Zahlenbereiche, die Sie evtl. aus der Schule schon kennen:
\begin{itemize}
\item[$ \N $] $ := \{ 1,2,3,\ldots \} $ die Menge der natürlichen Zahlen. Uns fehlt dort die Null, daher... 
\item[$ \N_0 $] $ := \{ 0,1,2,\ldots \} $. Hier können wir nicht beliebig subtrahieren, daher... 
\item[$ \Z $] $ := \{ 0,1,-1,2,-2,\ldots \} $ die Menge der ganzen Zahlen. Hier können wir nicht beliebig dividieren, daher...
\item[$ \Q $] $ := \{ \frac{p}{q} : p \in \Z, q \in \N \} $ die Menge der rationalen Zahlen. In dieser Menge gibt es ``Löcher'' , die man merkt, wenn man Geometrie oder Analysis macht. Gemeint ist das Folgende: betrachtet man unendliche viele rationale Zahlen $a_1 \le a_2 \le a_3 \le \cdots $ und $ \cdots \le b_3 \le b_2 \le b_1$ mit $a_n \le b_n$ für jedes $n \in \N$, so gibt es nicht immer eine \underline{rationale} Zahl $x$ die $a_n \le x \le b_n$ für alle $n \in \N$ erfüllt. Informell: zwischen zwei Schranken $a_n$ und $b_n$, die sich mit jeder ``Iteration'' $n \in \N$ immer verbessert, wird nicht immer eine rationale Zahl eingefangen. In diesem Fall kann man von einem ``Loch'' spricht. Durch reelle Zahlen werden solche Löcher gestopft. 
\item[$ \R $] die Menge der reellen Zahlen. Format einer reellen Zahl: Vorzeichen $\pm$ endlich viele Stellen vor dem Komma, unendlich viele Nachkommastellen. Wir benutzen gerne die Computer-Formatierung mit Punkt an der Stelle von Komma (weil man in Mathe die Kommas gerne für viele andere Zwecke benutzt). Beispiele: 
\begin{itemize}
	\item[] $0{.}00000\cdots$ ist die $0$, 
	\item[] $1{.}00000 \cdots$ ist die $1$ 
	\item[] $-0{.}99999 \cdots$ ist das selbe wie $-1{.}000 \cdots$ und ist die $-1$. 
	\item[] $0{.}1\underbrace{0}_11\underbrace{00}_21 \underbrace{ 000}_3 1 \underbrace{0000}_41 \cdots $ ist eine reelle aber keine rationale Zahl. (Warum?)
\end{itemize} 
	\item[$ \C $] die Menge der komplexen Zahlen (werden wir noch einführen) 
\end{itemize}
\end{bem}

\begin{bem}
	Es gelten die strikten Inklusionen. 
$ \N \subsetneq \N_0 \subsetneq \Z \subsetneq \Q \subsetneq \R \subsetneq \C $. 
 \end{bem} 

\begin{bem}
	Manche Quellen definieren die Menge der natürlichen Zahlen als $ \set{0,1,2,\ldots} $, es ist mittlerweile sogar die ISO-Norm 80000-2. Aktuell spielen aber die ISO-Normen in den mathematischen Texten keine so große Rolle. Vgl. auch den ISO-Standard 31-11, in dem man z.B. $\subset$ für die strikte Inklusion reserviert. 
\end{bem}


\begin{bem}
	Für Zahlenbereiche $B  \in \{\Z,  \Q, \R\}$ und ein $a \in \R$ benutzen wir die folgenden Bezeichnungen:
	\begin{align*}
			B_{\ge a}&  := \setcond{x \in B}{x \ge a},  &  B_{>a} & := \setcond{x \in B}{x > a}, 
			\\ B_{\le a} & := \setcond{x \in B}{x \le a}, &  B_{<a} & := \setcond{x \in B}{x < a}. 
	\end{align*} 
	Beispiel: $\Z_{\ge 2}$. 
\end{bem} 


\begin{bem}[Intervalle]
	Seien $ a,b \in \R $ mit $ a \leq b $. Dann können Intervalle wie folgt definiert werden:
	\begin{align*}
		[a,b] &:= \{ x \in \R : a \leq x \leq b \}\\
		(a,b) &:= \{ x \in \R : a < x < b \}\\
		(a,b] &:= \{ x \in \R : a < x \leq b \}\\
		[a,b) &:= \{ x \in \R : a \leq x < b \} \\
		[a,\infty) & := \setcond{x \in \R}{x \ge a} \\
		(a,\infty) &:= \setcond{x \in \R}{x > a} \\ 
		(-\infty,a] &:= \setcond{x \in \R}{x \le a} \\
		(-\infty,a) &:= \setcond{x \in \R}{x < a}
	\end{align*}
\end{bem}


\begin{bem}
		Die Erweiterung der  Zahlenbereiche zu immer größeren Bereichen ist durch den Wunsch nach einer (gewissen) Vollständigkeit motiviert.
\end{bem} 


\begin{defn}
Seien $ A,B $ Mengen. Dann heißt:
\begin{itemize}
\item $ A \cap B := \{ x : (x \in A) \wedge (x \in B) \} $ \textbf{Durchschnitt} von $ A $ und $ B $,
\item $ A \cup B := \{ x : (x \in A) \vee (x \in B) \} $ \textbf{Vereinigung} von $ A $ und $ B $,
\item $ A \setminus B := \{ x : (x \in A) \wedge (x \notin B) \} $ \textbf{Mengendifferenz} von $ A $ und $ B $,
\item $A \operatorname{\triangle} B := (A \setminus B) \cup (B \setminus A)$, \textbf{Symmetrische Differenz} von $A$ und $B$. 
\end{itemize}
\end{defn}

\begin{bem}
	Für eine Grundmenge $X$ wird die Menge $2^X$ aller Teilmengen von $X$ zu einer sogenannten \textbf{booleschen Algebra} der Teilmengen von $X$, indem man $2^X$ mit den Verknüpfungen $A \cap B$, $A \cap B$ und der unären Verknüpfung $\overline{A} := X \setminus A$ ausstattet. 
\end{bem} 

\begin{defn}
Seinen $ A,B $ Mengen. $ A $ und $ B $ heißen genau dann \textbf{disjunkt}, wenn $ A \cap B = \emptyset $ gilt. In diesem Fall wird die Vereinigung von $A$ und $B$ eine  \textbf{disjunkte Vereinigung} genannt
und als $A \cupdot B$ bezeichnet. 
\end{defn}

\section{Abbildungen}


\begin{defn}
Seien $ X,Y $ Mengen. Eine \textbf{Abbildung} $ f $ von $ X $ nach $ Y $ ist eine Vorschrift, die jedem $ x \in X $ genau ein Element aus $ Y $ zuordnet. Dieses Element aus $ Y $, das dem $x$ zugeordnet wird, wird durch $ f(x) $ bezeichnet. Wenn $ f $ eine Abbildung von $ X $ nach $ Y $ ist, dann bezeichnet man das als $ f : X \to Y $. Die Menge $ X $ heißt der \textbf{Definitionsbereich} von $ f $, $ Y $ heißt der \textbf{Wertebereich} von $ f $. Wenn der Wertebereich $Y$ von $f$ Teilmengen von $\R$ ist, so nennen wir $f : X \to Y$ eine \textbf{Funktion}. 
\end{defn} 

\begin{bem} 
In manchen Quellen benutzt man den Begriff Funktion als ein Synonym zum Begriff Abbildung. 
\end{bem} 
% 16.10.2014

%Beispiel:
%\begin{enumeration}
%\item Kekse
%\item Nüsse
%\item Riegel
%\end{enumeration}
%
%\begin{table}[h]
%	\begin{tabular}{c|c|c}
%		$ X $ & $ Y $ & $ Y $ \\
%		\hline
%		1 & Kekse & Kekse \\
%		2 & Kekse & Nüsse \\
%		3 & Kekse & Riegel
%	\end{tabular}
%\end{table}

\begin{bsp}\ 
\begin{itemize}
	\item $ f : \R \to \R,\\ f(x) := x^2 -2x + 7 $
	\item $ f : \R \setminus \{ 1 \} \to \R,\\ f(x) := \frac{1}{x-1}$
	\item Die Vorzeichen-Funktion $ \sign: \R \to \R $ wird durch die Fallunterscheidung als 
	\[
		\sign(x) := \begin{cases} 
				-1 &  \text{für} \ x < 0,
				\\ 0 & \text{für} \ x =0,
				\\ 1 & \text{für}  \ x >0
			\end{cases} 
	\]
	definiert. 
	\item Definitions- bzw. Wertebereiche von Abbildungen müssen keine Teilmengen der Zahlenbereiche sein. 
	
	$ f : 2^{\N} \to \N, f(A) := \min(A) $. Die Eingabe dieser Funktion ist keine Zahl sondern eine Menge, man hat z.B. $ f( \{ 2,7,43 \} ) = 2 $.
	\item $ f : \N \to 2^\N, f(k) := \is{1}{k}.$ Die Rückgabe dieser Abbildung ist keine Zahl sondern eine Menge, z.B. $f(3) = \{1,2,3\}$. 
\end{itemize}
\end{bsp} 

\begin{bem}
Zwei Abbildungen $ f,g : X \to Y $ heißen gleich, falls $ f(x) = g(x) $ für alle $ x \in X $ gilt. 
\end{bem} 

\begin{bem} 
	Für Mengen $X$ und $Y$, 
	bezeichnet man als $ Y^X $  die Menge aller Abbildungen von $ X $ nach $ Y $.
\end{bem} 

\begin{aufg}
	Aus wie vielen Abbildungen besteht die Menge $\{1,2,3\}^{\{1,2\}}$? Zählen Sie diese Abbildungen auf? Was ist  mit $\{1,2\}^{\{-1,0,1\}}$?
\end{aufg} 

\begin{defn}
Seien $ X,Y,A,B $ Mengen mit $ A \subseteq X $ und $ B \subseteq Y $. Sei $ f : X \to Y $. Dann heißt $ f(A) := \{ f(x) : x \in A \} $ das Bild von $ A $ bzgl. $ f $ und $ f^{-1}(B) := \{ x \in X : f(x) \in B \} $ das Urbild von $ B $ bzgl. $ f $.
\end{defn} 

\begin{bsp}
	Sei $ f : \R \to \R $ mit $ f(x) := x^2 $ für alle $ x \in \R $.
	\begin{itemize}
		\item $ f( [1,2] ) = [1,4] $ $ \nearrow $ Abb. 1
		\item $ f^{-1}( [1,4] ) = [1,2] \cup [-2,-1] $ $ \nearrow $ Abb. 2
		\item
		$ \begin{aligned}[t]
			f^{-1}( [-7,8] ) &= \{ x \in \R : f(x) \in [-7,8] \} \\
			&= \{ x \in \R : -7 \leq f(x) \leq 8 \} \\
			&= \{ x \in \R : -7 \leq x^2 \leq 8 \} \\
			&= \{ x \in \R : x^2 \leq 8 \} \\
			&= \{ x \in \R : |x| \leq \sqrt{8} \} \\
			&= [-\sqrt{8},\sqrt{8}]
		\end{aligned} $
\end{itemize}
\end{bsp}


\begin{defn}
Seien $ X,Y $ Mengen und sei $ f : X \to Y $. Dann heißt $ f $:
\begin{itemize}
	\item injektiv, falls für alle $ x_1, x_2 \in X : x_1 \neq x_2 $ die Bedingung $ f(x_1) \neq f(x_2) $ gilt.
	\item surjektiv, falls für jedes $ y \in Y $ ein $ x \in X $ mit der Eigenschaft $ f(x) = y $ existiert.
	\item bijektiv, falls $ f $ injektiv und surjektiv ist.
\end{itemize}
\end{defn}

\begin{bsp}
	Untersuche folgende Funktionen auf Bijektivität:
	\begin{itemize}
		\item $ f : \R \to \R, f(x) := x^2 $ für alle $ x \in \R $\\
		surjektiv ? nein, $ -1 \neq f(x) $ für alle $ x \in \R $\\
		injektiv ? nein, $ f(x) = f(-x) $ für alle $ x \in \R $
		\item $ f : \R \to [0,+\infty), f(x) := x^2 $ für alle $ x \in \R $\\
		surjektiv ? ja\\
		injektiv ? nein (analog)
		\item $ f : \R \setminus \{ 0 \} \to \R, f(x) = \frac{1}{x} $ für alle $ x \in \R $\\
		surjektiv ? nein, 0 wird nicht angenommen\\
		injektiv ? ja
		\item $ f : \R \to \R, f(x) = 2x + 3 $ für alle $ x \in \R $\\
		bijektiv ? ja
	\end{itemize}
\end{bsp}


\begin{defn} 
Seien $ X,Y $ Mengen und sei $ f : X \to Y $ bijektiv. Die Abbildung, die jedem $ y \in Y $ das eindeutige $ x \in X $ mit $ f(x) = y $ zuordnet, heißt die Umkehrabbildung von $ f $ und wird durch $ f^{-1} $ bezeichnet.
\end{defn} 

\begin{aufg}
	Was ist die Umkehrfunktion von $ f : \R \to \R : f(x) := 2x + 3 $ ?
\end{aufg}


\begin{defn}
Seien $ X,Y,Z $ Mengen, $ f : X \to Y $ und $ g : Y \to Z $. Dann heißt $ g \circ f : X \to Z $ mit $ ( g \circ f )(x) := g( f(x) ) $ für alle $ x \in X $ die Komposition von $ g $ und $ f $.
\end{defn}

\begin{bsp}
	Seien $ f : \R \to \R : f(x) = 2x + 3 $ für alle $ x \in \R $ und $ g : \R \to \R : g(x) = x^2 $ für alle $ x \in \R $. Dann ist $ ( f \circ g )(x) = 2x^2 + 3 $ und $ ( g \circ f )(x) = (2x + 3)^2 $.
\end{bsp}


\begin{defn}
Sei $ X $ eine Menge. Dann heißt die Abbildung $ \id_X : X \to X $ mit $ \id_X(x) := x $ für alle $ x \in X $ die identische Abbildung auf $ X $. Man schreibt auch häufig $ \id $, wenn $ X $ nicht angegeben werden muss.
\end{defn}

\begin{bem}
	Seien $ X,Y $ Mengen und sei $ f : X \to Y $ bijektiv. Dann gilt
	\begin{itemize}
		\item $ f \circ f^{-1} = \id_Y $,
		\item $ f^{-1} \circ f = \id_X $.
	\end{itemize}
\end{bem} 

\section{Vereinigung und Durchschnitt einer indexierten Mengenfamilie}

\begin{defn}
Seien $ I,X $ Mengen und sei $ A : I \to 2^X $. Man schreibt auch in diesem Fall $ A_i $ statt $ A(i) $ für $ i \in I $, $ (A_i)_{i \in I} $ ist eine Familie (Schar) von Teilmengen von $ X $.

Für die Familie $ (A_i)_{i \in I} $ definiert man
\begin{align*}
	\text{den Durchschnitt} && \bigcap_{i \in I} A_i &:= \{ x \in X : x \in A_i \:\text{für alle}\: i \in I \},\\
	\text{und} \\ 
	\text{die Vereinigung} && \bigcup_{i \in I} A_i &:= \{ x \in X : x \in A_i \:\text{für ein}\: i \in I \}.
\end{align*}
\end{defn} 

\begin{bsp}
	Sei $ \alpha \in (0,\pi) $ und $ v_0 > 0 $ ($ \nearrow $ Abb. 3). $ K_\alpha $ ist die Flugbahn beim Auswurf eines Objekts mit der Anfangsgeschwindigkeit $ v_0 $ unter dem Winkel $ \alpha $ zu Erde.
	\begin{gather*}
		K_\alpha := \{ (x,y) \in R^2 : x = \cos(\alpha)t, y = \sin(\alpha)t - \frac{gt^2}{2}, y \geq 0, t \geq 0 \}\\
		( K_\alpha )_{\alpha \in (0,\pi)}\\
		\bigcap_{\alpha \in (0,\pi)} K_\alpha = \{ (0,0) \}\\
		\bigcup_{\alpha \in (0,\pi)} K_\alpha = \text{alle Werte unter der Parabel ($ \nearrow $ Abb. 4)}
	\end{gather*}
\end{bsp}

\section{Summen und Produkte}

\begin{defn} 
Eine Menge $ X $ heißt endlich, falls $ X = \emptyset $ oder falls eine bijektive Abbildung von $ \is{1}{n} $ nach $ X $ existiert mit $ n \in \N $. Der Wert $ n $ heißt die Anzahl der Elemente (Kardinalität) von $ X $ und wird durch $ |X| $ bezeichnet. Man setzt die Kardinalität von $ \emptyset $ gleich 0. Bei einer unendlichen Mengen $X$ setzt man $|X| = \infty$. 
\end{defn} 

\begin{bem}
$ |X| $ ist wohldefiniert, d.h. eine Menge kann nicht zwei unterschiedliche Kardinalitäten haben.
\end{bem} 

\begin{defn}
Sei $ X $ eine nichtleere endliche Menge. Dann kann $ X $ als $ X = \is{x_1}{x_n} $ dargestellt werden mit  $ x_i \neq x_j $ für alle $ i,j \in \is{1}{n} $ mit $i \ne j$. 

Für eine Abbildung $ f : X \to \R $ definiert man
\begin{align*}
	\sum\limits_{x \in X} f(x) &:= f(x_1) + \ldots + f(x_n),
\\
	\prod\limits_{x \in X} f(x) &:= f(x_1) \cdot \ldots \cdot f(x_n).
\end{align*}
Im Fall $ X = \emptyset $ definiert man für $ f : X \to \R $ und $ \sum\limits_{x \in X} f(x) = 0 $ und $ \prod\limits_{x \in X} f(x) = 1 $.

In der Summe $\sum\limits_{x \in X} f(x)$ heißt $f(x)$ der \emph{Summand} und im Produkt $\prod\limits_{x \in X} f(x)$ heißt $f(x)$ der \emph{Faktor}.  
\end{defn}

\begin{bem} 
	Die Summe und das Produkt über eine Menge $X$ sind wohldefiniert: die beiden Werte sind von der Nummerierung $x_1,\ldots,x_n$ der Elemente von $X$ unabhängig. Das liegt daran, dass $+$ und $\cdot$ beide kommutative Operationen sind. 
\end{bem} 


\begin{bem}
	$\sum\limits_{i=a}^b$ benutzt man als eine kurze Schreibweise für $\sum\limits_{i \in \{a,\ldots,b\}}$. Im entarteten Fall $a>b$ ist $\sum\limits_{i =a}^b$ eine Summe über die leere Menge. 
\end{bem} 

\begin{bsp}
	Für jedes $n \in \N$ gilt $S:=\sum_{i=1}^n i = \frac{1}{2} n (n+1).$ Die Abbildung $i \mapsto n+1 - i$ ist eine Bijektion von $\{1,\ldots,n\}$ nach $\{1,\ldots,n\}$. Daher gilt  $S = \sum_{i=1}^n (n+1 - i)$. Daraus ergibt sich 
	\[
		2 S = S +  S = \sum_{i=1}^n i + \sum_{i=1}^n (n+1 - i) = \sum_{i=1}^n ( i + n + 1 -i) = \sum_{i=1}^n (n+1) = n (n+1). 
	\]
	Das ergibt die gewünschte Darstellung $S = \frac{1}{2} n (n+1)$. 
\end{bsp} 


\section{Tupel und Kreuzprodukte}

\begin{defn}[Paare und das Kreuzprodukt] 
Für beliebige Objekte $ a,b $ kann man das \emph{(geordnete) Paar} $ (a,b) $ definieren. Das Paar $(a,b)$ besteht aus der ersten Komponente $a$ und der zweiten Komponente $b$. 
Die Gleichheit $ (a,b) = (c,d) $ von Paaren wird durch die Gleichheit $ a=c $ und $ b=d $ der jeweiligen Komponenten definiert.  Für Mengen $ A,B $ definiert man das \emph{Kreuzprodukt} (wird auch das \emph{kartesische Produkt} genannt) $ A \times B $ als die Menge
\[
A \times B := \{ (a,b) \,:\, a \in A, \ b \in B \}
\]
aller Paare bei denen die erste Komponente in $A$ und die zweite Komponente in $B$ ist. 
\end{defn} 

\begin{defn}[Tupel und Kreuzprodukt] 
	Komplett analog zu (geordneten) Paaren definiert man auch (geordnete) Tripel $(a,b,c)$, (geordnete) Quadrupel $(a,b,c)$ und noch allgemeiner (geordnete) $n$-Tupel $(x_1,\ldots,x_n)$ mit $n \in \N_0$. Dem entsprechend betrachtet man auch das Kreuzprodukt $A \times B \times C$ von drei Mengen, das Kreuzprodukt $A \times B \times C \times D$ von vier Mengen und allgemein auch das Kreuzprodukt 
	\[
		X_1 \times \ldots \times X_n := \{ (x_1,\ldots,x_n) \,:\, x_1 \in X_1, \ldots , x_n \in X_n \} 
	\]
	 von Mengen $ X_1,\ldots,X_n $.
	 
	 Das Element $ x_i $ mit $ i \in \is{1}{n} $ im $ n $-Tupel $ (x_1,\ldots,x_n) $ heißt die $ i $-te \emph{Komponente} des Tupels.
	 
Für eine Menge $ X $ führt man die Bezeichnung
\begin{equation*}
	X^n := \underbrace{X \times \ldots \times X}_{n \:\text{mal}} = \{ (x_1,\ldots,x_n) : x_1,\ldots,x_n \in X \}
\end{equation*}
ein. 
\end{defn} 


\begin{aufg} Was stellen die folgenden Kreuzprodukte geometrisch dar? Zeichnen Sie diese Mengen:
	\begin{itemize}
		\item $ [0,1] \times [0,2] $, $ \{ 0 \} \times [0,2] $  und $ \{ 0,1 \} \times \{ 0,2 \} $ 
		\item $ [0,1]^3 $, $ [0,1]^2 \times \{ 0 \} $, $ [0,1]^2 \times \{ 1 \} $  und $ \{ 0 \}^2 \times [0,1] $. 
	\end{itemize}
	\textbf{Hinweis:} Geometrische Darstellung von Teilmengen von $\R^n$ für $n \in \{2,3\}$ im kartesischen Koordinatensystem wird in den Kursen IT-2 und IT-3 nützlich sein. 
\end{aufg}


\section{Prädikate und Quantoren}


\begin{defn}
Sei $ X $ Menge. Dann heißt $ P : X \to \{ \text{falsch},\text{wahr} \} $ \emph{Prädikat} auf $ X $. 
\end{defn} 

\begin{bem}
	Informell beschrieben ist ein Prädikat eine Aussage über ein variables Element $x$ aus $X$. Dabei hängt im Allgemeinen der Wahrheitswert der Aussage von der Wahl von $x$ ab.  Sonst lässt sich ein Prädikat auf $X$ als eine Eigenschaft eines variablen Elements $x \in X$, die entweder vorhanden oder nicht vorhanden ist. Zum Beispiel: $P(n):=$``$n$ ist gerade'' als Eigenschaft einer natürlichen Zahl $n \in \N$. Diese Eigenschaft ist bei $n=123$ nicht vorhanden und bei $n=2020$ vorhanden. 
\end{bem}

\begin{bsp}
$ P : \N \to \{ \text{falsch},\text{wahr} \}$ mit $P(k) := $ \textquote{$ k(k+1) $ ist durch $3$ teilbar}.
\end{bsp} 

\begin{bsp}
$ P : \R^2 \to \{ \text{falsch},\text{wahr} \}$ mit $P(x,y) := $ \textquote{$ x^2 + y^2 \le 1 $} ist die Eigenschaft  ``der Punkt $(x,y)$ hat den Abstand höchstens $1$ zum Punkt $(0,0)$''. 
\end{bsp} 

\begin{bem} 
	Für eine gegebene Menge $X$ hat man eine natürliche Bijektion zwischen der Menge $ \{ \text{falsch},\text{wahr} \}^X$ aller Prädikate auf $X$ und der Menge $2^X$ aller Teilmengen von $X$. Jedes Prädikat $P : X \to \{ \text{falsch},\text{wahr} \}$ erzeugt die Menge $\setcond{x \in X }{P(x)}$ und jede Menge $A \subseteq X$ erzeugt das Prädikat $P (x) :=$ \textquote{$x \in X$}.
\end{bem} 


\begin{defn}
$ \forall\: x \in X : P(x) $ für ein Prädikat $ P $ auf eine Menge $ X $ steht für die Aussage \textquote{die Bedingung $ P(x) $ gilt für alle $ x \in X $.} Das Symbol $ \forall $ heißt das \emph{Allgemeinheitsquantor} (Bedeutung: für $ \forall $lle). \\[10pt]
%
$ \exists\: x \in X : P(x) $ bezeichnet die Aussage \textquote{die Bedingung $ P(x) $ gilt für ein $ x \in X $.} $ \exists $ heißt \emph{Existenzquantor} (Bedeutung: es $ \exists $xistiert).
\end{defn}

\begin{bem}
	Negierung von quantifizierten Aussagen erfolgt auf die folgende naheliegende Weise: 
	\begin{itemize}
		\item $ \overline{\forall\: x \in X : P(x)} \Leftrightarrow \exists\: x \in X : \overline{P(x)} $
		\item $ \overline{\exists\: x \in X : P(x)} \Leftrightarrow \forall\: x \in X : \overline{P(x)} $
	\end{itemize}
\end{bem}


\begin{bem}
	$ \forall $ und $ \exists $ lassen sich kombinieren. Seien $X, Y$ Mengen. Wenn man ein Prädikat $ P $ auf $ X \times Y $ hat, so kann man dafür die Aussagen wie 
	\[
		 \forall x \in X \,\exists\: y \in Y : P(x,y), 
	\]
	und
	\[\exists x \in X  \, \forall y \in Y : P(x,y) 
	\]
	usw. einführen. 
\end{bem}


\begin{bem}
	In der vorigen Bemerkung ist die Reihenfolge des Quantifizierens relevant. Sei $X$ eine Menge von  Personen und $A$ eine Menge von Adressen im Stadtteil Sandow. Die Aussage 
	\[
		\forall x \in X \, \exists a \in A \, :\, x \ \text{wohnt unter der Adresse} \  a .
	\]
	lautet, dass alle Personen aus $X$ irgendwo in Sandow wohnen. Die Aussage 
	\[
		\exists a \in A \, \forall x \in X \,:\, x \ \text{wohnt unter der Adresse} \ a.
	\]
	lautet dagegen, dass alle Personen aus $X$ unter einer und der selben Adresse in Sandow wohnen (z.B. als Wohngemeinschaft). Man sieht, die letztere Aussage ist eine stärkere Bedingung. 
\end{bem} 

\begin{bsp}
	Hier ein Beispiel einer Definition aus der Analysis, die man kompakt mit Quantoren und Prädikaten einführen kann. 
	
	Sei $ (a_n)_{n \in \N} $ Folge reeller Zahlen (mit anderen Worten: $ a : \N \to \R $) und sei $ \alpha \in \R $. Dann heißt $ \alpha $ Grenzwert von $ (a_n)_{n \in \N} $, falls das Folgende gilt:
	\begin{equation*}
		\forall\: \epsilon \in \R_{>0} \:\exists\: N \in \R \:\forall\: n \in \N: \left( (n \geq N) \Rightarrow (|a_n - \alpha| < \epsilon) \right)
	\end{equation*}
\end{bsp} 

\section{Relationen}


\begin{defn}
Seien $ X,Y $ Mengen. Dann heißt eine Teilmenge $ R $ von $ X \times Y $ eine (binäre) \emph{Relation} zwischen $ X $ und $ Y $. Bei $X = Y$, heßt $R$ eine (binäre) Relation auf $X$. 
\end{defn}

\begin{bem}
	Da man Prädikate auf $X \times Y$ mit Teilmengen von $X \times Y$ identifizieren kann, lassen sich Relationen auch als Prädikate auf $X \times Y$ auffassen. 
\end{bem} 

\begin{bem}
Wenn für $ x \in X $ und $ y \in Y $ die Bedingung $ (x,y) \in R $ gilt, so schreibt man $ x \, R \, y $. Das bedeutet: $x$ steht in der Relation $R$ zu $y$. 
\end{bem}

\begin{bsp}\ % hack to force itemize to new line
\begin{itemize}
	\item $ X = \{f_1,f_2,f_3,f_4\} $ - Menge von Fahrzeugen\\
	$ Y =  \{ \text{Ersatzrad}, \text{Radio}, \text{Navi}, \text{Automatik} \}$ - Menge von Features von Fahrzeugen. Hier 

	\begin{tabular}{l|c|c|c|c}
		& Ersatzrad & Radio & Navi & Automatik \\
		\hline
		$ f_1 $ & ja & ja & ja & ja \\
		$ f_2 $ & ja & ja & ja & nein \\
		$ f_3 $ & nein & nein & ja & ja \\
		$ f_4 $ & nein & ja & ja & nein
	\end{tabular} \\ 
	Diese Tabelle legt eine Relation auf $X \times Y$ fest. 
	\item $ \leq, <, \geq, > $ sind binäre Relationen auf $ \R$
	\item Sei $X$ Menge. Dann ist $ \subseteq $ ist eine binäre Relation auf $ 2^X $.
	\item Für $ a,b \in \N $ schreibt man $ a | b $, wenn $ b $ durch $ a $ ohne Rest teilbar ist. Dies ist eine binäre Relation auf $\N$. 
\end{itemize}
\end{bsp}


\begin{defn}
Sei $ X $ Menge und $ \sim $ eine Relation auf $ X $. Dann heißt $ \sim $ eine Äquivalenzrelation, falls:
\begin{enumerate}
	\item $ \sim $ ist \emph{reflexiv}, d.h. $ x \sim x $ für alle $ x \in X $.
	\item $ \sim $ ist \emph{symmetrisch}, d.h. $ x \sim y $ ist äquivalent zu $ y \sim x $ für alle $ x \in X $.
	\item $ \sim $ ist \emph{transitiv}, d.h. aus $ x \sim y $ und $ y \sim z $ folgt $ x \sim z $ für alle $ x,y,z \in X $.
\end{enumerate}
Für eine Äquivalenzrelation $ \sim $ auf einer Menge $ X $ und ein $ x \in X $ heißt
\begin{equation*}
	[x]_\sim := \{ y \in X : x \sim y \}
\end{equation*}
die Äquivalenzklasse von $ x $ bzgl. $ \sim $. Die Menge aller Äquivalenzklassen von $ \sim $ ist
\begin{equation*}
	X/{\sim} := \{ [x]_\sim : x \in X \}. % {\sim} supresses space between to "/"
\end{equation*}
\end{defn} 

\begin{bsp}\
\begin{itemize}
	\item Sei $ V $ endliche Menge und sei $ \binom{V}{2} := \{ \{ u,v \} : u,v \in V, u \neq v \} $. Das Paar $ (V,E) $ mit $ E \subseteq \binom{V}{2} $ heißt \emph{Graph} mit Kantenmenge $ V $ und Knotenmenge $ E $.
	
	$ G = (V,E), G = \{ 1,\ldots,6 \}, E = \{ \{ 1,2 \}, \{ 2,3 \}, \{ 3,4 \}, \{ 4,1 \}, \{ 1,3 \}, \{ 5,6 \} \} $
	
	Für $ a,b \in V $ heißt $ b $ von $ a $ aus \emph{erreichbar} (im Graphen $ G = (V,E) $), falls ein $ k \in \N_0 $ und Elemente $ u_0,\ldots,u_k \in V $ existieren mit $ u_0 = a $, $ u_k = b $ und $ \{ u_i,u_{i+1} \} \in E $ für alle $ i \in \N_0 $ mit $ i < k $.
	
	Die Erreichbarkeit ist eine Äquivalenzklasse auf $ V $.	Die Äquivalenzklassen (Zusammenhangskomponenten) für dieses Beispiel sind $ \{ 1,2,3,4 \} $ und $ \{ 5,6 \} $.
	
	\item Sei $ m \in \N $. Für $ a,b \in \Z $ sagt man, dass $ a $ kongruent zu $ b $ modulo $ m $ ist, falls $ a-b \in m\Z $, wobei $ m\Z := \{ mz : z \in Z \} $.
	
	Schreibweise: $ a \equiv b \mod{m} $.
	
	Die Kongruenz modulo $ m $ ist eine Äquivalenzrelation auf $ \Z $.
	
	\item Sei $ \sim $ Relation auf $ \Z \times \N $, definiert durch $ (a,b) \sim (c,d) $ für $ a,c \in \Z, b,d \in \N $, wenn $ ad = bc $ gilt.
	
	Diese Relation ist eine Äquivalenzrelation (Aufgabe).
	
	D.h. jede rationale Zahl ist eine Äquivalenzklasse von diesem $ \sim $.
\end{itemize}
\end{bsp}


\begin{defn}
	Eine Menge $X$ mit einer binären Relation $\succeq$ darauf heißt Poset (partiell geordnete Menge), wenn für alle $x,y,z \in X$ folgendes gilt: 
	\begin{itemize}
		\item $x \preceq x$ (Reflexivität)
		\item $x \preceq y, \ y \preceq z$ $\Rightarrow$ $x \preceq z$ (Transitivität). 
		\item $x \preceq y, \ y \preceq x$ $\Rightarrow$ $x = y$ (Antisymmetrie). 
	\end{itemize} 
	Die binäre Relation $\succeq$ heißt in diesem Fall die partielle Ordnung auf $X$. 
\end{defn} 

\begin{defn}
	Wenn für ein Poset $(X,\succeq)$ für alle $x, y \in X$, die Bedingung $x \succeq y$ oder die Bedingung $y \succeq x$ erfüllt ist, so nennt  man $(X,\succeq)$ eine total geordnete Menge und $\succeq$ eine totale Ordnung auf $X$. 
\end{defn} 

\begin{bsp} Beispiele von Posets. 
	\begin{itemize}
		\item $2^X$ mit Inklusion. 
		\item $\N$ mit Teilbarkeit. 
		\item Substring-Relation auf Strings. Ist $A$ eine endliche nichtleere Menge so heißt 
		\[
				A^\ast = \bigcup_{n \in \N_0} A^n
		\]
		die Menge der Strings über dem Alphabet $A$. Die Menge $\{0,1\}^\ast$ heißt die Menge der binären Strings. In diesem Kontext schreibt man oft $a_1 \cdots a_n$ an der Stelle von $(a_1,\ldots,a_n)$, etwa $10011 \in \{0,1\}^\ast$ und nicht $(1,0,1,1) \in \{0,1\}^\ast$. 
	\end{itemize} 
\end{bsp} 

\begin{defn}
	Für $n \in \N$ ist eine $n$-stellige Relation auf Mengen $X_1,\ldots,X_n$ eine Teilmenge $R \subseteq X_1 \times \cdots \times X_n$. 
\end{defn} 

\begin{bsp}
	Betrachten wir eine Tabelle, in welcher die Besucher:innen eines Hotels durch die Angaben \textbf{Name, Zimmer, Checkin-Datum, Checkout-Datum} geführt werden. Ist $S$ die Menge aller Strings und $D$ die Menge aller Daten, so kann mann die Tabelle als eine $4$-stellige Relation  $R \subseteq S \times S \times D \times D$ auffassen. Die Bedingung $(p,z,d_1,d_2) \in R$, dass $(p,z,d_1,d_2)$ sich in der Relation $R$ befinden,  bedeutet, dass die Person $p$ am Tag $d_1$ im Zimmer $z$ untergebracht wurde und am Tag $d_2$ das Hotel verlassen hat. 
	
	Wie man an diesem Beispiel sieht, sind die Tabellen eine Möglichkeit Relationen $R$ durch eine Aufzählung (durch die Zeilen einer Tabelle) zu beschreiben. 
\end{bsp} 


\section{Beweisansätze} 

\subsection{Direkter Beweis} 

\begin{bem}
	Eine grobe Beschreibung eines direkten Beweises ist wie folgt. Ein direkter Beweis einer Implikation $a \Rightarrow b$ ist ein Beweis der auf Implikationen basiert, die von $a$ ausgehen und zum $b$ führen.
\end{bem} 

\begin{bem}
	Man kann für $x \in \R$ die Äquivalenz: 
	\begin{align*}
			x^2 - 5 x + 6 & = 0  &  & \Longleftrightarrow & & x \in \{2,3\}
	\end{align*} 
	als zwei Implikationen ausschreiben und anschließend 
	folgendermaßen direkt verifizieren. Ist $x^2 - 5 x + 6 = (x-2) (x-3)$. Also folgt aus $(x-2) (x-3)=0$, dass $x-2=0$ oder $x-3=0$ gilt. Im ersten Fall erhält man aus $x-2=0$, dass $x=2$ ist. Im zweiten Fall erhält man aus $x-3$, dass $x=3$ ist. Folglich hat man $x \in \{2,3\}$. Umgekehrt: ist $x \in \{2,3\}$, so hat man im Fall $x=2$ die Gleichheiten $x^2 - 5 x + 6= 2^2 - 5 \cdot 2 + 6 = 4- 10 + 6 =0$ und im Fall $x=3$ die Gleichheiten $x^2 - 5 x + 6 = 3^2 - 5 \cdot 3 + 6 = 9 - 15 + 6 = 0$. Aus $x \in \{2,3\}$ folgt also $x^2 - 5 x + 6$. 
\end{bem} 

\subsection{Indirekte Beweise} 


\begin{bem}
	Ein Widerspruchsbeweis ist ein Beweis, bei dem man eine Aussage $a$, die man zeigen möchte, durch die Herleitung der Implikation $\overline{a} \Rightarrow \text{falsch}$ bestätigt bzw. 
	eine Implikation $a \Rightarrow b$, die man zeigen möchte, durch die Herleitung der Implikation $a \wedge \overline{b} \Rightarrow \text{falsch}$ bestätigt. 
	
	In den beiden Fällen erhält man aus einer Annahme einen sogenannten Widerspruch, d.h., eine falsche Aussage. Den Widerspruch erhält man oft in der Form $c \wedge \overline{c}$ für eine Aussage $c$. 
\end{bem} 

\begin{lem}
	Für $t \in \N$ seien $p_1,\ldots,p_t \in \N$ Zahlen mit $p_i \ge 2$ für alle $i \in \{1,\ldots,t\}$ und sei $n := p_1 \cdots p_t + 1$. Dann ist $n$ durch keine der Zahlen $p_1,\ldots,p_t$ teilbar. 
\end{lem} 
\begin{proof} 
	Angenommen, $n$ wäre durch ein $p_i$ mit $i=1,\ldots,t$ teilbar. Da aber das Produkt $p_1 \cdots p_t$ durch $p_i$ teilbar ist, ist $1 = n - p_1 \cdots p_t$ ebenfalls durch $p_i$ teilbar. Wir haben also gezeigt, dass $1$ durch die ganze Zahl $p_i$, mit $p_i \ge 2$, teilbar ist. Das ist ein Widerspruch, der uns die Behauptung unseres Lemmas bestätigt. 
\end{proof} 

\begin{bem}
	Eine Beweis durch Kontraposition ist der Beweis der Implikation $a \Rightarrow b$ dadurch, dass man die Implikation $\overline{b} \Rightarrow \overline{a}$ bestätigt. Ein  Beweis durch Kontraposition und der Widerspruchsbeweis sind miteinander verwandt, denn einen Beweis durch Kontraposition kann man in einen Widerspruchsbeweis konvertieren. 
\end{bem} 

\begin{bem}
	Beweise durch Kontraposition und Widerspruch nennt man \emph{indirekt.} 
\end{bem} 

\begin{lem} \label{lem:a:a^3}
	Sei $a \in \N$. Dann sind die folgenden Aussagen äquivalent: 
	\begin{enuma}
		\item $a$ ist gerade. 
		\item $a^3$ ist gerade. 
	\end{enuma} 
\end{lem} 
\begin{proof} 
	Wir zeigen (a) $\Rightarrow$ (b) direkt. Ist $a$ gerade, so hat $a$ die Form $a = 2 k$ mit $k \in \N$. Somit ist $a^3 = (2 k)^3 = 8 k^3$ ebenfalls gerade. 
	
	Die Implikation (b) $\Rightarrow$ (a) können wir durch die Kontraposition herleiten: wir zeigen also $\neg$ (a) $\Rightarrow$ $\neg$ (b). Wenn $a$ ungerade ist, so hat $a$ die Form $a = 2 k+1$ mit $k \in \N_0$. Somit ist $a^3  = (2k + 1)^3 = (2k)^3 + 3 (2k)^2 + 3 (2k) + 1 = 2 ( 4 k^3 + 6 k^2 + 6 k) + 1$ eine ungerade Zahl.  
\end{proof} 

\begin{thm}
	Die Zahl $\sqrt[3]{2}$ ist nicht rational. 
\end{thm} 
\begin{proof} 
	Angenommen, $\sqrt[3]{2}$ wäre rational. Dann hätte die Zahl die Form $\sqrt[3]{2} = \frac{a}{b}$ mit $a, b \in \N$. Darüber hinaus können wir annehmen, dass $a$ und $b$ nicht beide gerade sind, denn sonst kann man $a$ und $b$, solange sie beide gerade sind, durch $2$ teilen, wodurch sich $a$ und $b$ um Faktor zwei verkleinern. Es ist klar, dass dieser Prozess nach endlich vielen Schritten terminiert. 
	
	$\sqrt[3]{2} = \frac{a}{b}$ folgt $2 b^3  = a^3$. Es folgt also, dass $a^3$ gerade ist. Dann ist aber nach Lemma~\ref{lem:a:a^3} die Zahl $a$ gerade ist und somit die Form $a = 2 k$ mit $k \in \N$ hat. Dann ist $2 b^3 = a^3 = (2k)^3 = 8 k^3$, woraus $b^3= 4 k^3$ folgt. Die Zahl $b^3$ ist also gerade. Nach Lemma~\ref{lem:a:a^3}, die wir nun zur Zahl $b$ anwenden können, ist die Zahl $b$ ebenfalls gerade. Wir haben also gezeigt, dass $a$ und $b$ beide gerade sind. Unsere Annahme war aber, dass $a$ oder $b$ ungerade ist. Dieser Widerspruch zeigt, dass die Zahl $\sqrt[3]{2}$ nicht rational ist. 
\end{proof} 


\subsection{Vollständige Induktion} 

\begin{thm}[Vollständige Induktion, Version~1] \label{thm:ind}
	Sei $P$ ein Prädikat auf $\N$. Dann sind die folgenden Bedingungen äquivalent: 
	\begin{enuma}
			\item $P(n)$ gilt für alle $n \in \N$. 
			\item  $P(1)$ gilt und, aus $P(n)$ folgt $P(n+1)$, für alle $n \in \N$. 
	\end{enuma} 
\end{thm} 
\begin{proof} 
	Die Implikation (a) $\Rightarrow$ (b) ist klar: $P(1)$ ist erfüllt und da $P(n)$ und $P(n+1)$ beide Wahr ist die Implikation $P(n) \Rightarrow P(n+1)$ für jedes $n$ eine wahre Aussage. 
	
	Nun zeigen wir (a) $\Rightarrow$ (b) durch Kontraposition. Angenommen, (a) ist nicht erfüllt. Dann gibt ein $n \in \N$ für welches $P(n)$ falsch ist. Wir fixieren das kleinste solche $n \in \N$. Ist unser $n=1$ so, ist (b) nicht erfüllt, weil $P(1)$ nicht erfüllt ist. Ist $n>1$ so ist (b) nicht erfüllt, weil $P(n)$ falsch und $P(n-1)$ wahr ist, wodurch die Implikation $P(n-1) \Rightarrow P(n)$ nicht erfüllt ist. 
\end{proof} 

\begin{bem}
	Beim Verwenden von Theorem~\ref{thm:ind} unterteilt sich die Argumentation in die folgende Schritte. 
	\begin{itemize}
		\item Induktionsanfang (IA): man verifiziert, dass $P(1)$ gilt. 
		\item Induktionsvoraussetzung (IV): man macht die Annahme: sei $n \in \N$ und sei die Aussage $P(n)$ erfüllt. 
		\item Induktionsschritt (IS): man folgert $P(n+1)$ aus der Induktionsvoraussetzung.  
	\end{itemize} 
\end{bem} 

\begin{thm}
	Für jedes $n \in \N$ gilt 
	\[
		\sum_{i=1}^n i = \frac{1}{2} n (n+1). 
	\]
\end{thm} 
\begin{proof} 
		Das Prädikat mit dem wir uns in dieser Aussage befassen ist die Gleichung $$\sum_{i=1}^n i = \frac{1}{2} n (n+1)$$ die von einem variablen $n \in \N$ abhängig ist. 
		
		Diese Formel ist für $n=1$ erfüllt, denn $\sum_{i=1}^1 i = 1$ und $\frac{1}{2} 1 \cdot (1+1) = 1$. 
		
		Sei nun $n \in \N$ ein beliebiger Wert, für welche die Formel $\sum_{i=1}^n  i = \frac{1}{2} n (n+1)$ erfüllt ist. Wir zeigen, dass die Formel mit $n+1$ an der Stelle von $n$ ebenfalls erfüllt ist. Es gilt 
		\[
			\sum_{i=1}^{n+1} = \sum_{i=1}^n i  + (n+1),
		\]
		da wir in der Summe den Summanden zum Index $i=n+1$ abspalten können. Nach der Induktionsvoraussetzung ist $\sum_{i=1}^n i = \frac{1}{2} n (n+1)$. Somit hat man 
		\[
			\sum_{i=1}^{n+1} = \frac{1}{2} n (n+1) + (n+1) = \frac{1}{2} (n+1) (n+2). 
		\]
		Zusammenfassend: Unsere Formel gilt für $n=1$ und wenn unsere Formel für ein $n \in \N$ erfüllt ist, so ist sie auch mit $n+1$ an der Stelle von $n$ erfüllt. Aus Theorem~\ref{thm:ind} folgt, dass unsere Formel für jedes $n \in \N$ erfüllt ist. 
\end{proof} 


\begin{bsp}
	Sei $q \in \R \setminus \{1\}$ und $n \in \N_0$. Dann gilt $\sum_{i=1}^n q^i = \frac{q^{n+1} - 1}{q-1}$. Zeigen Sie diese Formel durch die Induktion über $n$. Hier ist der Induktionsanfang $n=0$. 
\end{bsp} 

\begin{bsp}
	Finden Sie eine Formel für $\sum_{i=1}^n i q^i$ mit $q \in \R \setminus \{1\}$ und $n \in \N_0$ und beweisen Sie diese Formel durch Induktion. 
\end{bsp} 

\begin{bem}
	Durch Induktion lassen sich nicht nur Gleichungen herleiten. Es gibt viele verschiedene Situationen in der Mathematik (und insbesondere diskreter Mathematik), in denen man durch die Induktion Aussagen verifizieren kann. 
\end{bem} 

\begin{thm}
	$n \le 2^n$ gilt für alle $n \in \N$. 
\end{thm} 
\begin{proof} 
	Diese Ungleichung kann man mit der Verwendung Ihrer Schulkenntnisse aus der Analysis herleiten. Der folgende Beweis durch die Induktion ist aber elementarer. 
	
	Die Ungleichung gilt für $n=1$, denn $1 \le 2^1$. Sei nun $n \in \N$ ein Wert, für welchen $n \le 2^n$ gilt. Im Induktionsschritt sollen wir nun $n + 1 \le 2^{n+1}$ herleiten. Da wir $n \le 2^n$ voraussetzen, gilt $n+1 \le 2^n + 1$, daher reicht es zu verifizieren, dass $2^n + 1 \le 2^{n+1}$ erfüllt ist. Das letztere ist Äquivalent zur Ungleichung $2^n \ge 1$, die trivialerweise für $n \in \N$ erfüllt ist. 
\end{proof} 

\begin{aufg} 
	Zeigen Sie, dass $100 n \le 2^n$ für alle $n \in \N$ mit $n \ge 10$ erfüllt ist. 
\end{aufg} 


\begin{thm}[Vollständige Induktion, Version~2]
	\label{thm:ind:ver2}
	Sei $P$ ein Prädikat auf  $\N$. Dann sind die folgenden Bedingungen äquivalent: 
	\begin{enuma}
			\item $P(n)$ gilt für alle $n \in \N$. 
			\item es gilt $P(1)$ und, für jedes $n \in \N$, gilt die Implikation 
			\[
				P(1) \wedge \cdots \wedge P(n)  \Rightarrow P(n+1).
			\] 
	\end{enuma} 
\end{thm} 
\begin{proof} 
	Es gibt zwei einfache Weisen, diese Version der vollständigen Induktion herzuleiten. Zum einen kann man den Beweis von Theorem~\ref{thm:ind} sehr geringfügig modifizieren, um dieses Theorem herzuleiten. Zum anderen kann man die Behauptung von Theorem~\ref{thm:ind} für das Prädikat $Q(n) := P(1) \wedge \cdots \wedge P(n)$ benutzen. 
\end{proof} 

\begin{thm}[Primfaktorzerlegung - Existenz] 
	Für jedes $n \in \N$ existieren  Primzahlen $p_1,\ldots,p_t$ ($t \in \N_0$), deren Produkt gleich $n$ ist, d.h.:  
	\[
		n = \prod_{i=1}^t p_i.
	\] 
\end{thm} 

\begin{bem}
	Kommentar zur  vorigen Behauptung: man hat $\prod_{i=1}^0 p_i = 1$ im Fall $t=0$ und $\prod_{i=1}^1 p_i = p_1$ im Fall $t=0$. 
\end{bem} 


\begin{proof} 
	Die Behauptung ``es existieren $t \in \N_0$ Primzahlen $p_1,\ldots,p_t$ mit $n=\prod_{i=1}^t p_i$'' ist wahr für $n \in\{1,2\}$. Denn $n=1$ ist Produkt $1= \prod_{i=1}^0 p_i = 1$ und im Fall $n=2$ ist $2 = \prod_{i=1}^1 p_i$ mit $p_1:=1$.
	
	Sei nun $n \in \N$ mit $n \ge 3$ so, dass jede Zahl $a \in \{1,\ldots,n-1\}$ Produkt von endlich vielen Primzahlen ist (im Sinne der Behauptung). Ist $n$ Primzahl, so gilt die Behauptung mit $t=1$ und $p_1 = n$. Ist $n$ keine Primzahl, so besitzt $n$ einen Teiler $a \in \{2,\ldots,n-1\}$. Es folgt $n = a b$ mit $b = n / a \in \N$ und $b \le \frac{n}{2} \le n-1$. Die Anwendung der Induktionsvoraussetzung zu $a$ und $b$ ergibt, dass man $a$ sowie $b$ als Produkt von Primzahlen darstellen hat. Es gilt also 
	\begin{align*}
			a & = \prod_{i=1}^{r} u_i, 
		\\	b & = \prod_{i=1}^s  v_i.
	\end{align*} 
	mit $r, s \in \N$ für geweisse Primzahlen $u_1,\ldots,u_r,v_1,\ldots,v_s$ (hier ist weder $r$ noch $s$ gleich $0$, denn $a,b \ge 2$). 
	Dann ist $n$ Produkt von $t = r+s$ Primzahlen $p_1,\ldots,p_t$ mit $p_i = u_i$ für $i \in \{1,\ldots,r\}$ und $p_i = v_{i-r}$ für $i \in \{r+1,\ldots,r+s\}$. 
\end{proof} 

\begin{bem}[Fallunterscheidung] 
	Ein weiteres verbreitetes Element eines Beweises ist die Fallunterscheidung. Im vorigen Beweis haben wir z.B. zwischen den Fällen, dass $n$ eine Primzahl und $n$ keine Primzahl ist, unterschieden. In jedem der beiden Fällen gaben wir ein eigenes Argument, wieso $n$ in die Primfaktoren zerlegbar ist.  
\end{bem} 

\begin{bem}
	Im vorigen Beweis setzen wir das Muster $P(1) \wedge \cdots \wedge P(n) \Rightarrow P(n+1)$ aus Theorem~\ref{thm:ind:ver2} als $P(1) \wedge \cdots \wedge P(n-1) \Rightarrow P(n)$ um. Das heißt, in Theorem~\ref{thm:ind:ver2} werden die schon abgearbeiteten Werte als $1,\ldots,n$ bezeichnet und der nächste Wert als $n+1$. In unserer Umsetzung werden die abgearbeiteten Werte als $1,\ldots,n-1$ bezeichnet und der nächste Wert als $n$. 
\end{bem} 


\begin{bem}[Vollständige Induktion und Algorithmen]
	Induktionsbeweise haben oft eine algorithmische Interpretation und können zu iterativen oder rekursiven Algorithmen konvertiert werden. In diesem Fall, kann man sich die folgende Umsetzung des vorigen Beweises als Algorithmus vorstellen. Nehmen wir an, wir wollen für alle Zahlen $1,\ldots,N$ für ein gegebenes $N \in \N$, $N \ge 2$, deren Zerlegung in Primfaktoren berechnen. Dann können wir folgendermaßen vorgehen. Wir führen für jedes $n \in \{1,\ldots,N\}$, eine Liste $L[i]$ der Primfaktoren von $1,\ldots,N$ ein. Für $n=1$ ist dies eine leere Liste. Angenommen, die Liste sei bereits für die Zahlen $1,\ldots,n-1$ erzeugt. Dann können wir die Liste $L[n]$ anhand der bereits vorhanden Listen $L[1],\ldots,L[n-1]$ so generieren. Wir testen, ob $n$ Primzahl ist. Ist das der Fall so erzeugen wir $L[n]$ als Liste aus einer einzigen Zahl $n$. Ist $n$ keine Primzahl, so Faktorisieren wir $n$ als $n = ab$ mit $a,b \in \{2,\ldots,n-1\}$ und erzeugen dann $L[n]$ durch das Zusammenfügen der Listen $L[a]$ und $L[b]$. 
	
	\lstinputlisting{code/prime_factorizations.sage}
	
\end{bem} 

\section{Schnupperstunde in Algebra} 

\subsection{Was ist Algebra?}

\begin{bem}
	Algebra ist die Theorie algebraischer Strukturen. Während man in der Schule mit einer relativ kleiner Anzahl algebraischer Strukturen wie $(\R,+,\cdot)$ oder dem Vektorraum $\R^3$ arbeitet, befasst man sich in Algebra mit verschiedenen Kategorien algebraischer Strukturen, wie z.B. Halbgruppen, Gruppen, Ringe, Körper und Vektorräume. 

Man entwickelt auch Mittel,  neue/eigene algebraische Strukturen anzulegen. Wenn man diesen Prozess mit der Programmierung vergleicht, so ist der Prozess sehr ähnlich zur Entwicklung eigener Datenstrukturen (im Gegensatz zur Nutzung der standardmäßig vorhandenen Datenstrukturen). 
\end{bem} 

\begin{bem}[Algebraische Struktur] 
	Eine algebraische Struktur ist in der Regel eine Menge $A$, die mit einer oder mehreren Verknüpfungen ausgestattet ist. In den allermeisten Fällen sind die Verknüpfungen, die man betrachtet, binär: sie sind Abbildungen $\ast : A \times A \rightarrow A$.  Für solche Abbildungen schreibt man dann $a \ast b$ an der Stelle von $\ast(a,b)$. Sehr oft handelt es sich auch um die Verknüpfungen, für welche (zumindest) das Assoziativgesetz $a \ast (b \ast c) = (a \ast b) \ast c$ erfüllt ist. 
\end{bem}	
	
\begin{bem}[Polymorphismus in Algebra] 
	Man benutzt oft zum Bezeichnen der Verknüpfungen (bzw. der Verknüpfung) einer algebraischen Struktur die Symbole $+$ (Plus) und $\cdot$ (Mal). Hierbei meint man dann die Plus-Operation bzw. die Mal-Operation innerhalb der gegebenen algebraischen Struktur $A$. Das heißt, diese Operationen müssen $+$ und/oder $\cdot$ innerhalb einer algebraische Struktur $A$ nicht unbedingt mit Operation $+$ und $\cdot$ innerhalb der Menge $\R$ der reellen Zahlen etwas zu tun haben. Das bedeutet:  genau so, wie Symbole $a,b,c,d,\ldots$ in Mathematik kontextabhängig sind (können verschiedene Bedeutung in verschiedenen Kontexten haben), sind auch die Bezeichnungen wie $+$ und $\cdot$  kontextabhängig (bzw. Strukturabhängig) und können so, wie man es sich wünscht, eingeführt werden. Wenn man also $+$ in der Struktur $A$ hat, so ist das streng genommen $+_A$ -- die Plusoperation aus der Struktur $A$ -- man schreibt aber einfach nur $+$ und nimmt stillschweigend  an, dass es aus dem Kontext klar ist, welche Struktur $A$ gemeint ist. Die Nutzung der selben Bezeichnungen für verschiedene Operationen nennt man in der Programmierung der Polymorphismus.
\end{bem} 

\begin{bsp}
	Für $n \in \N$ heißt die Menge $S_n$ aller bijektiven Abbildungen von $\{1,\ldots,n\}$ nach $\{1,\ldots,n\}$ mit der Multiplikation 
	\[(\sigma \cdot \tau )(i) := \sigma(\tau(i))
	\] die symmetrische Gruppe. Was eine (allgemeine) Gruppe ist, wird in IT-2 diskutiert. 
\end{bsp} 

\begin{bsp} 
	Die algebraische Struktur $\F_2$, welche man als Menge  $\{0,1\}$ mit den binären Operationen 
\begin{align*}
\begin{array}{c|cc}
	+ & 0 & 1 \\
	\hline 
	0 & 0 & 1 \\
	1 & 1 & 0
\end{array}
& & \text{und} & & 
\begin{array}{c|cc}
\cdot & 0 & 1 \\
\hline 
0 & 0 & 0 \\
1 & 0 & 1
\end{array}
\end{align*} 
einführt, ist ein sogenannter binärer Körper. Die Bezeichnungen $+$, $\cdot$, $0$ und $1$, die wir hier verwenden, sind polymorph. 

Wir meinen $+_{\F_2}, \cdot_{\F_2}$, $0_{\F_2}$ und $1_{\F_2}$ schreiben aber in unserem Kontext von $\F_2$ vereinfachend $+, \cdot, 0, 1$. 

Der binäre Körper spielt in der Kodierungstheorie und der Kryptographie eine wichtige Rolle. 

\end{bsp} 


\subsection{Kommutativer Ring} 

\begin{defn}
	Eine Menge $R$ mit zwei binären Verknünfungen $+, -$ und zwei verschiedenen ausgezeichneten Elementen $0, 1 \in R$ heißt kommutativer Ring, wenn für alle $a,b,c \in R$ Folgendes erfüllt ist: 
	\begin{itemize}
		\item $a + b =b +a$ und $a \cdot b = b \cdot a$ 
		\item $a + 0 = a$ und $a \cdot 1 = a$ 
		\item $(a+b)+c = a+(b+c)$ und $a \cdot (b \cdot c) = (a \cdot b) \cdot c$
		\item Zu jedem $a$ gibt es ein eindeutiges Element aus $R$, das man als $-a$ bezeichnet, für welches $a+(-a)=0$ erfüllt ist. 
		\item $a \cdot (b+c) = a \cdot b + a \cdot c$
	\end{itemize} 
\end{defn}

\begin{aufg} 
	Ist $R$ kommutativer Ring mit $1$, dann gilt $a \cdot 0=0$ für alle $a \in R$. Zeigen Sie das. 
\end{aufg} 

\begin{bsp}\ 
\begin{itemize}
		\item $(\N,+,\cdot)$ kein Ring. 
		\item $(\N_0,+,\cdot)$ (immer noch) kein Ring. 
		\item $(\Z,+,\cdot)$ ein kommutativer Ring. 
		\item $(\Q,+,\cdot)$ ein kommutativer Ring. 
		\item $(\R,+,\cdot)$ ein kommutativer Ring. 
		\item $(\C,+,\cdot)$ ein kommutativer Ring. 
\end{itemize} 
\end{bsp} 

\subsection{Körper} 

\begin{defn}
	Eine Menge $K$ mit zwei binären Verknüpfungen $+$ und $\cdot$ heißt Körper, wenn $K$ bzgl. $+$ und $\cdot$ kommutativer Ring ist und darüber hinaus für jedes $a \in K \setminus \{0\}$ ein eindeutiges Element $a^{-1} \in K$ existiert, für welches $a \cdot a^{-1}  = 1$ gilt. 
\end{defn} 

\begin{bsp}\ 
\begin{itemize} 
	\item $(\F_2,+,\cdot)$
	\item Führen Sie auf einer dreielementigen Menge $\{0,1,a\}$ die Verknüpfungen $+$ und $\cdot$ so ein, dass die Menge mit diesen Verknüpfungen zu einem Körper wird. 
	\item $(\Z,+,\cdot)$ kein Körper, da in $\Z \setminus \{0\}$ nichts außer $-1$ und $1$ invertierbar ist. 
	\item $(\Q,+,\cdot)$ ein Körper. 
	\item $(\R, + ,\cdot)$ ein Körper. 
	\item $(\C, + ,\cdot)$ ein Körper. 
\end{itemize} 
\end{bsp} 

\begin{defn}
	Ein Körper $K$ heißt algebraisch abgeschlossen, wenn für jede Wahl von $d \in \N$ und alle $a_d \in K \setminus \{0\}, a_{d-1},\ldots,a_0 \in K$ die Gleichung 
	\[
	a_d x^d + a_{d-1} x^{d-1} + \cdots + a_0 = 0
	\]
	mindestens eine Lösung $x$ aus $K$ besitzt. Eine Gleichung wie oben nennt man Polynomgleichung vom Grad $d$ mit Koeffizienten in $K$. 
\end{defn} 

\begin{bsp}\ 
	\begin{itemize} 
		\item $\Q$ ist nicht algebraisch abgeschlossen, vgl. die Gleichung $x^2 - 2 = 0$, mit den Koeffizienten $ -2, 0 , 1 \in \Q$, die keine Lösung $x$ in $\Q$ besitzt. 
		\item $\R$ ist nicht algebraisch abgeschlossen, vgl. die Gleichung $x^2 + 1 = 0$ mit den Koeffizienten $1, 0, 1 \in \R$, die keine Lösung $x$ in $\R$ besitzt. 
	\end{itemize} 
\end{bsp} 

\begin{defn} 
	Sind $A$ und $B$ Mengen mit $A \subseteq B$ und $\ast_A : A \times A \to A$ und $\ast_B : B \times B \to B$ binäre Verknüpfungen, so nennt man $\ast_B$ Erweiterung von $\ast_A$ und $\ast_A$ Einschränkung von $\ast_B$ auf $A$, wenn $x \ast_A y = x \ast_B y$ für alle $a,b \in A$ erfüllt ist (mit anderen Worten: $\ast_B$ wirkt genau so wie $\ast_A$ innerhalb von $A$). 
\end{defn} 

\begin{defn}
	Sind $(F,+,\cdot)$ und $(K,+,\cdot)$ Körper mit $F \subseteq K$, bei denen $+$ und $\cdot$ von $K$ Erweiterungen von $+$ bzw. $\cdot$ auf $F$ sind, so nennt man den Körper $K$ eine Erweiterung des Körpers $F$. 
\end{defn} 

\begin{bsp}
	$\R$ ist Erweiterung von $\Q$. Es gibt aber viele Körper dazwischen. Zum Beispiel ist 
	\[
		\Q[\sqrt{2}] := \setcond{ a + \sqrt{2} b }{a,b \in \Q}
	\]
	ebenfalls ein Körper. Es gilt $\Q \varsubsetneq \Q[\sqrt{2}] \varsubsetneq \R$. 
	Wie sieht das inverse eines Elements aus $\Q[\sqrt{2} ] \setminus \{0\}$ aus? 
\end{bsp} 

\begin{thm}
	Jeder Körper besitzt eine algebraisch abgeschlossene Körpererweiterung. 	
\end{thm} 

\begin{bem}
	Es gilt sogar eine stärkere Aussage: jeder Körper eine (im einem bestimmten Sinn) minimale algebraisch abgeschlossene Körpererweiterung. 
\end{bem} 

\subsection{Der Körper der komplexen Zahlen} 

\begin{defn} 
	Die Menge $\C$ der komplexen Zahlen führen wir als die Menge der formalen Ausdrücke der Form $x +  y \, \iu$ mit $x, y \in \R$ ein. Hierbei ist $\iu$ ein formales Element, für welches wir $\iu^2 := -1$ festlegen. Das Element $\iu$ nennt man die \emph{imaginäre Einheit} oder die \emph{Wurzel aus $-1$}. Die Menge der reellen Zahlen $\R$ wird als eine Teilmenge von $\C$ aufgefasst, indem man $x \in \R$ als $x +  y  \, \iu$ mit $y=0$ schreibt. 
	
	Nach diesen Festlegungen lassen sich die Operationen $+$ und $\cdot$ vom Körper $\R$ der reellen Zahlen auf $\C$ auf eine eindeutige Weise erweitern, wenn man fordert, dass  $\C$ mit Operationen $+$ und $\cdot$ ein kommutativer Ring sein soll, vgl. dazu die Gesetze für einen kommutativen Ring.  (Wie wir in Kürze sehen werden, ist $(\C,+,\cdot)$ sogar ein Körper.) Die Addition und Multiplikation führen wir also auf die folgende Weise eingeführt: 
	\begin{align*}
			(x_1 + y_1 \iu ) + (x_2 +  y_2 \iu ) & := (x_1 +y_1) +  (y_1 + y_2) \iu
			\\ (x_1 + y_1 \iu) \cdot (x_2 + y_2 \iu) & := (x_1 x_2 - y_1 y_2) + (x_1 y_2 + x_2 y_1) \iu,
	\end{align*} 
für $x_1,x_2, y_1, y_2 \in \R$. 

Ist $z = x  + y \iu$ mit $x, y \in \R$ so führen, wir den Realteil von $z$ als $\Re(z) :=x$ und den Imaginärteil von $z$ als $\Im(z):= y$ ein; die Zahl $\overline{z} = x - y \iu$ nennen wir komplex konjugiert zu $z$; den Wert $|z| = \sqrt{x^2 + y^2}$ nennen wir den Betrag von $z$. 
\end{defn} 

\begin{aufg}
	Berechnen Sie $(3+ 2 \iu) ( 5 - \iu)$, indem Sie eine Darstellung dieser Zahl als $x+y \iu$ mit $x, y \in \R$ bestimmen. 
\end{aufg} 

\begin{bem}
	In Algebra werden oft Strukturen formal nach ``eigenen Vorgaben'' eingeführt. Bei der Definition von komplexen Zahlen sieht man ein Beispiel dafür. 
\end{bem} 

\begin{thm}
		$\C$ ist ein algebraisch abgeschlossener Körper. 
\end{thm}
\begin{proof} 
	Dass $(\C,+,\cdot)$ ein kommutativer Ring ist, lässt sich direkt verifizieren (Aufgabe). 
	
	Um zu zeigen, dass $(\C,+,\cdot)$ sogar ein Körper ist, muss man verifizieren, dass jedes $z = x + y \iu$ mit $x, y \in \R$ mit $|z| \ne 0$ ein inverses Element in $\C$ besitzt. Es stellt sich heraus, dass man das inverse Element $z^{-1}$ als $z = \frac{1}{|z|^2} \bar{z}$ beschreiben kann. Mit der Verwendung der dritten binomischen Formel erhalten wir 
	
	\[
		 	z z^{-1} = \frac{z \bar{z} }{|z|^2} = \frac{ ( x+ y \iu) ( x - y\iu) }{x^2 + y^2}   = \frac{ x^2 - (y \iu)^2 }{x^2 + y^2} = \frac{x^2 - y^2 \iu^2}{x^2 + y^2} = \frac{x^2 + y^2}{x^2 + y^2} = 1. 
	\]
	
	Dass der Körper $(\C,+,\cdot)$ algebraisch abgeschlossen ist, ist ziemlich bemerkenswert. Bedenken Sie, dass wir nur die imaginäre Einheit $\iu$ eine formale Lösung der Polynomgleichung $z^2 + 1=0$ in einem unbekannten $z$ eingeführt haben. Die Behauptung über die algebraische Abgeschlossenheit ist, dass wird durch diese Ergänzung für eine beliebige Polynomgleichung von einem positiven Grad (un d mit Koeffizienten in $\C$) eine Lösung in $\C$ finden. Um diese Behauptung herzuleiten braucht man wissen aus der Analysis (wir geben also an dieser Stelle keinen Beweis). 
\end{proof} 

\begin{aufg}
	Zeigen Sie $|u \cdot v| = |u| \cdot |v|$ für alle $u, v \in \C$. \textbf{Hinweis:} Am besten zeigt man $|u \cdot v|^2 = |u|^2  \cdot |v|^2$, um die Wurzeln zu vermeiden. 
\end{aufg} 

\begin{bem}[Der Satz von Pythagoras, Radianten und Grade, Kosinus und Sinus, und der  Taschenrechner] 
	Für das nachfolgende Thema soll man zuerst an einige Themen aus dem Mathe-Unterricht in der Schule erinnern. 
	\begin{itemize}
		\item \emph{Der Satz des Pythagoras}. Der Abstand zwischen dem Punkt $(0,0) \in \R^2$ und einem Punkt $(x,y) \in \R^2$ ist gleich $\sqrt{x^2 + y^2}$. Wenn man diese Behauptung in einer koordinaten-freien Form mit Hilfe von rechtwinkligen Dreiecken formuliert, so nennt man sie den Satz des Pythagoras. 
		\item \emph{Radianten und Grade.} Im alten Babylonien dachte man, das Jahr wäre 360 Jahre lang (das stimmt nicht, wie wir jetzt wessen). Daher teilte man den Jahreskreis in 360 Teile auf, die den Tagen entsprechen. Ein Grad steht daher für einen Tag im babylonischen Jahreskreis. Das zeigt, dass die Herkunft der Messung der Winkel in Graden nicht mathematisch ist. Sie ist anthropologisch: sie hängt mit dem Planeten Erde zusammen, auf dem wir uns befinden, und mit den Babylonier:innen, die bei der Bestimmung der Anzahl der Tage im Jahr sich ein Wenig verschätzten. Dennoch hat sich die Messung mit 360 Graden für den vollen Winkeln bis jetzt erhalten. Das liegt vielleicht daran, dass einige für uns interessante Winkel mit Graden durch eine ganze Zahl darstellbar sind ($90^\circ$, $60^\circ$, $30^\circ$). Die Messung mit Radianten ist eine dimensionslose Messung und sie ist intrinsisch mathematisch. Man nimmt einen Kreis mit dem Radius $1$ und misst Winkel durch die Längen der Bögen dieses Kreises. Dabei bezeichnet man die Länge einer Hälfte des Einheitskreises als $\pi$ und nennt die Zahl $\pi$ die Kreiszahl. Diese Zahl $\pi$ ist etwas größer als $3$ (das sieht man, wenn man in den Einheitskreis ein reguläres Sechseck einschreibt). 
		
		Ein Grad ist nichts Anderes als $1^\circ := \frac{\pi}{180}= \frac{2\pi}{360}.$ Wenn man Winkel in Radianten misst, kann man etwa $1{.}2$ Radianten aber auch einfach nur $1{.}2$ sagen, denn die Einheit Radiant ist dimensionslos. 
		
		An sich gibt es an der Messung der Winkel mit Graden nichts Falsches. Dieser Kommentar dient einfach nur dazu, darauf hinzuweisen, dass die Zahlen wie $360$ und $180$ in Bezug auf die Winkelmessung keine mathematische sondern eine anthropologische Natur haben. 
		
		\item \emph{Kosinus und Sinus.}  Man betrachte eine kreisförmige Radrennbahn mit Zentrum im Punkt $(0,0)$ vom Radius $1$. Diese Bahn ist nach dem Satz des Pythagoras durch die Gleichung $x^2 + y^2 = 1$ beschrieben. Nun legen wir den Punkt $(1,0)$ dieser Bahn als den Startpunkt fest. Von diesem Punkt aus kann man nun Strecken einer beliebigen Länge zurücklegen. Wie lang die Strecke ist und ob man sich im Gegenuhrzeiger oder im Uhrzeigersinn bewegt wird durch eine Zahl $\alpha \in \R$ notiert. Der Betrag von $\alpha$ gibt die Länge der Strecke an, die man zurücklegen will. Das Vorzeichen von $\alpha$ gibt an, ob man sich im Gegenuhrzeigersinn oder im Uhrzeigersinn bewegt (bei einem positiven Vorzeichen - im Gegenuhrzeigersinn und bei einem negativen Vorzeichen - im Uhrzeigersinn). Für jedes $\alpha \in \R$ erhält man einen Punkt $(x,y)$, in dem man sich nach dem Zurücklegen der vorgegebenen Strecke in die vorgegebene Richtung landet. Die $x$-Komponente dieses Punkts nennt man den Kosinus von $\alpha$ (Bezeichnung: $x = \cos \alpha$) und die $y$-Komponente dieses Punkts nennt man den Sinus von $\alpha$ (Bezeichnung: $y = \sin \alpha$). 
		
		Die Eingabe für $\cos$ und $\sin$ ist also eine reelle Zahl und die Rückgabe ist oben beschrieben. 
		\item \emph{Kosinus und Sinus im Taschenrechner}. Wenn die Studierenden den Kosinus und Sinus (unter anderem für sehr einfache Werte $\alpha$) im Taschenrechner berechnen, so sieht man, dass es immer wieder dazu kommt, dass ihre Ergebnisse falsch sind. Das liegt daran, dass man in vielen Taschenrechnern eine Umschaltung zwischen Grad und Radianten hat. Ist der Taschenrechner auf Radianten eingestellt, so berechnet er die eigentlichen  Kosinus und Sinus, wie sie in Mathematik (und in den meisten Programmiersprachen) zu finden sind. Ist der Taschenrechner auf Grade eingestellt, so berechnet er die Funktionen $t \mapsto \cos( \frac{\pi}{180}t)$ und $t \mapsto \sin( \frac{\pi}{180})t$ an der Stelle von $\cos$ und $\sin$. Übrigens: in Excel wird die Funktion $t \mapsto \frac{\pi}{180} t$, die oben in $\cos$ und $\sin$ eingesetzt wurde, das Bogenmaß von $t$ genannt. 
		\item Die vielen Formeln, die man für den Kosinus und Sinus und andere trigonometrische Funktionen hat, lassen sich im Rahmen der linearen Algebra (IT-3) viel besser verstehen. 
	\end{itemize} 
\end{bem} 

\begin{thm}
	Jede komplexe Zahl $z \in \C$ besitzt eine Darstellung als 
	\[
		z = \rho ( \cos \phi + \iu \sin \phi )
	\]
	mit $\rho \in \R_{\ge 0}$ und $\phi \in \R$. Hierbei gilt $\rho = |z|$. Bei $z \ne 0$, ist $\phi$ eindeutig durch $z$ bis auf das addieren eines ganzzahligen Vielfachen von $2 \pi$ definiert. 
\end{thm}

\begin{proof}  
	{\color{red} WARNUNG:} Der nachfolgende Beweis und unsere Definition von $\cos$ und $\sin$ entspricht nicht ganz den mathematischen Standards, solange wir den Begriff  Länge (eines Bogens) und Orientierung (einer Kurve), auf den  wir uns bei der Einführung von $\cos$ und $\sin$ beziehen, nicht mathematisch formal definiert haben. Wir verlassen uns also auf Intuition und darauf, dass man (später) den Begriff Länge mathematisch korrekt einführen kann (solche Begriffe führt man in der Analysis ein). Es gibt auch einen formalen nicht-geometrischen Zugang zum Kosinus und Sinus (dieser Zugang ist aber nicht wirklich intuitiv, sodass man dadurch nicht wirklich versteht, was Kosinus und Sinus eigentlich sind). 
	
	Da jede komplexe Zahl $z = x + y \iu$ eindeutig durch $x, y \in \R$ gegeben ist, kann man $z$ als einen Punkt $(x,y) \in \R^2$ visualisieren. Die Visualisierung von $\C$ auf diese Weise nennt man die gaußsche Zahlenebene. Dabei werden $1$ und $\iu$ als die zueinander senkrechte Vektoren $(1,0)$ und $(0,1)$ dargestellt. Man sieht, dass die Menge $K := \setcond{z \in \C}{|z|=1} = \setcond{ x+ \iu y}{x^2 + y^2 =1}$ als  der Einheitskreis mit Zentrum in $0 \in \C$ und dem Radius $1$ in der gaußschen Zahlenebene darstellbar ist. 
	
	\emph{Existenz:} Ist $z \ne 0$,so ist $z / |z|$ ist ein Punkt im Kreis $K$ und so hat $z$ die Darstellung $ z / |z| = \cos \phi + \iu \sin \phi$ für ein $\phi \in \R$ nach unserer Beschreibung von $\cos$ und $\sin$. Es folgt also, dass $z = \rho ( \cos \phi + \iu \sin \phi)$ mit $\rho = |z|$ gilt. Im Fall $z= 0 \in \C$ kann man $\rho =0$ und ein beliebiges $\phi$ fixieren. 
	
	\emph{Eindeutigkeit:} Ist $z = \rho (\cos \phi + \iu \sin \phi)$ mit $\rho \in \R_{\ge 0}$ und $\phi \in \R$ so gilt $|z| = | \rho (\cos \phi + \iu \sin \phi) | = \rho | \cos \phi + \iu \sin \phi| = \rho \sqrt{ \cos^2 \phi + \sin^2\phi } = \rho$. Ist $z \ne 0$, so ist 
	$ z/ |z|$ der Punkt $\cos \phi + \iu \sin \phi$ auf Einheitskreis $K$. Der Punkt $\cos \phi + \iu \sin \phi$ im Kreis $K$ ändert sich nicht, wenn man zum Wert von $\phi$  ein ganzzahliges Vielfaches von $2 \pi$ dazu addiert, weil der Kreis $K$ die Länge $2\pi$ hat. So besteht die Möglichkeit als $\phi$ einen Wert aus $[0,2 \pi)$ zu wählen. 
	
	 Da $K$ die Länge $2\pi$ hat, ist jeder Punkt eindeutig durch die Angabe eines solchen $\phi \in [0,2 \pi)$ gegeben. 
\end{proof} 

\begin{defn}
	Wir erweitern die Exponentialfunktion $e^x$ auf $\R$ auf den Bereich $\C$ der komplexen Zahlen, in dem wir 
	\[
		e^{x+ i y} := e^x ( \cos y + i \sin y)
	\]
	für alle $x, y \in \R$ festlegen. (Insbesondere, $e^{i y} = \cos y + i \sin y$). 
\end{defn} 

\begin{bem}
	Jede Zahl $z \in \C$ besitzt eine Darstellung $z = \rho e^{i \phi}$ mit $\rho = |z|$ und $\phi \in \R$. 
\end{bem} 

\section{Asymptotische Notation}

\subsection{O, $\Omega$ und $\Theta$}

\begin{bem}
Bei der Analyse von Algorithmen und der Analysis redet man oft von der Größenordnung von Funktionen. Eine praktische Ausdrucksweise dafür ist die sogenannte asymptotische Notation.
\end{bem} 

\begin{defn}[$O$-Notation]  
Seien $f, g: \N \to \R$ Funktionen. 
Man schreibt $f(n) = O(g(n))$, wenn eine Konstante $c>0$ und ein $n_0 \in \N$ existiert, so dass $|f(n)| \le c |g(n)|$ für alle $n \ge n_0$ gilt. 
\end{defn} 

\begin{bem} 
Die Bezeichnung $f(n)=O(g(n))$ steht für ``$f(n)$ hat die Größenordnung höchstens $g(n)$ bis auf eine mutliplikative Konstante'' und man sagt \glqq$f(n)$ ist in Groß-O von $g(n)$\grqq. Die Schreibweise $f(n) = O(g(n))$ ist streng genommen nicht ganz korrekt, in der Literatur aber sehr verbreitet. Die korrekte Schreibweise wäre $f(n) \in O(g(n))$, d.h., $f(n)$ liegt in der Menge aller Funktionen der Größenordnung höchstens $g(n)$. In der Literatur verwendet man oft $O(g(n))$ als eine Schreibweise für eine anonyme Funktion der Größenordnung höchstens $g(n)$. In diesem Kurs spielen die Beträge in der Definition von $O(g(n))$ in der Regel keine Rolle, weil wir beim Anwenden der asymptotischen Notationen fast ausschließlich nichtnegative Funktionen benutzen. 
\end{bem} 

\begin{defn}[$\Omega$-Notation] 
Man schreibt $f(n) = \Omega(g(n))$, wenn eine Konstante $c>0$ und ein $n_0 \in \N$ existieren, so dass $|f(n)| \ge c |g(n)|$ für alle $n \ge n_0$ gilt. In diesem Fall: Die Größenordnung von $f(n)$ ist mindestens $g(n)$, bis auf eine multiplikative Konstante und man sagt \glqq$f(n)$ ist in Groß-Omega von $g(n)$\grqq. 
\end{defn} 

\begin{defn} 
Man schreibt $f(n) = \Theta(g(n))$, wenn sowohl $f(n) = O(g(n))$ als auch$f(n) = \Omega(g(n))$ gelten.
\end{defn} 

\begin{bem}
In diesem Fall: Die Größenordnung von $f(n)$ ist genau $g(n)$ bis auf eine multiplikative Konstante, und man sagt \glqq$f(n)$ ist in Groß-Theta von $g(n)$\grqq.
\end{bem} 

\begin{bem}
Die asymptotischen Notationen $O(g(n))$, $\Omega(g(n))$ und $\Theta(g(n))$ (und ihre weiteren Varianten) werden oft auch Landau-Symbole genannt.
\end{bem} 

\begin{bsp}
	Sei $f : \N \to \R$ definiert durch $f(n):=\sqrt{2 n + 5 } - 10$. Es gilt $f(n) = \Theta(\sqrt{n})$, denn einerseits ist $\sqrt{2n + 5} - 10 \le \sqrt{2n + 5} \le \sqrt{ 7n} = \sqrt{7} \sqrt{n}$ für alle $n \in \N$, woraus $f(n) = O(\sqrt{n})$ folgt. Andererseits ist $\sqrt{2n + 5} - 10 \ge \sqrt{n} - 10 \ge \frac{1}{2} \sqrt{n}$ für alle $n \ge 400$, woraus $f(n) = \Omega(\sqrt{n})$ folgt.  
\end{bsp}

\begin{aufg}
	Sind die folgenden asymptotischen Abschätzungen richtig?
	\begin{itemize}
		\item $n! = O(n^n)$
		\item $n^n = \Omega(n!)$
		\item $n! = O(2^n)$
		\item $n^n = O(n!)$
	\end{itemize}
\end{aufg}


\begin{bem}
	Seien $f_1,f_2,g_1,g_2 : \N \to \R$ Funktionen, wobei $g_1,g_2$ nicht-negativ sind und $f_i(n) = O(g_i(n))$, für $i=1,2$, vorausgesetzt wird. Dann gilt
	\[
	f_1(n) + f_2(n) = O(g_1(n)+g_2(n)) = O(\max\{g_1(n),g_2(n)\}),
	\]
	und
	\[
	f_1(n) \cdot f_2(n) = O(g_1(n)\cdot g_2(n)).
	\]
\end{bem}

\subsection{$o$ und $\omega$}

\begin{defn} 
Bei $g : \N \to \R$ steht $o(f(n))$ für die Menge aller Funktionen $f: \N \to \R$ mit der Eigenschaft, dass für jedes $c>0$ ein $n_0 \in \N$ existiert derart, dass $|f(n)| \le c |g(n)|$ für alle $n \in \N$ mit $n \ge n_0$ erfüllt ist. In der Literatur schreibt man oft $f(n) = o(g(n))$ an der Stelle von $f(n) \in o(g(n))$. 
\end{defn} 

\begin{defn}
Die Bezeichnung $\omega(g(n))$ steht für die Menge aller Funktionen $f : \N \to \R$, für welche für alle $c>0$ ein $n_0 \in \N$ existiert derart, dass $|f(n)| \ge c|g(n)|$ für alle $n \in \N$ mit $n \ge n_0$ erfüllt ist. 
\end{defn} 
